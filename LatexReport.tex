\documentclass[12pt]{article}
\usepackage[utf8]{inputenc}
\usepackage{geometry}
\usepackage{graphicx}
\usepackage{float}
\usepackage{placeins}
\geometry{margin=1in}

\title{\textbf{Static and Dynamic Analysis}}
\author{ISTEC-Cyber Security}
\date{\today}  % Automatically insert today's date

\begin{document}

\maketitle

\section*{Description}
This report contains static and dynamic analysis of the target. It uses Semgrep for static analysis and OWASP ZAP, Nmap, and SQLMap for dynamic analysis.

\vspace{10cm} % Space to push the footer down

\begin{center}
	\textbf{Provided by} \\[1em]
	\includegraphics[width=0.1\textwidth]{logo75.png}
\end{center}

\newpage  % Page break

\section{Static Analysis}
Details about static analysis...

\section{Analysis Report}

\subsection{Risk Summary}
\begin{table}[h!]
    \centering
    \renewcommand{\arraystretch}{1.5}
    \begin{tabular}{|c|c|}
        \hline
        \textbf{Risk Level} & \textbf{Number of Findings} \\
        \hline
        Low Risk & lowcount \\ 
        \hline
        Medium Risk & mediumcount \\ 
        \hline
        High Risk & highcount \\ 
        \hline
        Critical Risk & criticalcount \\ 
        \hline
    \end{tabular}
    \caption{Summary of Risk Findings}
    \label{tab:risk_summary}

\subsection{Vulnerability Categories}
    \begin{itemize}
	\item Categories:
    \end{itemize}
\end{table}

\subsection{Vulnerabilities by Page}
%Vulnerabilities by Page:
\newpage
\section{Dynamic Analysis}
Details about dynamic analysis...

\end{document}