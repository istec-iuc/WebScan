
\documentclass[12pt]{article}
\usepackage[utf8]{inputenc}
\usepackage{geometry}
\usepackage{graphicx}
\usepackage{float}
\usepackage{placeins}
\usepackage{hyperref}
\usepackage{longtable}
\geometry{margin=1in}


\title{\textbf{Static and Dynamic Analysis}}
\author{ISTEC-Cyber Security}
\date{\today} % Automatically insert today's date

\begin{document}

\maketitle

\section*{Description}
This report contains static and dynamic analysis of the target. It uses Semgrep for static analysis and OWASP ZAP, Nmap, and SQLMap for dynamic analysis.

\vspace{10cm} % Space to push the footer down

\begin{center}
\textbf{Provided by} \\[1em]
\includegraphics[width=0.1\textwidth]{logo75.png}
\end{center}

\newpage % Page break

\section{Static Analysis}
Details about static analysis...

\section{Analysis Report}

\subsection{Risk Summary}
\begin{table}[h!]
\centering
\renewcommand{\arraystretch}{1.5}
\begin{tabular}{|c|c|}
\hline
\textbf{Risk Level} & \textbf{Number of Findings} \\
\hline
Low Risk & 42 \\ 
\hline
Medium Risk & 28 \\ 
\hline
High Risk & 0 \\ 
\hline
Critical Risk & 0 \\ 
\hline
\end{tabular}
\caption{Summary of Risk Findings}
\label{tab:risk_summary}
\end{table}

\subsection{Vulnerability Categories}
\begin{itemize}
\item Category 1: Denial-of-Service (DoS) | 28
\item Category 2: Improper Encoding | 4
\item Category 3: Mass Assignment | 4
\item Category 4: Cross-Site-Scripting (XSS) | 4
\item Category 5: Cryptographic Issues | 29
\item Category 6: Mishandled Sensitive Information | 1

% Insert vulnerability categories here
\end{itemize}

\subsection{Vulnerabilities by Page}
%Vulnerabilities by Page:

%--------------------------------------------------------Dynamic Analysis-----------------------------------------------
\newpage
\section{Dynamic Analysis}
Details about dynamic analysis...

\section{Analysis Report}

\subsection*{Nmap Scan Results}
Host: \textbf{scanme.nmap.org, scanme.nmap.org} (45.33.32.156) \\
Scan Duration: \textit {4.46 seconds} \\
\begin{center}

\begin{tabular}{|c|c|c|}
\hline
\textbf{Port} & \textbf{Service} & \textbf{State} \\
\hline
22 & ssh & \textbf{open} \\
25 & smtp & filtered (\textit{Reason: no-response}) \\
80 & http & \textbf{open} \\
9929 & nping-echo & \textbf{open} \\
31337 & Elite & \textbf{open} \\
\hline
\end{tabular}
\end{center}

\subsection*{SQLMap Injection Points}
Injection Point 1:
\begin{itemize}
    \item \textbf{Parameter:} artist (GET)
        \subitem Type: boolean-based blind
        \subitem Title: AND boolean-based blind - WHERE or HAVING clause
        \subitem 
        \subitem Type: time-based blind
        \subitem Title: MySQL >= 5.0.12 AND time-based blind (query SLEEP)
        \subitem 
        \subitem Type: UNION query
        \subitem Title: MySQL UNION query (NULL) - 3 columns
\end{itemize}

\subsection{Risk Summary}
\begin{table}[h!]
\centering
\renewcommand{\arraystretch}{1.5}
\begin{tabular}{|c|c|}
\hline
\textbf{Risk Level} & \textbf{Number of Findings} \\
\hline
Low Risk & 204 \\ 
\hline
Medium Risk & 130 \\ 
\hline
High Risk & 1 \\ 
\hline
Critical Risk & 0 \\ 
\hline
\end{tabular}
\caption{Summary of Risk Findings}
\label{tab:risk_summary}
\end{table}

\subsection{Vulnerability Categories}
\begin{itemize}
\item Category 1: Cross Site Scripting (DOM Based) | 1
\item Category 2: Absence of Anti-CSRF Tokens | 40
\item Category 3: Content Security Policy (CSP) Header Not Set | 47
\item Category 4: Missing Anti-clickjacking Header | 43
\item Category 5: Server Leaks Information via "X-Powered-By" HTTP Response Header Field(s) | 62
\item Category 6: Server Leaks Version Information via "Server" HTTP Response Header Field | 74
\item Category 7: X-Content-Type-Options Header Missing | 68
\item Category 8: Authentication Request Identified | 1
\item Category 9: Charset Mismatch (Header Versus Meta Content-Type Charset) | 31
\item Category 10: Information Disclosure - Suspicious Comments | 1
\item Category 11: Modern Web Application | 9
\item Category 12: User Controllable HTML Element Attribute (Potential XSS) | 3

% Insert vulnerability categories here
\end{itemize}

\subsection{Vulnerabilities by Page}
\section*{Site 1: http://testphp.vulnweb.com}
Host: testphp.vulnweb.com, Port: 80, SSL: false

\begin{center}
\renewcommand{\arraystretch}{1.3}
\begin{longtable}{|l|p{10cm}|}
\hline
\multicolumn{2}{|c|}{\textbf{Vulnerability 1}} \\
\hline
\textbf{Risk Level} & High (High) \\
\hline
\textbf{Vulnerability Name} & Cross Site Scripting (DOM Based) \\
\hline
\textbf{Description} & Cross-site Scripting (XSS) is an attack technique that involves echoing attacker-supplied code into a user's browser instance. A browser instance can be a standard web browser client, or a browser object embedded in a software product such as the browser within WinAmp, an RSS reader, or an email client. The code itself is usually written in HTML/JavaScript, but may also extend to VBScript, ActiveX, Java, Flash, or any other browser-supported technology.When an attacker gets a user's browser to execute his/her code, the code will run within the security context (or zone) of the hosting web site. With this level of privilege, the code has the ability to read, modify and transmit any sensitive data accessible by the browser. A Cross-site Scripted user could have his/her account hijacked (cookie theft), their browser redirected to another location, or possibly shown fraudulent content delivered by the web site they are visiting. Cross-site Scripting attacks essentially compromise the trust relationship between a user and the web site. Applications utilizing browser object instances which load content from the file system may execute code under the local machine zone allowing for system compromise.There are three types of Cross-site Scripting attacks: non-persistent, persistent and DOM-based.Non-persistent attacks and DOM-based attacks require a user to either visit a specially crafted link laced with malicious code, or visit a malicious web page containing a web form, which when posted to the vulnerable site, will mount the attack. Using a malicious form will oftentimes take place when the vulnerable resource only accepts HTTP POST requests. In such a case, the form can be submitted automatically, without the victim's knowledge (e.g. by using JavaScript). Upon clicking on the malicious link or submitting the malicious form, the XSS payload will get echoed back and will get interpreted by the user's browser and execute. Another technique to send almost arbitrary requests (GET and POST) is by using an embedded client, such as Adobe Flash.Persistent attacks occur when the malicious code is submitted to a web site where it's stored for a period of time. Examples of an attacker's favorite targets often include message board posts, web mail messages, and web chat software. The unsuspecting user is not required to interact with any additional site/link (e.g. an attacker site or a malicious link sent via email), just simply view the web page containing the code. \\
\hline
\textbf{Instances} & \textbf{URI} \\
\hline
Instance 1 & \url{http://testphp.vulnweb.com/\#jaVasCript:/*-/*\`/*\\`/*'/*"/**/(/* */oNcliCk=alert(5397) )//\%0D\%0A\%0d\%0a//</stYle/</titLe/</teXtarEa/</scRipt/--!>sVg/<sVg/oNloAd=alert(5397)//>} \\
\hline
\end{longtable}
\end{center}\vspace{0.7cm}
\begin{center}
\renewcommand{\arraystretch}{1.3}
\begin{longtable}{|l|p{10cm}|}
\hline
\multicolumn{2}{|c|}{\textbf{Vulnerability 2}} \\
\hline
\textbf{Risk Level} & Medium (Low) \\
\hline
\textbf{Vulnerability Name} & Absence of Anti-CSRF Tokens \\
\hline
\textbf{Description} & No Anti-CSRF tokens were found in a HTML submission form.A cross-site request forgery is an attack that involves forcing a victim to send an HTTP request to a target destination without their knowledge or intent in order to perform an action as the victim. The underlying cause is application functionality using predictable URL/form actions in a repeatable way. The nature of the attack is that CSRF exploits the trust that a web site has for a user. By contrast, cross-site scripting (XSS) exploits the trust that a user has for a web site. Like XSS, CSRF attacks are not necessarily cross-site, but they can be. Cross-site request forgery is also known as CSRF, XSRF, one-click attack, session riding, confused deputy, and sea surf.CSRF attacks are effective in a number of situations, including:    * The victim has an active session on the target site.    * The victim is authenticated via HTTP auth on the target site.    * The victim is on the same local network as the target site.CSRF has primarily been used to perform an action against a target site using the victim's privileges, but recent techniques have been discovered to disclose information by gaining access to the response. The risk of information disclosure is dramatically increased when the target site is vulnerable to XSS, because XSS can be used as a platform for CSRF, allowing the attack to operate within the bounds of the same-origin policy. \\
\hline
\textbf{Instances} & \textbf{URI} \\
\hline
Instance 1 & \url{http://testphp.vulnweb.com/} \\
\hline
Instance 2 & \url{http://testphp.vulnweb.com/artists.php} \\
\hline
Instance 3 & \url{http://testphp.vulnweb.com/artists.php?artist=1} \\
\hline
Instance 4 & \url{http://testphp.vulnweb.com/artists.php?artist=2} \\
\hline
Instance 5 & \url{http://testphp.vulnweb.com/artists.php?artist=3} \\
\hline
Instance 6 & \url{http://testphp.vulnweb.com/cart.php} \\
\hline
Instance 7 & \url{http://testphp.vulnweb.com/categories.php} \\
\hline
Instance 8 & \url{http://testphp.vulnweb.com/disclaimer.php} \\
\hline
Instance 9 & \url{http://testphp.vulnweb.com/guestbook.php} \\
\hline
Instance 10 & \url{http://testphp.vulnweb.com/guestbook.php} \\
\hline
Instance 11 & \url{http://testphp.vulnweb.com/index.php} \\
\hline
Instance 12 & \url{http://testphp.vulnweb.com/listproducts.php?artist=1} \\
\hline
Instance 13 & \url{http://testphp.vulnweb.com/listproducts.php?artist=2} \\
\hline
Instance 14 & \url{http://testphp.vulnweb.com/listproducts.php?artist=3} \\
\hline
Instance 15 & \url{http://testphp.vulnweb.com/listproducts.php?cat=1} \\
\hline
Instance 16 & \url{http://testphp.vulnweb.com/listproducts.php?cat=2} \\
\hline
Instance 17 & \url{http://testphp.vulnweb.com/listproducts.php?cat=3} \\
\hline
Instance 18 & \url{http://testphp.vulnweb.com/listproducts.php?cat=4} \\
\hline
Instance 19 & \url{http://testphp.vulnweb.com/login.php} \\
\hline
Instance 20 & \url{http://testphp.vulnweb.com/login.php} \\
\hline
Instance 21 & \url{http://testphp.vulnweb.com/product.php?pic=1} \\
\hline
Instance 22 & \url{http://testphp.vulnweb.com/product.php?pic=1} \\
\hline
Instance 23 & \url{http://testphp.vulnweb.com/product.php?pic=2} \\
\hline
Instance 24 & \url{http://testphp.vulnweb.com/product.php?pic=2} \\
\hline
Instance 25 & \url{http://testphp.vulnweb.com/product.php?pic=3} \\
\hline
Instance 26 & \url{http://testphp.vulnweb.com/product.php?pic=3} \\
\hline
Instance 27 & \url{http://testphp.vulnweb.com/product.php?pic=4} \\
\hline
Instance 28 & \url{http://testphp.vulnweb.com/product.php?pic=4} \\
\hline
Instance 29 & \url{http://testphp.vulnweb.com/product.php?pic=5} \\
\hline
Instance 30 & \url{http://testphp.vulnweb.com/product.php?pic=5} \\
\hline
Instance 31 & \url{http://testphp.vulnweb.com/product.php?pic=6} \\
\hline
Instance 32 & \url{http://testphp.vulnweb.com/product.php?pic=6} \\
\hline
Instance 33 & \url{http://testphp.vulnweb.com/product.php?pic=7} \\
\hline
Instance 34 & \url{http://testphp.vulnweb.com/product.php?pic=7} \\
\hline
Instance 35 & \url{http://testphp.vulnweb.com/signup.php} \\
\hline
Instance 36 & \url{http://testphp.vulnweb.com/signup.php} \\
\hline
Instance 37 & \url{http://testphp.vulnweb.com/cart.php} \\
\hline
Instance 38 & \url{http://testphp.vulnweb.com/guestbook.php} \\
\hline
Instance 39 & \url{http://testphp.vulnweb.com/guestbook.php} \\
\hline
Instance 40 & \url{http://testphp.vulnweb.com/search.php?test=query} \\
\hline
\end{longtable}
\end{center}\vspace{0.7cm}
\begin{center}
\renewcommand{\arraystretch}{1.3}
\begin{longtable}{|l|p{10cm}|}
\hline
\multicolumn{2}{|c|}{\textbf{Vulnerability 3}} \\
\hline
\textbf{Risk Level} & Medium (High) \\
\hline
\textbf{Vulnerability Name} & Content Security Policy (CSP) Header Not Set \\
\hline
\textbf{Description} & Content Security Policy (CSP) is an added layer of security that helps to detect and mitigate certain types of attacks, including Cross Site Scripting (XSS) and data injection attacks. These attacks are used for everything from data theft to site defacement or distribution of malware. CSP provides a set of standard HTTP headers that allow website owners to declare approved sources of content that browsers should be allowed to load on that page --- covered types are JavaScript, CSS, HTML frames, fonts, images and embeddable objects such as Java applets, ActiveX, audio and video files. \\
\hline
\textbf{Instances} & \textbf{URI} \\
\hline
Instance 1 & \url{http://testphp.vulnweb.com/} \\
\hline
Instance 2 & \url{http://testphp.vulnweb.com/AJAX/index.php} \\
\hline
Instance 3 & \url{http://testphp.vulnweb.com/artists.php} \\
\hline
Instance 4 & \url{http://testphp.vulnweb.com/artists.php?artist=1} \\
\hline
Instance 5 & \url{http://testphp.vulnweb.com/artists.php?artist=2} \\
\hline
Instance 6 & \url{http://testphp.vulnweb.com/artists.php?artist=3} \\
\hline
Instance 7 & \url{http://testphp.vulnweb.com/cart.php} \\
\hline
Instance 8 & \url{http://testphp.vulnweb.com/categories.php} \\
\hline
Instance 9 & \url{http://testphp.vulnweb.com/disclaimer.php} \\
\hline
Instance 10 & \url{http://testphp.vulnweb.com/guestbook.php} \\
\hline
Instance 11 & \url{http://testphp.vulnweb.com/high} \\
\hline
Instance 12 & \url{http://testphp.vulnweb.com/hpp/} \\
\hline
Instance 13 & \url{http://testphp.vulnweb.com/hpp/?pp=12} \\
\hline
Instance 14 & \url{http://testphp.vulnweb.com/hpp/params.php?p=valid\&pp=12} \\
\hline
Instance 15 & \url{http://testphp.vulnweb.com/index.php} \\
\hline
Instance 16 & \url{http://testphp.vulnweb.com/listproducts.php?artist=1} \\
\hline
Instance 17 & \url{http://testphp.vulnweb.com/listproducts.php?artist=2} \\
\hline
Instance 18 & \url{http://testphp.vulnweb.com/listproducts.php?artist=3} \\
\hline
Instance 19 & \url{http://testphp.vulnweb.com/listproducts.php?cat=1} \\
\hline
Instance 20 & \url{http://testphp.vulnweb.com/listproducts.php?cat=2} \\
\hline
Instance 21 & \url{http://testphp.vulnweb.com/listproducts.php?cat=3} \\
\hline
Instance 22 & \url{http://testphp.vulnweb.com/listproducts.php?cat=4} \\
\hline
Instance 23 & \url{http://testphp.vulnweb.com/login.php} \\
\hline
Instance 24 & \url{http://testphp.vulnweb.com/Mod\_Rewrite\_Shop/} \\
\hline
Instance 25 & \url{http://testphp.vulnweb.com/Mod\_Rewrite\_Shop/BuyProduct-1/} \\
\hline
Instance 26 & \url{http://testphp.vulnweb.com/Mod\_Rewrite\_Shop/BuyProduct-2/} \\
\hline
Instance 27 & \url{http://testphp.vulnweb.com/Mod\_Rewrite\_Shop/BuyProduct-3/} \\
\hline
Instance 28 & \url{http://testphp.vulnweb.com/Mod\_Rewrite\_Shop/Details/color-printer/3/} \\
\hline
Instance 29 & \url{http://testphp.vulnweb.com/Mod\_Rewrite\_Shop/Details/network-attached-storage-dlink/1/} \\
\hline
Instance 30 & \url{http://testphp.vulnweb.com/Mod\_Rewrite\_Shop/Details/web-camera-a4tech/2/} \\
\hline
Instance 31 & \url{http://testphp.vulnweb.com/Mod\_Rewrite\_Shop/RateProduct-1.html} \\
\hline
Instance 32 & \url{http://testphp.vulnweb.com/Mod\_Rewrite\_Shop/RateProduct-2.html} \\
\hline
Instance 33 & \url{http://testphp.vulnweb.com/Mod\_Rewrite\_Shop/RateProduct-3.html} \\
\hline
Instance 34 & \url{http://testphp.vulnweb.com/privacy.php} \\
\hline
Instance 35 & \url{http://testphp.vulnweb.com/product.php?pic=1} \\
\hline
Instance 36 & \url{http://testphp.vulnweb.com/product.php?pic=2} \\
\hline
Instance 37 & \url{http://testphp.vulnweb.com/product.php?pic=3} \\
\hline
Instance 38 & \url{http://testphp.vulnweb.com/product.php?pic=4} \\
\hline
Instance 39 & \url{http://testphp.vulnweb.com/product.php?pic=5} \\
\hline
Instance 40 & \url{http://testphp.vulnweb.com/product.php?pic=6} \\
\hline
Instance 41 & \url{http://testphp.vulnweb.com/product.php?pic=7} \\
\hline
Instance 42 & \url{http://testphp.vulnweb.com/robots.txt} \\
\hline
Instance 43 & \url{http://testphp.vulnweb.com/signup.php} \\
\hline
Instance 44 & \url{http://testphp.vulnweb.com/cart.php} \\
\hline
Instance 45 & \url{http://testphp.vulnweb.com/guestbook.php} \\
\hline
Instance 46 & \url{http://testphp.vulnweb.com/search.php?test=query} \\
\hline
Instance 47 & \url{http://testphp.vulnweb.com/secured/newuser.php} \\
\hline
\end{longtable}
\end{center}\vspace{0.7cm}
\vspace{0.7cm}

\end{document}
