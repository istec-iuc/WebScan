\documentclass[12pt]{article}
\usepackage[utf8]{inputenc}
\usepackage{geometry}
\usepackage{graphicx}
\geometry{margin=1in}

\title{\textbf{Static and Dynamic Analysis}}
\author{ISTEC-Cyber Security}
\date{\today}  % Automatically insert today's date

\begin{document}

\maketitle

\section*{Description}
This report contains static and dynamic analysis of the target. It uses Semgrep for static analysis and OWASP ZAP, Nmap, and SQLMap for dynamic analysis.

\vspace{10cm} % Space to push the footer down

\begin{center}
	\textbf{Provided by} \\[1em]
	\includegraphics[width=0.1\textwidth]{logo75.png}
\end{center}

\newpage  % Page break

\section{Static Analysis}
Details about static analysis...

\section{Analysis Report}

\subsection{Risk Summary}
\begin{table}[h!]
    \centering
    \renewcommand{\arraystretch}{1.5}
    \begin{tabular}{|c|c|}
        \hline
        \textbf{Risk Level} & \textbf{Number of Findings} \\
        \hline
        Low Risk & lowcount \\ %lowcount is a placeholder here
        \hline
        Medium Risk & mediumcount \\ %mediumcount is a placeholder here
        \hline
        High Risk & highcount \\ %highcount is a placeholder here
        \hline
        Critical Risk & criticalcount \\ %criticalcount is a placeholder here
        \hline
    \end{tabular}
    \caption{Summary of Risk Findings}
    \label{tab:risk_summary}

\subsection{Vulnerability Categories}
    \begin{itemize}
        \item Categories:
 
        % Add more categories as needed
    \end{itemize}
\end{table}

\subsection{Vulnerabilities by Page}
\begin{itemize}
\item \textbf{Path:} \{/mnt/c/Users/Administrator/source/repos/WebScan/ScrapedFiles/index/main/analytics.js\} \\ 
\item \textbf{Vulnerability Class:} \{['Denial-of-Service (DoS)']\} \\ 
\item \textbf{Start:} \{N/A\} \\ 
\item \textbf{End:} \{N/A\} \\ 
\item \textbf{Message:} \{RegExp() called with a `a` function argument, this might allow an attacker to cause a Regular Expression Denial-of-Service (ReDoS) within your application as RegExP blocks the main thread. For this reason, it is recommended to use hardcoded regexes instead. If your regex is run on user-controlled input, consider performing input validation or use a regex checking/sanitization library such as https://www.npmjs.com/package/recheck to verify that the regex does not appear vulnerable to ReDoS.\} \\ 
\item \textbf{Path:} \{/mnt/c/Users/Administrator/source/repos/WebScan/ScrapedFiles/index/main/analytics.js\} \\ 
\item \textbf{Vulnerability Class:} \{['Denial-of-Service (DoS)']\} \\ 
\item \textbf{Start:} \{N/A\} \\ 
\item \textbf{End:} \{N/A\} \\ 
\item \textbf{Message:} \{RegExp() called with a `a` function argument, this might allow an attacker to cause a Regular Expression Denial-of-Service (ReDoS) within your application as RegExP blocks the main thread. For this reason, it is recommended to use hardcoded regexes instead. If your regex is run on user-controlled input, consider performing input validation or use a regex checking/sanitization library such as https://www.npmjs.com/package/recheck to verify that the regex does not appear vulnerable to ReDoS.\} \\ 
\item \textbf{Path:} \{/mnt/c/Users/Administrator/source/repos/WebScan/ScrapedFiles/index/main/analytics.js\} \\ 
\item \textbf{Vulnerability Class:} \{['Denial-of-Service (DoS)']\} \\ 
\item \textbf{Start:} \{N/A\} \\ 
\item \textbf{End:} \{N/A\} \\ 
\item \textbf{Message:} \{RegExp() called with a `a` function argument, this might allow an attacker to cause a Regular Expression Denial-of-Service (ReDoS) within your application as RegExP blocks the main thread. For this reason, it is recommended to use hardcoded regexes instead. If your regex is run on user-controlled input, consider performing input validation or use a regex checking/sanitization library such as https://www.npmjs.com/package/recheck to verify that the regex does not appear vulnerable to ReDoS.\} \\ 
\item \textbf{Path:} \{/mnt/c/Users/Administrator/source/repos/WebScan/ScrapedFiles/index/main/analytics.js\} \\ 
\item \textbf{Vulnerability Class:} \{['Improper Encoding']\} \\ 
\item \textbf{Start:} \{N/A\} \\ 
\item \textbf{End:} \{N/A\} \\ 
\item \textbf{Message:} \{`"https://www.google.%/ads/ga-audiences".replace` method will only replace the first occurrence when used with a string argument ("%"). If this method is used for escaping of dangerous data then there is a possibility for a bypass. Try to use sanitization library instead or use a Regex with a global flag.\} \\ 
\item \textbf{Path:} \{/mnt/c/Users/Administrator/source/repos/WebScan/ScrapedFiles/index/main/analytics.js\} \\ 
\item \textbf{Vulnerability Class:} \{['Denial-of-Service (DoS)']\} \\ 
\item \textbf{Start:} \{N/A\} \\ 
\item \textbf{End:} \{N/A\} \\ 
\item \textbf{Message:} \{RegExp() called with a `a` function argument, this might allow an attacker to cause a Regular Expression Denial-of-Service (ReDoS) within your application as RegExP blocks the main thread. For this reason, it is recommended to use hardcoded regexes instead. If your regex is run on user-controlled input, consider performing input validation or use a regex checking/sanitization library such as https://www.npmjs.com/package/recheck to verify that the regex does not appear vulnerable to ReDoS.\} \\ 
\item \textbf{Path:} \{/mnt/c/Users/Administrator/source/repos/WebScan/ScrapedFiles/index/main/conversion.js\} \\ 
\item \textbf{Vulnerability Class:} \{['Mass Assignment']\} \\ 
\item \textbf{Start:} \{N/A\} \\ 
\item \textbf{End:} \{N/A\} \\ 
\item \textbf{Message:} \{Possibility of prototype polluting function detected. By adding or modifying attributes of an object prototype, it is possible to create attributes that exist on every object, or replace critical attributes with malicious ones. This can be problematic if the software depends on existence or non-existence of certain attributes, or uses pre-defined attributes of object prototype (such as hasOwnProperty, toString or valueOf). Possible mitigations might be: freezing the object prototype, using an object without prototypes (via Object.create(null) ), blocking modifications of attributes that resolve to object prototype, using Map instead of object.\} \\ 
\item \textbf{Path:} \{/mnt/c/Users/Administrator/source/repos/WebScan/ScrapedFiles/index/main/conversion.js\} \\ 
\item \textbf{Vulnerability Class:} \{['Denial-of-Service (DoS)']\} \\ 
\item \textbf{Start:} \{N/A\} \\ 
\item \textbf{End:} \{N/A\} \\ 
\item \textbf{Message:} \{RegExp() called with a `a` function argument, this might allow an attacker to cause a Regular Expression Denial-of-Service (ReDoS) within your application as RegExP blocks the main thread. For this reason, it is recommended to use hardcoded regexes instead. If your regex is run on user-controlled input, consider performing input validation or use a regex checking/sanitization library such as https://www.npmjs.com/package/recheck to verify that the regex does not appear vulnerable to ReDoS.\} \\ 
\item \textbf{Path:} \{/mnt/c/Users/Administrator/source/repos/WebScan/ScrapedFiles/index/main/conversion.js\} \\ 
\item \textbf{Vulnerability Class:} \{['Denial-of-Service (DoS)']\} \\ 
\item \textbf{Start:} \{N/A\} \\ 
\item \textbf{End:} \{N/A\} \\ 
\item \textbf{Message:} \{RegExp() called with a `a` function argument, this might allow an attacker to cause a Regular Expression Denial-of-Service (ReDoS) within your application as RegExP blocks the main thread. For this reason, it is recommended to use hardcoded regexes instead. If your regex is run on user-controlled input, consider performing input validation or use a regex checking/sanitization library such as https://www.npmjs.com/package/recheck to verify that the regex does not appear vulnerable to ReDoS.\} \\ 
\item \textbf{Path:} \{/mnt/c/Users/Administrator/source/repos/WebScan/ScrapedFiles/index/main/conversion.js\} \\ 
\item \textbf{Vulnerability Class:} \{['Denial-of-Service (DoS)']\} \\ 
\item \textbf{Start:} \{N/A\} \\ 
\item \textbf{End:} \{N/A\} \\ 
\item \textbf{Message:} \{RegExp() called with a `b` function argument, this might allow an attacker to cause a Regular Expression Denial-of-Service (ReDoS) within your application as RegExP blocks the main thread. For this reason, it is recommended to use hardcoded regexes instead. If your regex is run on user-controlled input, consider performing input validation or use a regex checking/sanitization library such as https://www.npmjs.com/package/recheck to verify that the regex does not appear vulnerable to ReDoS.\} \\ 
\item \textbf{Path:} \{/mnt/c/Users/Administrator/source/repos/WebScan/ScrapedFiles/index/main/conversion.js\} \\ 
\item \textbf{Vulnerability Class:} \{['Cross-Site-Scripting (XSS)']\} \\ 
\item \textbf{Start:} \{N/A\} \\ 
\item \textbf{End:} \{N/A\} \\ 
\item \textbf{Message:} \{User controlled data in methods like `innerHTML`, `outerHTML` or `document.write` is an anti-pattern that can lead to XSS vulnerabilities\} \\ 
\item \textbf{Path:} \{/mnt/c/Users/Administrator/source/repos/WebScan/ScrapedFiles/index/main/index.html\} \\ 
\item \textbf{Vulnerability Class:} \{['Cryptographic Issues']\} \\ 
\item \textbf{Start:} \{N/A\} \\ 
\item \textbf{End:} \{N/A\} \\ 
\item \textbf{Message:} \{This tag is missing an 'integrity' subresource integrity attribute. The 'integrity' attribute allows for the browser to verify that externally hosted files (for example from a CDN) are delivered without unexpected manipulation. Without this attribute, if an attacker can modify the externally hosted resource, this could lead to XSS and other types of attacks. To prevent this, include the base64-encoded cryptographic hash of the resource (file) you�re telling the browser to fetch in the 'integrity' attribute for all externally hosted files.\} \\ 
\item \textbf{Path:} \{/mnt/c/Users/Administrator/source/repos/WebScan/ScrapedFiles/index/main/index.html\} \\ 
\item \textbf{Vulnerability Class:} \{['Cryptographic Issues']\} \\ 
\item \textbf{Start:} \{N/A\} \\ 
\item \textbf{End:} \{N/A\} \\ 
\item \textbf{Message:} \{This tag is missing an 'integrity' subresource integrity attribute. The 'integrity' attribute allows for the browser to verify that externally hosted files (for example from a CDN) are delivered without unexpected manipulation. Without this attribute, if an attacker can modify the externally hosted resource, this could lead to XSS and other types of attacks. To prevent this, include the base64-encoded cryptographic hash of the resource (file) you�re telling the browser to fetch in the 'integrity' attribute for all externally hosted files.\} \\ 
\item \textbf{Path:} \{/mnt/c/Users/Administrator/source/repos/WebScan/ScrapedFiles/index/main/index.html\} \\ 
\item \textbf{Vulnerability Class:} \{['Cryptographic Issues']\} \\ 
\item \textbf{Start:} \{N/A\} \\ 
\item \textbf{End:} \{N/A\} \\ 
\item \textbf{Message:} \{This tag is missing an 'integrity' subresource integrity attribute. The 'integrity' attribute allows for the browser to verify that externally hosted files (for example from a CDN) are delivered without unexpected manipulation. Without this attribute, if an attacker can modify the externally hosted resource, this could lead to XSS and other types of attacks. To prevent this, include the base64-encoded cryptographic hash of the resource (file) you�re telling the browser to fetch in the 'integrity' attribute for all externally hosted files.\} \\ 
\item \textbf{Path:} \{/mnt/c/Users/Administrator/source/repos/WebScan/ScrapedFiles/index/main/index.html\} \\ 
\item \textbf{Vulnerability Class:} \{['Cryptographic Issues']\} \\ 
\item \textbf{Start:} \{N/A\} \\ 
\item \textbf{End:} \{N/A\} \\ 
\item \textbf{Message:} \{This tag is missing an 'integrity' subresource integrity attribute. The 'integrity' attribute allows for the browser to verify that externally hosted files (for example from a CDN) are delivered without unexpected manipulation. Without this attribute, if an attacker can modify the externally hosted resource, this could lead to XSS and other types of attacks. To prevent this, include the base64-encoded cryptographic hash of the resource (file) you�re telling the browser to fetch in the 'integrity' attribute for all externally hosted files.\} \\ 
\item \textbf{Path:} \{/mnt/c/Users/Administrator/source/repos/WebScan/ScrapedFiles/index/main/index.html\} \\ 
\item \textbf{Vulnerability Class:} \{['Cryptographic Issues']\} \\ 
\item \textbf{Start:} \{N/A\} \\ 
\item \textbf{End:} \{N/A\} \\ 
\item \textbf{Message:} \{This tag is missing an 'integrity' subresource integrity attribute. The 'integrity' attribute allows for the browser to verify that externally hosted files (for example from a CDN) are delivered without unexpected manipulation. Without this attribute, if an attacker can modify the externally hosted resource, this could lead to XSS and other types of attacks. To prevent this, include the base64-encoded cryptographic hash of the resource (file) you�re telling the browser to fetch in the 'integrity' attribute for all externally hosted files.\} \\ 
\item \textbf{Path:} \{/mnt/c/Users/Administrator/source/repos/WebScan/ScrapedFiles/index/main/index.html\} \\ 
\item \textbf{Vulnerability Class:} \{['Cryptographic Issues']\} \\ 
\item \textbf{Start:} \{N/A\} \\ 
\item \textbf{End:} \{N/A\} \\ 
\item \textbf{Message:} \{This tag is missing an 'integrity' subresource integrity attribute. The 'integrity' attribute allows for the browser to verify that externally hosted files (for example from a CDN) are delivered without unexpected manipulation. Without this attribute, if an attacker can modify the externally hosted resource, this could lead to XSS and other types of attacks. To prevent this, include the base64-encoded cryptographic hash of the resource (file) you�re telling the browser to fetch in the 'integrity' attribute for all externally hosted files.\} \\ 
\item \textbf{Path:} \{/mnt/c/Users/Administrator/source/repos/WebScan/ScrapedFiles/index/main/index.html\} \\ 
\item \textbf{Vulnerability Class:} \{['Cryptographic Issues']\} \\ 
\item \textbf{Start:} \{N/A\} \\ 
\item \textbf{End:} \{N/A\} \\ 
\item \textbf{Message:} \{This tag is missing an 'integrity' subresource integrity attribute. The 'integrity' attribute allows for the browser to verify that externally hosted files (for example from a CDN) are delivered without unexpected manipulation. Without this attribute, if an attacker can modify the externally hosted resource, this could lead to XSS and other types of attacks. To prevent this, include the base64-encoded cryptographic hash of the resource (file) you�re telling the browser to fetch in the 'integrity' attribute for all externally hosted files.\} \\ 
\item \textbf{Path:} \{/mnt/c/Users/Administrator/source/repos/WebScan/ScrapedFiles/pages/forms/main/analytics.js\} \\ 
\item \textbf{Vulnerability Class:} \{['Denial-of-Service (DoS)']\} \\ 
\item \textbf{Start:} \{N/A\} \\ 
\item \textbf{End:} \{N/A\} \\ 
\item \textbf{Message:} \{RegExp() called with a `a` function argument, this might allow an attacker to cause a Regular Expression Denial-of-Service (ReDoS) within your application as RegExP blocks the main thread. For this reason, it is recommended to use hardcoded regexes instead. If your regex is run on user-controlled input, consider performing input validation or use a regex checking/sanitization library such as https://www.npmjs.com/package/recheck to verify that the regex does not appear vulnerable to ReDoS.\} \\ 
\item \textbf{Path:} \{/mnt/c/Users/Administrator/source/repos/WebScan/ScrapedFiles/pages/forms/main/analytics.js\} \\ 
\item \textbf{Vulnerability Class:} \{['Denial-of-Service (DoS)']\} \\ 
\item \textbf{Start:} \{N/A\} \\ 
\item \textbf{End:} \{N/A\} \\ 
\item \textbf{Message:} \{RegExp() called with a `a` function argument, this might allow an attacker to cause a Regular Expression Denial-of-Service (ReDoS) within your application as RegExP blocks the main thread. For this reason, it is recommended to use hardcoded regexes instead. If your regex is run on user-controlled input, consider performing input validation or use a regex checking/sanitization library such as https://www.npmjs.com/package/recheck to verify that the regex does not appear vulnerable to ReDoS.\} \\ 
\item \textbf{Path:} \{/mnt/c/Users/Administrator/source/repos/WebScan/ScrapedFiles/pages/forms/main/analytics.js\} \\ 
\item \textbf{Vulnerability Class:} \{['Denial-of-Service (DoS)']\} \\ 
\item \textbf{Start:} \{N/A\} \\ 
\item \textbf{End:} \{N/A\} \\ 
\item \textbf{Message:} \{RegExp() called with a `a` function argument, this might allow an attacker to cause a Regular Expression Denial-of-Service (ReDoS) within your application as RegExP blocks the main thread. For this reason, it is recommended to use hardcoded regexes instead. If your regex is run on user-controlled input, consider performing input validation or use a regex checking/sanitization library such as https://www.npmjs.com/package/recheck to verify that the regex does not appear vulnerable to ReDoS.\} \\ 
\item \textbf{Path:} \{/mnt/c/Users/Administrator/source/repos/WebScan/ScrapedFiles/pages/forms/main/analytics.js\} \\ 
\item \textbf{Vulnerability Class:} \{['Improper Encoding']\} \\ 
\item \textbf{Start:} \{N/A\} \\ 
\item \textbf{End:} \{N/A\} \\ 
\item \textbf{Message:} \{`"https://www.google.%/ads/ga-audiences".replace` method will only replace the first occurrence when used with a string argument ("%"). If this method is used for escaping of dangerous data then there is a possibility for a bypass. Try to use sanitization library instead or use a Regex with a global flag.\} \\ 
\item \textbf{Path:} \{/mnt/c/Users/Administrator/source/repos/WebScan/ScrapedFiles/pages/forms/main/analytics.js\} \\ 
\item \textbf{Vulnerability Class:} \{['Denial-of-Service (DoS)']\} \\ 
\item \textbf{Start:} \{N/A\} \\ 
\item \textbf{End:} \{N/A\} \\ 
\item \textbf{Message:} \{RegExp() called with a `a` function argument, this might allow an attacker to cause a Regular Expression Denial-of-Service (ReDoS) within your application as RegExP blocks the main thread. For this reason, it is recommended to use hardcoded regexes instead. If your regex is run on user-controlled input, consider performing input validation or use a regex checking/sanitization library such as https://www.npmjs.com/package/recheck to verify that the regex does not appear vulnerable to ReDoS.\} \\ 
\item \textbf{Path:} \{/mnt/c/Users/Administrator/source/repos/WebScan/ScrapedFiles/pages/forms/main/conversion.js\} \\ 
\item \textbf{Vulnerability Class:} \{['Mass Assignment']\} \\ 
\item \textbf{Start:} \{N/A\} \\ 
\item \textbf{End:} \{N/A\} \\ 
\item \textbf{Message:} \{Possibility of prototype polluting function detected. By adding or modifying attributes of an object prototype, it is possible to create attributes that exist on every object, or replace critical attributes with malicious ones. This can be problematic if the software depends on existence or non-existence of certain attributes, or uses pre-defined attributes of object prototype (such as hasOwnProperty, toString or valueOf). Possible mitigations might be: freezing the object prototype, using an object without prototypes (via Object.create(null) ), blocking modifications of attributes that resolve to object prototype, using Map instead of object.\} \\ 
\item \textbf{Path:} \{/mnt/c/Users/Administrator/source/repos/WebScan/ScrapedFiles/pages/forms/main/conversion.js\} \\ 
\item \textbf{Vulnerability Class:} \{['Denial-of-Service (DoS)']\} \\ 
\item \textbf{Start:} \{N/A\} \\ 
\item \textbf{End:} \{N/A\} \\ 
\item \textbf{Message:} \{RegExp() called with a `a` function argument, this might allow an attacker to cause a Regular Expression Denial-of-Service (ReDoS) within your application as RegExP blocks the main thread. For this reason, it is recommended to use hardcoded regexes instead. If your regex is run on user-controlled input, consider performing input validation or use a regex checking/sanitization library such as https://www.npmjs.com/package/recheck to verify that the regex does not appear vulnerable to ReDoS.\} \\ 
\item \textbf{Path:} \{/mnt/c/Users/Administrator/source/repos/WebScan/ScrapedFiles/pages/forms/main/conversion.js\} \\ 
\item \textbf{Vulnerability Class:} \{['Denial-of-Service (DoS)']\} \\ 
\item \textbf{Start:} \{N/A\} \\ 
\item \textbf{End:} \{N/A\} \\ 
\item \textbf{Message:} \{RegExp() called with a `a` function argument, this might allow an attacker to cause a Regular Expression Denial-of-Service (ReDoS) within your application as RegExP blocks the main thread. For this reason, it is recommended to use hardcoded regexes instead. If your regex is run on user-controlled input, consider performing input validation or use a regex checking/sanitization library such as https://www.npmjs.com/package/recheck to verify that the regex does not appear vulnerable to ReDoS.\} \\ 
\item \textbf{Path:} \{/mnt/c/Users/Administrator/source/repos/WebScan/ScrapedFiles/pages/forms/main/conversion.js\} \\ 
\item \textbf{Vulnerability Class:} \{['Denial-of-Service (DoS)']\} \\ 
\item \textbf{Start:} \{N/A\} \\ 
\item \textbf{End:} \{N/A\} \\ 
\item \textbf{Message:} \{RegExp() called with a `b` function argument, this might allow an attacker to cause a Regular Expression Denial-of-Service (ReDoS) within your application as RegExP blocks the main thread. For this reason, it is recommended to use hardcoded regexes instead. If your regex is run on user-controlled input, consider performing input validation or use a regex checking/sanitization library such as https://www.npmjs.com/package/recheck to verify that the regex does not appear vulnerable to ReDoS.\} \\ 
\item \textbf{Path:} \{/mnt/c/Users/Administrator/source/repos/WebScan/ScrapedFiles/pages/forms/main/conversion.js\} \\ 
\item \textbf{Vulnerability Class:} \{['Cross-Site-Scripting (XSS)']\} \\ 
\item \textbf{Start:} \{N/A\} \\ 
\item \textbf{End:} \{N/A\} \\ 
\item \textbf{Message:} \{User controlled data in methods like `innerHTML`, `outerHTML` or `document.write` is an anti-pattern that can lead to XSS vulnerabilities\} \\ 
\item \textbf{Path:} \{/mnt/c/Users/Administrator/source/repos/WebScan/ScrapedFiles/pages/forms/main/forms.html\} \\ 
\item \textbf{Vulnerability Class:} \{['Cryptographic Issues']\} \\ 
\item \textbf{Start:} \{N/A\} \\ 
\item \textbf{End:} \{N/A\} \\ 
\item \textbf{Message:} \{This tag is missing an 'integrity' subresource integrity attribute. The 'integrity' attribute allows for the browser to verify that externally hosted files (for example from a CDN) are delivered without unexpected manipulation. Without this attribute, if an attacker can modify the externally hosted resource, this could lead to XSS and other types of attacks. To prevent this, include the base64-encoded cryptographic hash of the resource (file) you�re telling the browser to fetch in the 'integrity' attribute for all externally hosted files.\} \\ 
\item \textbf{Path:} \{/mnt/c/Users/Administrator/source/repos/WebScan/ScrapedFiles/pages/forms/main/forms.html\} \\ 
\item \textbf{Vulnerability Class:} \{['Cryptographic Issues']\} \\ 
\item \textbf{Start:} \{N/A\} \\ 
\item \textbf{End:} \{N/A\} \\ 
\item \textbf{Message:} \{This tag is missing an 'integrity' subresource integrity attribute. The 'integrity' attribute allows for the browser to verify that externally hosted files (for example from a CDN) are delivered without unexpected manipulation. Without this attribute, if an attacker can modify the externally hosted resource, this could lead to XSS and other types of attacks. To prevent this, include the base64-encoded cryptographic hash of the resource (file) you�re telling the browser to fetch in the 'integrity' attribute for all externally hosted files.\} \\ 
\item \textbf{Path:} \{/mnt/c/Users/Administrator/source/repos/WebScan/ScrapedFiles/pages/forms/main/forms.html\} \\ 
\item \textbf{Vulnerability Class:} \{['Mishandled Sensitive Information']\} \\ 
\item \textbf{Start:} \{N/A\} \\ 
\item \textbf{End:} \{N/A\} \\ 
\item \textbf{Message:} \{This link points to a plaintext HTTP URL. Prefer an encrypted HTTPS URL if possible.\} \\ 
\item \textbf{Path:} \{/mnt/c/Users/Administrator/source/repos/WebScan/ScrapedFiles/pages/forms/main/forms.html\} \\ 
\item \textbf{Vulnerability Class:} \{['Cryptographic Issues']\} \\ 
\item \textbf{Start:} \{N/A\} \\ 
\item \textbf{End:} \{N/A\} \\ 
\item \textbf{Message:} \{This tag is missing an 'integrity' subresource integrity attribute. The 'integrity' attribute allows for the browser to verify that externally hosted files (for example from a CDN) are delivered without unexpected manipulation. Without this attribute, if an attacker can modify the externally hosted resource, this could lead to XSS and other types of attacks. To prevent this, include the base64-encoded cryptographic hash of the resource (file) you�re telling the browser to fetch in the 'integrity' attribute for all externally hosted files.\} \\ 
\item \textbf{Path:} \{/mnt/c/Users/Administrator/source/repos/WebScan/ScrapedFiles/pages/forms/main/forms.html\} \\ 
\item \textbf{Vulnerability Class:} \{['Cryptographic Issues']\} \\ 
\item \textbf{Start:} \{N/A\} \\ 
\item \textbf{End:} \{N/A\} \\ 
\item \textbf{Message:} \{This tag is missing an 'integrity' subresource integrity attribute. The 'integrity' attribute allows for the browser to verify that externally hosted files (for example from a CDN) are delivered without unexpected manipulation. Without this attribute, if an attacker can modify the externally hosted resource, this could lead to XSS and other types of attacks. To prevent this, include the base64-encoded cryptographic hash of the resource (file) you�re telling the browser to fetch in the 'integrity' attribute for all externally hosted files.\} \\ 
\item \textbf{Path:} \{/mnt/c/Users/Administrator/source/repos/WebScan/ScrapedFiles/pages/forms/main/forms.html\} \\ 
\item \textbf{Vulnerability Class:} \{['Cryptographic Issues']\} \\ 
\item \textbf{Start:} \{N/A\} \\ 
\item \textbf{End:} \{N/A\} \\ 
\item \textbf{Message:} \{This tag is missing an 'integrity' subresource integrity attribute. The 'integrity' attribute allows for the browser to verify that externally hosted files (for example from a CDN) are delivered without unexpected manipulation. Without this attribute, if an attacker can modify the externally hosted resource, this could lead to XSS and other types of attacks. To prevent this, include the base64-encoded cryptographic hash of the resource (file) you�re telling the browser to fetch in the 'integrity' attribute for all externally hosted files.\} \\ 
\item \textbf{Path:} \{/mnt/c/Users/Administrator/source/repos/WebScan/ScrapedFiles/pages/forms/main/forms.html\} \\ 
\item \textbf{Vulnerability Class:} \{['Cryptographic Issues']\} \\ 
\item \textbf{Start:} \{N/A\} \\ 
\item \textbf{End:} \{N/A\} \\ 
\item \textbf{Message:} \{This tag is missing an 'integrity' subresource integrity attribute. The 'integrity' attribute allows for the browser to verify that externally hosted files (for example from a CDN) are delivered without unexpected manipulation. Without this attribute, if an attacker can modify the externally hosted resource, this could lead to XSS and other types of attacks. To prevent this, include the base64-encoded cryptographic hash of the resource (file) you�re telling the browser to fetch in the 'integrity' attribute for all externally hosted files.\} \\ 
\item \textbf{Path:} \{/mnt/c/Users/Administrator/source/repos/WebScan/ScrapedFiles/pages/forms/main/forms.html\} \\ 
\item \textbf{Vulnerability Class:} \{['Cryptographic Issues']\} \\ 
\item \textbf{Start:} \{N/A\} \\ 
\item \textbf{End:} \{N/A\} \\ 
\item \textbf{Message:} \{This tag is missing an 'integrity' subresource integrity attribute. The 'integrity' attribute allows for the browser to verify that externally hosted files (for example from a CDN) are delivered without unexpected manipulation. Without this attribute, if an attacker can modify the externally hosted resource, this could lead to XSS and other types of attacks. To prevent this, include the base64-encoded cryptographic hash of the resource (file) you�re telling the browser to fetch in the 'integrity' attribute for all externally hosted files.\} \\ 
\item \textbf{Path:} \{/mnt/c/Users/Administrator/source/repos/WebScan/ScrapedFiles/pages/frames/main/analytics.js\} \\ 
\item \textbf{Vulnerability Class:} \{['Denial-of-Service (DoS)']\} \\ 
\item \textbf{Start:} \{N/A\} \\ 
\item \textbf{End:} \{N/A\} \\ 
\item \textbf{Message:} \{RegExp() called with a `a` function argument, this might allow an attacker to cause a Regular Expression Denial-of-Service (ReDoS) within your application as RegExP blocks the main thread. For this reason, it is recommended to use hardcoded regexes instead. If your regex is run on user-controlled input, consider performing input validation or use a regex checking/sanitization library such as https://www.npmjs.com/package/recheck to verify that the regex does not appear vulnerable to ReDoS.\} \\ 
\item \textbf{Path:} \{/mnt/c/Users/Administrator/source/repos/WebScan/ScrapedFiles/pages/frames/main/analytics.js\} \\ 
\item \textbf{Vulnerability Class:} \{['Denial-of-Service (DoS)']\} \\ 
\item \textbf{Start:} \{N/A\} \\ 
\item \textbf{End:} \{N/A\} \\ 
\item \textbf{Message:} \{RegExp() called with a `a` function argument, this might allow an attacker to cause a Regular Expression Denial-of-Service (ReDoS) within your application as RegExP blocks the main thread. For this reason, it is recommended to use hardcoded regexes instead. If your regex is run on user-controlled input, consider performing input validation or use a regex checking/sanitization library such as https://www.npmjs.com/package/recheck to verify that the regex does not appear vulnerable to ReDoS.\} \\ 
\item \textbf{Path:} \{/mnt/c/Users/Administrator/source/repos/WebScan/ScrapedFiles/pages/frames/main/analytics.js\} \\ 
\item \textbf{Vulnerability Class:} \{['Denial-of-Service (DoS)']\} \\ 
\item \textbf{Start:} \{N/A\} \\ 
\item \textbf{End:} \{N/A\} \\ 
\item \textbf{Message:} \{RegExp() called with a `a` function argument, this might allow an attacker to cause a Regular Expression Denial-of-Service (ReDoS) within your application as RegExP blocks the main thread. For this reason, it is recommended to use hardcoded regexes instead. If your regex is run on user-controlled input, consider performing input validation or use a regex checking/sanitization library such as https://www.npmjs.com/package/recheck to verify that the regex does not appear vulnerable to ReDoS.\} \\ 
\item \textbf{Path:} \{/mnt/c/Users/Administrator/source/repos/WebScan/ScrapedFiles/pages/frames/main/analytics.js\} \\ 
\item \textbf{Vulnerability Class:} \{['Improper Encoding']\} \\ 
\item \textbf{Start:} \{N/A\} \\ 
\item \textbf{End:} \{N/A\} \\ 
\item \textbf{Message:} \{`"https://www.google.%/ads/ga-audiences".replace` method will only replace the first occurrence when used with a string argument ("%"). If this method is used for escaping of dangerous data then there is a possibility for a bypass. Try to use sanitization library instead or use a Regex with a global flag.\} \\ 
\item \textbf{Path:} \{/mnt/c/Users/Administrator/source/repos/WebScan/ScrapedFiles/pages/frames/main/analytics.js\} \\ 
\item \textbf{Vulnerability Class:} \{['Denial-of-Service (DoS)']\} \\ 
\item \textbf{Start:} \{N/A\} \\ 
\item \textbf{End:} \{N/A\} \\ 
\item \textbf{Message:} \{RegExp() called with a `a` function argument, this might allow an attacker to cause a Regular Expression Denial-of-Service (ReDoS) within your application as RegExP blocks the main thread. For this reason, it is recommended to use hardcoded regexes instead. If your regex is run on user-controlled input, consider performing input validation or use a regex checking/sanitization library such as https://www.npmjs.com/package/recheck to verify that the regex does not appear vulnerable to ReDoS.\} \\ 
\item \textbf{Path:} \{/mnt/c/Users/Administrator/source/repos/WebScan/ScrapedFiles/pages/frames/main/conversion.js\} \\ 
\item \textbf{Vulnerability Class:} \{['Mass Assignment']\} \\ 
\item \textbf{Start:} \{N/A\} \\ 
\item \textbf{End:} \{N/A\} \\ 
\item \textbf{Message:} \{Possibility of prototype polluting function detected. By adding or modifying attributes of an object prototype, it is possible to create attributes that exist on every object, or replace critical attributes with malicious ones. This can be problematic if the software depends on existence or non-existence of certain attributes, or uses pre-defined attributes of object prototype (such as hasOwnProperty, toString or valueOf). Possible mitigations might be: freezing the object prototype, using an object without prototypes (via Object.create(null) ), blocking modifications of attributes that resolve to object prototype, using Map instead of object.\} \\ 
\item \textbf{Path:} \{/mnt/c/Users/Administrator/source/repos/WebScan/ScrapedFiles/pages/frames/main/conversion.js\} \\ 
\item \textbf{Vulnerability Class:} \{['Denial-of-Service (DoS)']\} \\ 
\item \textbf{Start:} \{N/A\} \\ 
\item \textbf{End:} \{N/A\} \\ 
\item \textbf{Message:} \{RegExp() called with a `a` function argument, this might allow an attacker to cause a Regular Expression Denial-of-Service (ReDoS) within your application as RegExP blocks the main thread. For this reason, it is recommended to use hardcoded regexes instead. If your regex is run on user-controlled input, consider performing input validation or use a regex checking/sanitization library such as https://www.npmjs.com/package/recheck to verify that the regex does not appear vulnerable to ReDoS.\} \\ 
\item \textbf{Path:} \{/mnt/c/Users/Administrator/source/repos/WebScan/ScrapedFiles/pages/frames/main/conversion.js\} \\ 
\item \textbf{Vulnerability Class:} \{['Denial-of-Service (DoS)']\} \\ 
\item \textbf{Start:} \{N/A\} \\ 
\item \textbf{End:} \{N/A\} \\ 
\item \textbf{Message:} \{RegExp() called with a `a` function argument, this might allow an attacker to cause a Regular Expression Denial-of-Service (ReDoS) within your application as RegExP blocks the main thread. For this reason, it is recommended to use hardcoded regexes instead. If your regex is run on user-controlled input, consider performing input validation or use a regex checking/sanitization library such as https://www.npmjs.com/package/recheck to verify that the regex does not appear vulnerable to ReDoS.\} \\ 
\item \textbf{Path:} \{/mnt/c/Users/Administrator/source/repos/WebScan/ScrapedFiles/pages/frames/main/conversion.js\} \\ 
\item \textbf{Vulnerability Class:} \{['Denial-of-Service (DoS)']\} \\ 
\item \textbf{Start:} \{N/A\} \\ 
\item \textbf{End:} \{N/A\} \\ 
\item \textbf{Message:} \{RegExp() called with a `b` function argument, this might allow an attacker to cause a Regular Expression Denial-of-Service (ReDoS) within your application as RegExP blocks the main thread. For this reason, it is recommended to use hardcoded regexes instead. If your regex is run on user-controlled input, consider performing input validation or use a regex checking/sanitization library such as https://www.npmjs.com/package/recheck to verify that the regex does not appear vulnerable to ReDoS.\} \\ 
\item \textbf{Path:} \{/mnt/c/Users/Administrator/source/repos/WebScan/ScrapedFiles/pages/frames/main/conversion.js\} \\ 
\item \textbf{Vulnerability Class:} \{['Cross-Site-Scripting (XSS)']\} \\ 
\item \textbf{Start:} \{N/A\} \\ 
\item \textbf{End:} \{N/A\} \\ 
\item \textbf{Message:} \{User controlled data in methods like `innerHTML`, `outerHTML` or `document.write` is an anti-pattern that can lead to XSS vulnerabilities\} \\ 
\item \textbf{Path:} \{/mnt/c/Users/Administrator/source/repos/WebScan/ScrapedFiles/pages/frames/main/frames.html\} \\ 
\item \textbf{Vulnerability Class:} \{['Cryptographic Issues']\} \\ 
\item \textbf{Start:} \{N/A\} \\ 
\item \textbf{End:} \{N/A\} \\ 
\item \textbf{Message:} \{This tag is missing an 'integrity' subresource integrity attribute. The 'integrity' attribute allows for the browser to verify that externally hosted files (for example from a CDN) are delivered without unexpected manipulation. Without this attribute, if an attacker can modify the externally hosted resource, this could lead to XSS and other types of attacks. To prevent this, include the base64-encoded cryptographic hash of the resource (file) you�re telling the browser to fetch in the 'integrity' attribute for all externally hosted files.\} \\ 
\item \textbf{Path:} \{/mnt/c/Users/Administrator/source/repos/WebScan/ScrapedFiles/pages/frames/main/frames.html\} \\ 
\item \textbf{Vulnerability Class:} \{['Cryptographic Issues']\} \\ 
\item \textbf{Start:} \{N/A\} \\ 
\item \textbf{End:} \{N/A\} \\ 
\item \textbf{Message:} \{This tag is missing an 'integrity' subresource integrity attribute. The 'integrity' attribute allows for the browser to verify that externally hosted files (for example from a CDN) are delivered without unexpected manipulation. Without this attribute, if an attacker can modify the externally hosted resource, this could lead to XSS and other types of attacks. To prevent this, include the base64-encoded cryptographic hash of the resource (file) you�re telling the browser to fetch in the 'integrity' attribute for all externally hosted files.\} \\ 
\item \textbf{Path:} \{/mnt/c/Users/Administrator/source/repos/WebScan/ScrapedFiles/pages/frames/main/frames.html\} \\ 
\item \textbf{Vulnerability Class:} \{['Cryptographic Issues']\} \\ 
\item \textbf{Start:} \{N/A\} \\ 
\item \textbf{End:} \{N/A\} \\ 
\item \textbf{Message:} \{This tag is missing an 'integrity' subresource integrity attribute. The 'integrity' attribute allows for the browser to verify that externally hosted files (for example from a CDN) are delivered without unexpected manipulation. Without this attribute, if an attacker can modify the externally hosted resource, this could lead to XSS and other types of attacks. To prevent this, include the base64-encoded cryptographic hash of the resource (file) you�re telling the browser to fetch in the 'integrity' attribute for all externally hosted files.\} \\ 
\item \textbf{Path:} \{/mnt/c/Users/Administrator/source/repos/WebScan/ScrapedFiles/pages/frames/main/frames.html\} \\ 
\item \textbf{Vulnerability Class:} \{['Cryptographic Issues']\} \\ 
\item \textbf{Start:} \{N/A\} \\ 
\item \textbf{End:} \{N/A\} \\ 
\item \textbf{Message:} \{This tag is missing an 'integrity' subresource integrity attribute. The 'integrity' attribute allows for the browser to verify that externally hosted files (for example from a CDN) are delivered without unexpected manipulation. Without this attribute, if an attacker can modify the externally hosted resource, this could lead to XSS and other types of attacks. To prevent this, include the base64-encoded cryptographic hash of the resource (file) you�re telling the browser to fetch in the 'integrity' attribute for all externally hosted files.\} \\ 
\item \textbf{Path:} \{/mnt/c/Users/Administrator/source/repos/WebScan/ScrapedFiles/pages/frames/main/frames.html\} \\ 
\item \textbf{Vulnerability Class:} \{['Cryptographic Issues']\} \\ 
\item \textbf{Start:} \{N/A\} \\ 
\item \textbf{End:} \{N/A\} \\ 
\item \textbf{Message:} \{This tag is missing an 'integrity' subresource integrity attribute. The 'integrity' attribute allows for the browser to verify that externally hosted files (for example from a CDN) are delivered without unexpected manipulation. Without this attribute, if an attacker can modify the externally hosted resource, this could lead to XSS and other types of attacks. To prevent this, include the base64-encoded cryptographic hash of the resource (file) you�re telling the browser to fetch in the 'integrity' attribute for all externally hosted files.\} \\ 
\item \textbf{Path:} \{/mnt/c/Users/Administrator/source/repos/WebScan/ScrapedFiles/pages/frames/main/frames.html\} \\ 
\item \textbf{Vulnerability Class:} \{['Cryptographic Issues']\} \\ 
\item \textbf{Start:} \{N/A\} \\ 
\item \textbf{End:} \{N/A\} \\ 
\item \textbf{Message:} \{This tag is missing an 'integrity' subresource integrity attribute. The 'integrity' attribute allows for the browser to verify that externally hosted files (for example from a CDN) are delivered without unexpected manipulation. Without this attribute, if an attacker can modify the externally hosted resource, this could lead to XSS and other types of attacks. To prevent this, include the base64-encoded cryptographic hash of the resource (file) you�re telling the browser to fetch in the 'integrity' attribute for all externally hosted files.\} \\ 
\item \textbf{Path:} \{/mnt/c/Users/Administrator/source/repos/WebScan/ScrapedFiles/pages/frames/main/frames.html\} \\ 
\item \textbf{Vulnerability Class:} \{['Cryptographic Issues']\} \\ 
\item \textbf{Start:} \{N/A\} \\ 
\item \textbf{End:} \{N/A\} \\ 
\item \textbf{Message:} \{This tag is missing an 'integrity' subresource integrity attribute. The 'integrity' attribute allows for the browser to verify that externally hosted files (for example from a CDN) are delivered without unexpected manipulation. Without this attribute, if an attacker can modify the externally hosted resource, this could lead to XSS and other types of attacks. To prevent this, include the base64-encoded cryptographic hash of the resource (file) you�re telling the browser to fetch in the 'integrity' attribute for all externally hosted files.\} \\ 
\item \textbf{Path:} \{/mnt/c/Users/Administrator/source/repos/WebScan/ScrapedFiles/pages/frames/main/frames.html\} \\ 
\item \textbf{Vulnerability Class:} \{['Cryptographic Issues']\} \\ 
\item \textbf{Start:} \{N/A\} \\ 
\item \textbf{End:} \{N/A\} \\ 
\item \textbf{Message:} \{This tag is missing an 'integrity' subresource integrity attribute. The 'integrity' attribute allows for the browser to verify that externally hosted files (for example from a CDN) are delivered without unexpected manipulation. Without this attribute, if an attacker can modify the externally hosted resource, this could lead to XSS and other types of attacks. To prevent this, include the base64-encoded cryptographic hash of the resource (file) you�re telling the browser to fetch in the 'integrity' attribute for all externally hosted files.\} \\ 
\item \textbf{Path:} \{/mnt/c/Users/Administrator/source/repos/WebScan/ScrapedFiles/pages/main/analytics.js\} \\ 
\item \textbf{Vulnerability Class:} \{['Denial-of-Service (DoS)']\} \\ 
\item \textbf{Start:} \{N/A\} \\ 
\item \textbf{End:} \{N/A\} \\ 
\item \textbf{Message:} \{RegExp() called with a `a` function argument, this might allow an attacker to cause a Regular Expression Denial-of-Service (ReDoS) within your application as RegExP blocks the main thread. For this reason, it is recommended to use hardcoded regexes instead. If your regex is run on user-controlled input, consider performing input validation or use a regex checking/sanitization library such as https://www.npmjs.com/package/recheck to verify that the regex does not appear vulnerable to ReDoS.\} \\ 
\item \textbf{Path:} \{/mnt/c/Users/Administrator/source/repos/WebScan/ScrapedFiles/pages/main/analytics.js\} \\ 
\item \textbf{Vulnerability Class:} \{['Denial-of-Service (DoS)']\} \\ 
\item \textbf{Start:} \{N/A\} \\ 
\item \textbf{End:} \{N/A\} \\ 
\item \textbf{Message:} \{RegExp() called with a `a` function argument, this might allow an attacker to cause a Regular Expression Denial-of-Service (ReDoS) within your application as RegExP blocks the main thread. For this reason, it is recommended to use hardcoded regexes instead. If your regex is run on user-controlled input, consider performing input validation or use a regex checking/sanitization library such as https://www.npmjs.com/package/recheck to verify that the regex does not appear vulnerable to ReDoS.\} \\ 
\item \textbf{Path:} \{/mnt/c/Users/Administrator/source/repos/WebScan/ScrapedFiles/pages/main/analytics.js\} \\ 
\item \textbf{Vulnerability Class:} \{['Denial-of-Service (DoS)']\} \\ 
\item \textbf{Start:} \{N/A\} \\ 
\item \textbf{End:} \{N/A\} \\ 
\item \textbf{Message:} \{RegExp() called with a `a` function argument, this might allow an attacker to cause a Regular Expression Denial-of-Service (ReDoS) within your application as RegExP blocks the main thread. For this reason, it is recommended to use hardcoded regexes instead. If your regex is run on user-controlled input, consider performing input validation or use a regex checking/sanitization library such as https://www.npmjs.com/package/recheck to verify that the regex does not appear vulnerable to ReDoS.\} \\ 
\item \textbf{Path:} \{/mnt/c/Users/Administrator/source/repos/WebScan/ScrapedFiles/pages/main/analytics.js\} \\ 
\item \textbf{Vulnerability Class:} \{['Improper Encoding']\} \\ 
\item \textbf{Start:} \{N/A\} \\ 
\item \textbf{End:} \{N/A\} \\ 
\item \textbf{Message:} \{`"https://www.google.%/ads/ga-audiences".replace` method will only replace the first occurrence when used with a string argument ("%"). If this method is used for escaping of dangerous data then there is a possibility for a bypass. Try to use sanitization library instead or use a Regex with a global flag.\} \\ 
\item \textbf{Path:} \{/mnt/c/Users/Administrator/source/repos/WebScan/ScrapedFiles/pages/main/analytics.js\} \\ 
\item \textbf{Vulnerability Class:} \{['Denial-of-Service (DoS)']\} \\ 
\item \textbf{Start:} \{N/A\} \\ 
\item \textbf{End:} \{N/A\} \\ 
\item \textbf{Message:} \{RegExp() called with a `a` function argument, this might allow an attacker to cause a Regular Expression Denial-of-Service (ReDoS) within your application as RegExP blocks the main thread. For this reason, it is recommended to use hardcoded regexes instead. If your regex is run on user-controlled input, consider performing input validation or use a regex checking/sanitization library such as https://www.npmjs.com/package/recheck to verify that the regex does not appear vulnerable to ReDoS.\} \\ 
\item \textbf{Path:} \{/mnt/c/Users/Administrator/source/repos/WebScan/ScrapedFiles/pages/main/conversion.js\} \\ 
\item \textbf{Vulnerability Class:} \{['Mass Assignment']\} \\ 
\item \textbf{Start:} \{N/A\} \\ 
\item \textbf{End:} \{N/A\} \\ 
\item \textbf{Message:} \{Possibility of prototype polluting function detected. By adding or modifying attributes of an object prototype, it is possible to create attributes that exist on every object, or replace critical attributes with malicious ones. This can be problematic if the software depends on existence or non-existence of certain attributes, or uses pre-defined attributes of object prototype (such as hasOwnProperty, toString or valueOf). Possible mitigations might be: freezing the object prototype, using an object without prototypes (via Object.create(null) ), blocking modifications of attributes that resolve to object prototype, using Map instead of object.\} \\ 
\item \textbf{Path:} \{/mnt/c/Users/Administrator/source/repos/WebScan/ScrapedFiles/pages/main/conversion.js\} \\ 
\item \textbf{Vulnerability Class:} \{['Denial-of-Service (DoS)']\} \\ 
\item \textbf{Start:} \{N/A\} \\ 
\item \textbf{End:} \{N/A\} \\ 
\item \textbf{Message:} \{RegExp() called with a `a` function argument, this might allow an attacker to cause a Regular Expression Denial-of-Service (ReDoS) within your application as RegExP blocks the main thread. For this reason, it is recommended to use hardcoded regexes instead. If your regex is run on user-controlled input, consider performing input validation or use a regex checking/sanitization library such as https://www.npmjs.com/package/recheck to verify that the regex does not appear vulnerable to ReDoS.\} \\ 
\item \textbf{Path:} \{/mnt/c/Users/Administrator/source/repos/WebScan/ScrapedFiles/pages/main/conversion.js\} \\ 
\item \textbf{Vulnerability Class:} \{['Denial-of-Service (DoS)']\} \\ 
\item \textbf{Start:} \{N/A\} \\ 
\item \textbf{End:} \{N/A\} \\ 
\item \textbf{Message:} \{RegExp() called with a `a` function argument, this might allow an attacker to cause a Regular Expression Denial-of-Service (ReDoS) within your application as RegExP blocks the main thread. For this reason, it is recommended to use hardcoded regexes instead. If your regex is run on user-controlled input, consider performing input validation or use a regex checking/sanitization library such as https://www.npmjs.com/package/recheck to verify that the regex does not appear vulnerable to ReDoS.\} \\ 
\item \textbf{Path:} \{/mnt/c/Users/Administrator/source/repos/WebScan/ScrapedFiles/pages/main/conversion.js\} \\ 
\item \textbf{Vulnerability Class:} \{['Denial-of-Service (DoS)']\} \\ 
\item \textbf{Start:} \{N/A\} \\ 
\item \textbf{End:} \{N/A\} \\ 
\item \textbf{Message:} \{RegExp() called with a `b` function argument, this might allow an attacker to cause a Regular Expression Denial-of-Service (ReDoS) within your application as RegExP blocks the main thread. For this reason, it is recommended to use hardcoded regexes instead. If your regex is run on user-controlled input, consider performing input validation or use a regex checking/sanitization library such as https://www.npmjs.com/package/recheck to verify that the regex does not appear vulnerable to ReDoS.\} \\ 
\item \textbf{Path:} \{/mnt/c/Users/Administrator/source/repos/WebScan/ScrapedFiles/pages/main/conversion.js\} \\ 
\item \textbf{Vulnerability Class:} \{['Cross-Site-Scripting (XSS)']\} \\ 
\item \textbf{Start:} \{N/A\} \\ 
\item \textbf{End:} \{N/A\} \\ 
\item \textbf{Message:} \{User controlled data in methods like `innerHTML`, `outerHTML` or `document.write` is an anti-pattern that can lead to XSS vulnerabilities\} \\ 
\item \textbf{Path:} \{/mnt/c/Users/Administrator/source/repos/WebScan/ScrapedFiles/pages/main/pages.html\} \\ 
\item \textbf{Vulnerability Class:} \{['Cryptographic Issues']\} \\ 
\item \textbf{Start:} \{N/A\} \\ 
\item \textbf{End:} \{N/A\} \\ 
\item \textbf{Message:} \{This tag is missing an 'integrity' subresource integrity attribute. The 'integrity' attribute allows for the browser to verify that externally hosted files (for example from a CDN) are delivered without unexpected manipulation. Without this attribute, if an attacker can modify the externally hosted resource, this could lead to XSS and other types of attacks. To prevent this, include the base64-encoded cryptographic hash of the resource (file) you�re telling the browser to fetch in the 'integrity' attribute for all externally hosted files.\} \\ 
\item \textbf{Path:} \{/mnt/c/Users/Administrator/source/repos/WebScan/ScrapedFiles/pages/main/pages.html\} \\ 
\item \textbf{Vulnerability Class:} \{['Cryptographic Issues']\} \\ 
\item \textbf{Start:} \{N/A\} \\ 
\item \textbf{End:} \{N/A\} \\ 
\item \textbf{Message:} \{This tag is missing an 'integrity' subresource integrity attribute. The 'integrity' attribute allows for the browser to verify that externally hosted files (for example from a CDN) are delivered without unexpected manipulation. Without this attribute, if an attacker can modify the externally hosted resource, this could lead to XSS and other types of attacks. To prevent this, include the base64-encoded cryptographic hash of the resource (file) you�re telling the browser to fetch in the 'integrity' attribute for all externally hosted files.\} \\ 
\item \textbf{Path:} \{/mnt/c/Users/Administrator/source/repos/WebScan/ScrapedFiles/pages/main/pages.html\} \\ 
\item \textbf{Vulnerability Class:} \{['Cryptographic Issues']\} \\ 
\item \textbf{Start:} \{N/A\} \\ 
\item \textbf{End:} \{N/A\} \\ 
\item \textbf{Message:} \{This tag is missing an 'integrity' subresource integrity attribute. The 'integrity' attribute allows for the browser to verify that externally hosted files (for example from a CDN) are delivered without unexpected manipulation. Without this attribute, if an attacker can modify the externally hosted resource, this could lead to XSS and other types of attacks. To prevent this, include the base64-encoded cryptographic hash of the resource (file) you�re telling the browser to fetch in the 'integrity' attribute for all externally hosted files.\} \\ 
\item \textbf{Path:} \{/mnt/c/Users/Administrator/source/repos/WebScan/ScrapedFiles/pages/main/pages.html\} \\ 
\item \textbf{Vulnerability Class:} \{['Cryptographic Issues']\} \\ 
\item \textbf{Start:} \{N/A\} \\ 
\item \textbf{End:} \{N/A\} \\ 
\item \textbf{Message:} \{This tag is missing an 'integrity' subresource integrity attribute. The 'integrity' attribute allows for the browser to verify that externally hosted files (for example from a CDN) are delivered without unexpected manipulation. Without this attribute, if an attacker can modify the externally hosted resource, this could lead to XSS and other types of attacks. To prevent this, include the base64-encoded cryptographic hash of the resource (file) you�re telling the browser to fetch in the 'integrity' attribute for all externally hosted files.\} \\ 
\item \textbf{Path:} \{/mnt/c/Users/Administrator/source/repos/WebScan/ScrapedFiles/pages/main/pages.html\} \\ 
\item \textbf{Vulnerability Class:} \{['Cryptographic Issues']\} \\ 
\item \textbf{Start:} \{N/A\} \\ 
\item \textbf{End:} \{N/A\} \\ 
\item \textbf{Message:} \{This tag is missing an 'integrity' subresource integrity attribute. The 'integrity' attribute allows for the browser to verify that externally hosted files (for example from a CDN) are delivered without unexpected manipulation. Without this attribute, if an attacker can modify the externally hosted resource, this could lead to XSS and other types of attacks. To prevent this, include the base64-encoded cryptographic hash of the resource (file) you�re telling the browser to fetch in the 'integrity' attribute for all externally hosted files.\} \\ 
\item \textbf{Path:} \{/mnt/c/Users/Administrator/source/repos/WebScan/ScrapedFiles/pages/main/pages.html\} \\ 
\item \textbf{Vulnerability Class:} \{['Cryptographic Issues']\} \\ 
\item \textbf{Start:} \{N/A\} \\ 
\item \textbf{End:} \{N/A\} \\ 
\item \textbf{Message:} \{This tag is missing an 'integrity' subresource integrity attribute. The 'integrity' attribute allows for the browser to verify that externally hosted files (for example from a CDN) are delivered without unexpected manipulation. Without this attribute, if an attacker can modify the externally hosted resource, this could lead to XSS and other types of attacks. To prevent this, include the base64-encoded cryptographic hash of the resource (file) you�re telling the browser to fetch in the 'integrity' attribute for all externally hosted files.\} \\ 
\item \textbf{Path:} \{/mnt/c/Users/Administrator/source/repos/WebScan/ScrapedFiles/pages/main/pages.html\} \\ 
\item \textbf{Vulnerability Class:} \{['Cryptographic Issues']\} \\ 
\item \textbf{Start:} \{N/A\} \\ 
\item \textbf{End:} \{N/A\} \\ 
\item \textbf{Message:} \{This tag is missing an 'integrity' subresource integrity attribute. The 'integrity' attribute allows for the browser to verify that externally hosted files (for example from a CDN) are delivered without unexpected manipulation. Without this attribute, if an attacker can modify the externally hosted resource, this could lead to XSS and other types of attacks. To prevent this, include the base64-encoded cryptographic hash of the resource (file) you�re telling the browser to fetch in the 'integrity' attribute for all externally hosted files.\} \\ 

\end{itemize}
\newpage  % Page break

\section{Dynamic Analysis}
Details about dynamic analysis...

\end{document}