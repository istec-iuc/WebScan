
\documentclass[12pt]{article}
\usepackage[utf8]{inputenc}
\usepackage{geometry}
\usepackage{graphicx}
\usepackage{float}
\usepackage{placeins}
\usepackage{hyperref}
\usepackage{longtable}
\geometry{margin=1in}


\title{\textbf{Static and Dynamic Analysis}}
\author{ISTEC-Cyber Security}
\date{\today} % Automatically insert today's date

\begin{document}

\maketitle

\section*{Description}
This report contains static and dynamic analysis of the target. It uses Semgrep for static analysis and OWASP ZAP, Nmap, and SQLMap for dynamic analysis.

\vspace{10cm} % Space to push the footer down

\begin{center}
\textbf{Provided by} \\[1em]
\includegraphics[width=0.1\textwidth]{logo75.png}
\end{center}

\newpage % Page break

\section{Static Analysis}
Details about static analysis...

\section{Analysis Report}

\subsection{Risk Summary}
\begin{table}[h!]
\centering
\renewcommand{\arraystretch}{1.5}
\begin{tabular}{|c|c|}
\hline
\textbf{Risk Level} & \textbf{Number of Findings} \\
\hline
Low Risk & 1465 \\ 
\hline
Medium Risk & 200 \\ 
\hline
High Risk & 0 \\ 
\hline
Critical Risk & 0 \\ 
\hline
\end{tabular}
\caption{Summary of Risk Findings}
\label{tab:risk_summary}
\end{table}

\subsection{Vulnerability Categories}
\begin{itemize}
\item Category 1: Cryptographic Issues | 1229
\item Category 2: Cross-Site Request Forgery (CSRF) | 75
\item Category 3: Hard-coded Secrets | 51
\item Category 4: Mishandled Sensitive Information | 128
\item Category 5: Denial-of-Service (DoS) | 45
\item Category 6: Improper Authentication | 29
\item Category 7: Improper Validation | 27
\item Category 8: Cross-Site-Scripting (XSS) | 18
\item Category 9: Code Injection | 18
\item Category 10: Mass Assignment | 45

% Insert vulnerability categories here
\end{itemize}

\subsection{Vulnerabilities by Page}
%Vulnerabilities by Page:

%--------------------------------------------------------Dynamic Analysis-----------------------------------------------
\newpage
\section{Dynamic Analysis}
Details about dynamic analysis...

\section{Analysis Report}

\subsection{Nmap Scan Results}

\subsection{SQLMap Injection Points}

\subsection{Risk Summary}
\begin{table}[h!]
\centering
\renewcommand{\arraystretch}{1.5}
\begin{tabular}{|c|c|}
\hline
\textbf{Risk Level} & \textbf{Number of Findings} \\
\hline
Low Risk & zaplc \\ 
\hline
Medium Risk & zapmc \\ 
\hline
High Risk & zaphc \\ 
\hline
Critical Risk & zapcc \\ 
\hline
\end{tabular}
\caption{Summary of Risk Findings}
\label{tab:risk_summary}
\end{table}

\subsection{Vulnerability Categories}
\begin{itemize}
\item ZapCategories:
% Insert vulnerability categories here
\end{itemize}

\subsection{Vulnerabilities by Page}
%ZapVulnerabilities by Page:
\end{document}
