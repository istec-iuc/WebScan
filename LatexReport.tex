
\documentclass[12pt]{article}
\usepackage[utf8]{inputenc}
\usepackage{geometry}
\usepackage{graphicx}
\usepackage{float}
\usepackage{placeins}
\usepackage{hyperref}
\usepackage{longtable}
\geometry{margin=1in}


\title{\textbf{Static and Dynamic Analysis}}
\author{ISTEC-Cyber Security}
\date{\today} % Automatically insert today's date

\begin{document}

\maketitle

\section*{Description}
This report contains static and dynamic analysis of the target. It uses Semgrep for static analysis and OWASP ZAP, Nmap, and SQLMap for dynamic analysis.

\vspace{10cm} % Space to push the footer down

\begin{center}
\textbf{Provided by} \\[1em]
\includegraphics[width=0.1\textwidth]{logo75.png}
\end{center}

\newpage % Page break

\section{Static Analysis}
Details about static analysis...

\section{Analysis Report}

\subsection{Risk Summary}
\begin{table}[h!]
\centering
\renewcommand{\arraystretch}{1.5}
\begin{tabular}{|c|c|}
\hline
\textbf{Risk Level} & \textbf{Number of Findings} \\
\hline
Low Risk & 76 \\ 
\hline
Medium Risk & 47 \\ 
\hline
High Risk & 0 \\ 
\hline
Critical Risk & 0 \\ 
\hline
\end{tabular}
\caption{Summary of Risk Findings}
\label{tab:risk_summary}
\end{table}

\subsection{Vulnerability Categories}
\begin{itemize}
\item Category 1: Denial-of-Service (DoS) | 46
\item Category 2: Improper Encoding | 7
\item Category 3: Mass Assignment | 6
\item Category 4: Cross-Site-Scripting (XSS) | 7
\item Category 5: Cryptographic Issues | 54
\item Category 6: Improper Authentication | 1
\item Category 7: Cross-Site Request Forgery (CSRF) | 1
\item Category 8: Mishandled Sensitive Information | 1

% Insert vulnerability categories here
\end{itemize}

\subsection{Vulnerabilities by Page}
\begin{table}[!h]
\centering
\renewcommand{\arraystretch}{1.3}
\begin{tabular}{|l|p{10cm}|}
\hline
\multicolumn{2}{|c|}{\textbf{Vulnerability 1}} \\
\hline
\multicolumn{2}{|l|}{\textbf{Path}: ScrapedFiles/faq/main/analytics.js} \\
\hline
\textbf{Vulnerability Class} & ['Denial-of-Service (DoS)'] \\
\hline
\textbf{Start} & line: 10 \quad col: 486 \\
\hline
\textbf{End} & line: 10 \quad col: 529 \\
\hline
\textbf{Message} & RegExp() called with a `a` function argument, this might allow an attacker to cause a Regular Expression Denial-of-Service (ReDoS) within your application as RegExP blocks the main thread. For this reason, it is recommended to use hardcoded regexes instead. If your regex is run on user-controlled input, consider performing input validation or use a regex checking/sanitization library such as https://www.npmjs.com/package/recheck to verify that the regex does not appear vulnerable to ReDoS. \\
\hline
\end{tabular}
\end{table}
\vspace{0.7cm}
\FloatBarrier
\begin{table}[!h]
\centering
\renewcommand{\arraystretch}{1.3}
\begin{tabular}{|l|p{10cm}|}
\hline
\multicolumn{2}{|c|}{\textbf{Vulnerability 2}} \\
\hline
\multicolumn{2}{|l|}{\textbf{Path}: ScrapedFiles/faq/main/analytics.js} \\
\hline
\textbf{Vulnerability Class} & ['Denial-of-Service (DoS)'] \\
\hline
\textbf{Start} & line: 27 \quad col: 150 \\
\hline
\textbf{End} & line: 27 \quad col: 189 \\
\hline
\textbf{Message} & RegExp() called with a `a` function argument, this might allow an attacker to cause a Regular Expression Denial-of-Service (ReDoS) within your application as RegExP blocks the main thread. For this reason, it is recommended to use hardcoded regexes instead. If your regex is run on user-controlled input, consider performing input validation or use a regex checking/sanitization library such as https://www.npmjs.com/package/recheck to verify that the regex does not appear vulnerable to ReDoS. \\
\hline
\end{tabular}
\end{table}
\vspace{0.7cm}
\FloatBarrier
\begin{table}[!h]
\centering
\renewcommand{\arraystretch}{1.3}
\begin{tabular}{|l|p{10cm}|}
\hline
\multicolumn{2}{|c|}{\textbf{Vulnerability 3}} \\
\hline
\multicolumn{2}{|l|}{\textbf{Path}: ScrapedFiles/faq/main/analytics.js} \\
\hline
\textbf{Vulnerability Class} & ['Denial-of-Service (DoS)'] \\
\hline
\textbf{Start} & line: 28 \quad col: 304 \\
\hline
\textbf{End} & line: 28 \quad col: 368 \\
\hline
\textbf{Message} & RegExp() called with a `a` function argument, this might allow an attacker to cause a Regular Expression Denial-of-Service (ReDoS) within your application as RegExP blocks the main thread. For this reason, it is recommended to use hardcoded regexes instead. If your regex is run on user-controlled input, consider performing input validation or use a regex checking/sanitization library such as https://www.npmjs.com/package/recheck to verify that the regex does not appear vulnerable to ReDoS. \\
\hline
\end{tabular}
\end{table}
\vspace{0.7cm}
\FloatBarrier
\begin{table}[!h]
\centering
\renewcommand{\arraystretch}{1.3}
\begin{tabular}{|l|p{10cm}|}
\hline
\multicolumn{2}{|c|}{\textbf{Vulnerability 4}} \\
\hline
\multicolumn{2}{|l|}{\textbf{Path}: ScrapedFiles/faq/main/analytics.js} \\
\hline
\textbf{Vulnerability Class} & ['Improper Encoding'] \\
\hline
\textbf{Start} & line: 37 \quad col: 130 \\
\hline
\textbf{End} & line: 37 \quad col: 184 \\
\hline
\textbf{Message} & `"https://www.google.\%/ads/ga-audiences".replace` method will only replace the first occurrence when used with a string argument ("\%"). If this method is used for escaping of dangerous data then there is a possibility for a bypass. Try to use sanitization library instead or use a Regex with a global flag. \\
\hline
\end{tabular}
\end{table}
\vspace{0.7cm}
\FloatBarrier
\begin{table}[!h]
\centering
\renewcommand{\arraystretch}{1.3}
\begin{tabular}{|l|p{10cm}|}
\hline
\multicolumn{2}{|c|}{\textbf{Vulnerability 5}} \\
\hline
\multicolumn{2}{|l|}{\textbf{Path}: ScrapedFiles/faq/main/analytics.js} \\
\hline
\textbf{Vulnerability Class} & ['Denial-of-Service (DoS)'] \\
\hline
\textbf{Start} & line: 48 \quad col: 629 \\
\hline
\textbf{End} & line: 48 \quad col: 650 \\
\hline
\textbf{Message} & RegExp() called with a `a` function argument, this might allow an attacker to cause a Regular Expression Denial-of-Service (ReDoS) within your application as RegExP blocks the main thread. For this reason, it is recommended to use hardcoded regexes instead. If your regex is run on user-controlled input, consider performing input validation or use a regex checking/sanitization library such as https://www.npmjs.com/package/recheck to verify that the regex does not appear vulnerable to ReDoS. \\
\hline
\end{tabular}
\end{table}
\vspace{0.7cm}
\FloatBarrier
\begin{table}[!h]
\centering
\renewcommand{\arraystretch}{1.3}
\begin{tabular}{|l|p{10cm}|}
\hline
\multicolumn{2}{|c|}{\textbf{Vulnerability 6}} \\
\hline
\multicolumn{2}{|l|}{\textbf{Path}: ScrapedFiles/faq/main/conversion.js} \\
\hline
\textbf{Vulnerability Class} & ['Mass Assignment'] \\
\hline
\textbf{Start} & line: 3 \quad col: 152 \\
\hline
\textbf{End} & line: 3 \quad col: 158 \\
\hline
\textbf{Message} & Possibility of prototype polluting function detected. By adding or modifying attributes of an object prototype, it is possible to create attributes that exist on every object, or replace critical attributes with malicious ones. This can be problematic if the software depends on existence or non-existence of certain attributes, or uses pre-defined attributes of object prototype (such as hasOwnProperty, toString or valueOf). Possible mitigations might be: freezing the object prototype, using an object without prototypes (via Object.create(null) ), blocking modifications of attributes that resolve to object prototype, using Map instead of object. \\
\hline
\end{tabular}
\end{table}
\vspace{0.7cm}
\FloatBarrier
\begin{table}[!h]
\centering
\renewcommand{\arraystretch}{1.3}
\begin{tabular}{|l|p{10cm}|}
\hline
\multicolumn{2}{|c|}{\textbf{Vulnerability 7}} \\
\hline
\multicolumn{2}{|l|}{\textbf{Path}: ScrapedFiles/faq/main/conversion.js} \\
\hline
\textbf{Vulnerability Class} & ['Denial-of-Service (DoS)'] \\
\hline
\textbf{Start} & line: 34 \quad col: 97 \\
\hline
\textbf{End} & line: 34 \quad col: 134 \\
\hline
\textbf{Message} & RegExp() called with a `a` function argument, this might allow an attacker to cause a Regular Expression Denial-of-Service (ReDoS) within your application as RegExP blocks the main thread. For this reason, it is recommended to use hardcoded regexes instead. If your regex is run on user-controlled input, consider performing input validation or use a regex checking/sanitization library such as https://www.npmjs.com/package/recheck to verify that the regex does not appear vulnerable to ReDoS. \\
\hline
\end{tabular}
\end{table}
\vspace{0.7cm}
\FloatBarrier
\begin{table}[!h]
\centering
\renewcommand{\arraystretch}{1.3}
\begin{tabular}{|l|p{10cm}|}
\hline
\multicolumn{2}{|c|}{\textbf{Vulnerability 8}} \\
\hline
\multicolumn{2}{|l|}{\textbf{Path}: ScrapedFiles/faq/main/conversion.js} \\
\hline
\textbf{Vulnerability Class} & ['Denial-of-Service (DoS)'] \\
\hline
\textbf{Start} & line: 39 \quad col: 1267 \\
\hline
\textbf{End} & line: 39 \quad col: 1331 \\
\hline
\textbf{Message} & RegExp() called with a `a` function argument, this might allow an attacker to cause a Regular Expression Denial-of-Service (ReDoS) within your application as RegExP blocks the main thread. For this reason, it is recommended to use hardcoded regexes instead. If your regex is run on user-controlled input, consider performing input validation or use a regex checking/sanitization library such as https://www.npmjs.com/package/recheck to verify that the regex does not appear vulnerable to ReDoS. \\
\hline
\end{tabular}
\end{table}
\vspace{0.7cm}
\FloatBarrier
\begin{table}[!h]
\centering
\renewcommand{\arraystretch}{1.3}
\begin{tabular}{|l|p{10cm}|}
\hline
\multicolumn{2}{|c|}{\textbf{Vulnerability 9}} \\
\hline
\multicolumn{2}{|l|}{\textbf{Path}: ScrapedFiles/faq/main/conversion.js} \\
\hline
\textbf{Vulnerability Class} & ['Denial-of-Service (DoS)'] \\
\hline
\textbf{Start} & line: 79 \quad col: 343 \\
\hline
\textbf{End} & line: 79 \quad col: 372 \\
\hline
\textbf{Message} & RegExp() called with a `b` function argument, this might allow an attacker to cause a Regular Expression Denial-of-Service (ReDoS) within your application as RegExP blocks the main thread. For this reason, it is recommended to use hardcoded regexes instead. If your regex is run on user-controlled input, consider performing input validation or use a regex checking/sanitization library such as https://www.npmjs.com/package/recheck to verify that the regex does not appear vulnerable to ReDoS. \\
\hline
\end{tabular}
\end{table}
\vspace{0.7cm}
\FloatBarrier
\begin{table}[!h]
\centering
\renewcommand{\arraystretch}{1.3}
\begin{tabular}{|l|p{10cm}|}
\hline
\multicolumn{2}{|c|}{\textbf{Vulnerability 10}} \\
\hline
\multicolumn{2}{|l|}{\textbf{Path}: ScrapedFiles/faq/main/conversion.js} \\
\hline
\textbf{Vulnerability Class} & ['Cross-Site-Scripting (XSS)'] \\
\hline
\textbf{Start} & line: 117 \quad col: 382 \\
\hline
\textbf{End} & line: 117 \quad col: 399 \\
\hline
\textbf{Message} & User controlled data in methods like `innerHTML`, `outerHTML` or `document.write` is an anti-pattern that can lead to XSS vulnerabilities \\
\hline
\end{tabular}
\end{table}
\vspace{0.7cm}
\FloatBarrier
\begin{table}[!h]
\centering
\renewcommand{\arraystretch}{1.3}
\begin{tabular}{|l|p{10cm}|}
\hline
\multicolumn{2}{|c|}{\textbf{Vulnerability 11}} \\
\hline
\multicolumn{2}{|l|}{\textbf{Path}: ScrapedFiles/faq/main/faq.html} \\
\hline
\textbf{Vulnerability Class} & ['Cryptographic Issues'] \\
\hline
\textbf{Start} & line: 58 \quad col: 141 \\
\hline
\textbf{End} & line: 58 \quad col: 1070 \\
\hline
\textbf{Message} & This tag is missing an 'integrity' subresource integrity attribute. The 'integrity' attribute allows for the browser to verify that externally hosted files (for example from a CDN) are delivered without unexpected manipulation. Without this attribute, if an attacker can modify the externally hosted resource, this could lead to XSS and other types of attacks. To prevent this, include the base64-encoded cryptographic hash of the resource (file) you're telling the browser to fetch in the 'integrity' attribute for all externally hosted files. \\
\hline
\end{tabular}
\end{table}
\vspace{0.7cm}
\FloatBarrier
\begin{table}[!h]
\centering
\renewcommand{\arraystretch}{1.3}
\begin{tabular}{|l|p{10cm}|}
\hline
\multicolumn{2}{|c|}{\textbf{Vulnerability 12}} \\
\hline
\multicolumn{2}{|l|}{\textbf{Path}: ScrapedFiles/faq/main/faq.html} \\
\hline
\textbf{Vulnerability Class} & ['Cryptographic Issues'] \\
\hline
\textbf{Start} & line: 58 \quad col: 1070 \\
\hline
\textbf{End} & line: 58 \quad col: 1149 \\
\hline
\textbf{Message} & This tag is missing an 'integrity' subresource integrity attribute. The 'integrity' attribute allows for the browser to verify that externally hosted files (for example from a CDN) are delivered without unexpected manipulation. Without this attribute, if an attacker can modify the externally hosted resource, this could lead to XSS and other types of attacks. To prevent this, include the base64-encoded cryptographic hash of the resource (file) you're telling the browser to fetch in the 'integrity' attribute for all externally hosted files. \\
\hline
\end{tabular}
\end{table}
\vspace{0.7cm}
\FloatBarrier
\begin{table}[!h]
\centering
\renewcommand{\arraystretch}{1.3}
\begin{tabular}{|l|p{10cm}|}
\hline
\multicolumn{2}{|c|}{\textbf{Vulnerability 13}} \\
\hline
\multicolumn{2}{|l|}{\textbf{Path}: ScrapedFiles/faq/main/faq.html} \\
\hline
\textbf{Vulnerability Class} & ['Cryptographic Issues'] \\
\hline
\textbf{Start} & line: 123 \quad col: 3 \\
\hline
\textbf{End} & line: 123 \quad col: 92 \\
\hline
\textbf{Message} & This tag is missing an 'integrity' subresource integrity attribute. The 'integrity' attribute allows for the browser to verify that externally hosted files (for example from a CDN) are delivered without unexpected manipulation. Without this attribute, if an attacker can modify the externally hosted resource, this could lead to XSS and other types of attacks. To prevent this, include the base64-encoded cryptographic hash of the resource (file) you're telling the browser to fetch in the 'integrity' attribute for all externally hosted files. \\
\hline
\end{tabular}
\end{table}
\vspace{0.7cm}
\FloatBarrier
\begin{table}[!h]
\centering
\renewcommand{\arraystretch}{1.3}
\begin{tabular}{|l|p{10cm}|}
\hline
\multicolumn{2}{|c|}{\textbf{Vulnerability 14}} \\
\hline
\multicolumn{2}{|l|}{\textbf{Path}: ScrapedFiles/faq/main/faq.html} \\
\hline
\textbf{Vulnerability Class} & ['Cryptographic Issues'] \\
\hline
\textbf{Start} & line: 128 \quad col: 3 \\
\hline
\textbf{End} & line: 128 \quad col: 99 \\
\hline
\textbf{Message} & This tag is missing an 'integrity' subresource integrity attribute. The 'integrity' attribute allows for the browser to verify that externally hosted files (for example from a CDN) are delivered without unexpected manipulation. Without this attribute, if an attacker can modify the externally hosted resource, this could lead to XSS and other types of attacks. To prevent this, include the base64-encoded cryptographic hash of the resource (file) you're telling the browser to fetch in the 'integrity' attribute for all externally hosted files. \\
\hline
\end{tabular}
\end{table}
\vspace{0.7cm}
\FloatBarrier
\begin{table}[!h]
\centering
\renewcommand{\arraystretch}{1.3}
\begin{tabular}{|l|p{10cm}|}
\hline
\multicolumn{2}{|c|}{\textbf{Vulnerability 15}} \\
\hline
\multicolumn{2}{|l|}{\textbf{Path}: ScrapedFiles/faq/main/faq.html} \\
\hline
\textbf{Vulnerability Class} & ['Cryptographic Issues'] \\
\hline
\textbf{Start} & line: 129 \quad col: 3 \\
\hline
\textbf{End} & line: 129 \quad col: 123 \\
\hline
\textbf{Message} & This tag is missing an 'integrity' subresource integrity attribute. The 'integrity' attribute allows for the browser to verify that externally hosted files (for example from a CDN) are delivered without unexpected manipulation. Without this attribute, if an attacker can modify the externally hosted resource, this could lead to XSS and other types of attacks. To prevent this, include the base64-encoded cryptographic hash of the resource (file) you're telling the browser to fetch in the 'integrity' attribute for all externally hosted files. \\
\hline
\end{tabular}
\end{table}
\vspace{0.7cm}
\FloatBarrier
\begin{table}[!h]
\centering
\renewcommand{\arraystretch}{1.3}
\begin{tabular}{|l|p{10cm}|}
\hline
\multicolumn{2}{|c|}{\textbf{Vulnerability 16}} \\
\hline
\multicolumn{2}{|l|}{\textbf{Path}: ScrapedFiles/faq/main/faq.html} \\
\hline
\textbf{Vulnerability Class} & ['Cryptographic Issues'] \\
\hline
\textbf{Start} & line: 197 \quad col: 5 \\
\hline
\textbf{End} & line: 198 \quad col: 14 \\
\hline
\textbf{Message} & This tag is missing an 'integrity' subresource integrity attribute. The 'integrity' attribute allows for the browser to verify that externally hosted files (for example from a CDN) are delivered without unexpected manipulation. Without this attribute, if an attacker can modify the externally hosted resource, this could lead to XSS and other types of attacks. To prevent this, include the base64-encoded cryptographic hash of the resource (file) you're telling the browser to fetch in the 'integrity' attribute for all externally hosted files. \\
\hline
\end{tabular}
\end{table}
\vspace{0.7cm}
\FloatBarrier
\begin{table}[!h]
\centering
\renewcommand{\arraystretch}{1.3}
\begin{tabular}{|l|p{10cm}|}
\hline
\multicolumn{2}{|c|}{\textbf{Vulnerability 17}} \\
\hline
\multicolumn{2}{|l|}{\textbf{Path}: ScrapedFiles/faq/main/faq.html} \\
\hline
\textbf{Vulnerability Class} & ['Cryptographic Issues'] \\
\hline
\textbf{Start} & line: 198 \quad col: 14 \\
\hline
\textbf{End} & line: 198 \quad col: 690 \\
\hline
\textbf{Message} & This tag is missing an 'integrity' subresource integrity attribute. The 'integrity' attribute allows for the browser to verify that externally hosted files (for example from a CDN) are delivered without unexpected manipulation. Without this attribute, if an attacker can modify the externally hosted resource, this could lead to XSS and other types of attacks. To prevent this, include the base64-encoded cryptographic hash of the resource (file) you're telling the browser to fetch in the 'integrity' attribute for all externally hosted files. \\
\hline
\end{tabular}
\end{table}
\vspace{0.7cm}
\FloatBarrier
\begin{table}[!h]
\centering
\renewcommand{\arraystretch}{1.3}
\begin{tabular}{|l|p{10cm}|}
\hline
\multicolumn{2}{|c|}{\textbf{Vulnerability 18}} \\
\hline
\multicolumn{2}{|l|}{\textbf{Path}: ScrapedFiles/faq/main/faq.html} \\
\hline
\textbf{Vulnerability Class} & ['Cryptographic Issues'] \\
\hline
\textbf{Start} & line: 215 \quad col: 1 \\
\hline
\textbf{End} & line: 215 \quad col: 687 \\
\hline
\textbf{Message} & This tag is missing an 'integrity' subresource integrity attribute. The 'integrity' attribute allows for the browser to verify that externally hosted files (for example from a CDN) are delivered without unexpected manipulation. Without this attribute, if an attacker can modify the externally hosted resource, this could lead to XSS and other types of attacks. To prevent this, include the base64-encoded cryptographic hash of the resource (file) you're telling the browser to fetch in the 'integrity' attribute for all externally hosted files. \\
\hline
\end{tabular}
\end{table}
\vspace{0.7cm}
\FloatBarrier
\begin{table}[!h]
\centering
\renewcommand{\arraystretch}{1.3}
\begin{tabular}{|l|p{10cm}|}
\hline
\multicolumn{2}{|c|}{\textbf{Vulnerability 19}} \\
\hline
\multicolumn{2}{|l|}{\textbf{Path}: ScrapedFiles/index/main/analytics.js} \\
\hline
\textbf{Vulnerability Class} & ['Denial-of-Service (DoS)'] \\
\hline
\textbf{Start} & line: 10 \quad col: 486 \\
\hline
\textbf{End} & line: 10 \quad col: 529 \\
\hline
\textbf{Message} & RegExp() called with a `a` function argument, this might allow an attacker to cause a Regular Expression Denial-of-Service (ReDoS) within your application as RegExP blocks the main thread. For this reason, it is recommended to use hardcoded regexes instead. If your regex is run on user-controlled input, consider performing input validation or use a regex checking/sanitization library such as https://www.npmjs.com/package/recheck to verify that the regex does not appear vulnerable to ReDoS. \\
\hline
\end{tabular}
\end{table}
\vspace{0.7cm}
\FloatBarrier
\begin{table}[!h]
\centering
\renewcommand{\arraystretch}{1.3}
\begin{tabular}{|l|p{10cm}|}
\hline
\multicolumn{2}{|c|}{\textbf{Vulnerability 20}} \\
\hline
\multicolumn{2}{|l|}{\textbf{Path}: ScrapedFiles/index/main/analytics.js} \\
\hline
\textbf{Vulnerability Class} & ['Denial-of-Service (DoS)'] \\
\hline
\textbf{Start} & line: 27 \quad col: 150 \\
\hline
\textbf{End} & line: 27 \quad col: 189 \\
\hline
\textbf{Message} & RegExp() called with a `a` function argument, this might allow an attacker to cause a Regular Expression Denial-of-Service (ReDoS) within your application as RegExP blocks the main thread. For this reason, it is recommended to use hardcoded regexes instead. If your regex is run on user-controlled input, consider performing input validation or use a regex checking/sanitization library such as https://www.npmjs.com/package/recheck to verify that the regex does not appear vulnerable to ReDoS. \\
\hline
\end{tabular}
\end{table}
\vspace{0.7cm}
\FloatBarrier
\begin{table}[!h]
\centering
\renewcommand{\arraystretch}{1.3}
\begin{tabular}{|l|p{10cm}|}
\hline
\multicolumn{2}{|c|}{\textbf{Vulnerability 21}} \\
\hline
\multicolumn{2}{|l|}{\textbf{Path}: ScrapedFiles/index/main/analytics.js} \\
\hline
\textbf{Vulnerability Class} & ['Denial-of-Service (DoS)'] \\
\hline
\textbf{Start} & line: 28 \quad col: 304 \\
\hline
\textbf{End} & line: 28 \quad col: 368 \\
\hline
\textbf{Message} & RegExp() called with a `a` function argument, this might allow an attacker to cause a Regular Expression Denial-of-Service (ReDoS) within your application as RegExP blocks the main thread. For this reason, it is recommended to use hardcoded regexes instead. If your regex is run on user-controlled input, consider performing input validation or use a regex checking/sanitization library such as https://www.npmjs.com/package/recheck to verify that the regex does not appear vulnerable to ReDoS. \\
\hline
\end{tabular}
\end{table}
\vspace{0.7cm}
\FloatBarrier
\begin{table}[!h]
\centering
\renewcommand{\arraystretch}{1.3}
\begin{tabular}{|l|p{10cm}|}
\hline
\multicolumn{2}{|c|}{\textbf{Vulnerability 22}} \\
\hline
\multicolumn{2}{|l|}{\textbf{Path}: ScrapedFiles/index/main/analytics.js} \\
\hline
\textbf{Vulnerability Class} & ['Improper Encoding'] \\
\hline
\textbf{Start} & line: 37 \quad col: 130 \\
\hline
\textbf{End} & line: 37 \quad col: 184 \\
\hline
\textbf{Message} & `"https://www.google.\%/ads/ga-audiences".replace` method will only replace the first occurrence when used with a string argument ("\%"). If this method is used for escaping of dangerous data then there is a possibility for a bypass. Try to use sanitization library instead or use a Regex with a global flag. \\
\hline
\end{tabular}
\end{table}
\vspace{0.7cm}
\FloatBarrier
\begin{table}[!h]
\centering
\renewcommand{\arraystretch}{1.3}
\begin{tabular}{|l|p{10cm}|}
\hline
\multicolumn{2}{|c|}{\textbf{Vulnerability 23}} \\
\hline
\multicolumn{2}{|l|}{\textbf{Path}: ScrapedFiles/index/main/analytics.js} \\
\hline
\textbf{Vulnerability Class} & ['Denial-of-Service (DoS)'] \\
\hline
\textbf{Start} & line: 48 \quad col: 629 \\
\hline
\textbf{End} & line: 48 \quad col: 650 \\
\hline
\textbf{Message} & RegExp() called with a `a` function argument, this might allow an attacker to cause a Regular Expression Denial-of-Service (ReDoS) within your application as RegExP blocks the main thread. For this reason, it is recommended to use hardcoded regexes instead. If your regex is run on user-controlled input, consider performing input validation or use a regex checking/sanitization library such as https://www.npmjs.com/package/recheck to verify that the regex does not appear vulnerable to ReDoS. \\
\hline
\end{tabular}
\end{table}
\vspace{0.7cm}
\FloatBarrier
\begin{table}[!h]
\centering
\renewcommand{\arraystretch}{1.3}
\begin{tabular}{|l|p{10cm}|}
\hline
\multicolumn{2}{|c|}{\textbf{Vulnerability 24}} \\
\hline
\multicolumn{2}{|l|}{\textbf{Path}: ScrapedFiles/index/main/conversion.js} \\
\hline
\textbf{Vulnerability Class} & ['Mass Assignment'] \\
\hline
\textbf{Start} & line: 3 \quad col: 152 \\
\hline
\textbf{End} & line: 3 \quad col: 158 \\
\hline
\textbf{Message} & Possibility of prototype polluting function detected. By adding or modifying attributes of an object prototype, it is possible to create attributes that exist on every object, or replace critical attributes with malicious ones. This can be problematic if the software depends on existence or non-existence of certain attributes, or uses pre-defined attributes of object prototype (such as hasOwnProperty, toString or valueOf). Possible mitigations might be: freezing the object prototype, using an object without prototypes (via Object.create(null) ), blocking modifications of attributes that resolve to object prototype, using Map instead of object. \\
\hline
\end{tabular}
\end{table}
\vspace{0.7cm}
\FloatBarrier
\begin{table}[!h]
\centering
\renewcommand{\arraystretch}{1.3}
\begin{tabular}{|l|p{10cm}|}
\hline
\multicolumn{2}{|c|}{\textbf{Vulnerability 25}} \\
\hline
\multicolumn{2}{|l|}{\textbf{Path}: ScrapedFiles/index/main/conversion.js} \\
\hline
\textbf{Vulnerability Class} & ['Denial-of-Service (DoS)'] \\
\hline
\textbf{Start} & line: 34 \quad col: 97 \\
\hline
\textbf{End} & line: 34 \quad col: 134 \\
\hline
\textbf{Message} & RegExp() called with a `a` function argument, this might allow an attacker to cause a Regular Expression Denial-of-Service (ReDoS) within your application as RegExP blocks the main thread. For this reason, it is recommended to use hardcoded regexes instead. If your regex is run on user-controlled input, consider performing input validation or use a regex checking/sanitization library such as https://www.npmjs.com/package/recheck to verify that the regex does not appear vulnerable to ReDoS. \\
\hline
\end{tabular}
\end{table}
\vspace{0.7cm}
\FloatBarrier
\begin{table}[!h]
\centering
\renewcommand{\arraystretch}{1.3}
\begin{tabular}{|l|p{10cm}|}
\hline
\multicolumn{2}{|c|}{\textbf{Vulnerability 26}} \\
\hline
\multicolumn{2}{|l|}{\textbf{Path}: ScrapedFiles/index/main/conversion.js} \\
\hline
\textbf{Vulnerability Class} & ['Denial-of-Service (DoS)'] \\
\hline
\textbf{Start} & line: 39 \quad col: 1267 \\
\hline
\textbf{End} & line: 39 \quad col: 1331 \\
\hline
\textbf{Message} & RegExp() called with a `a` function argument, this might allow an attacker to cause a Regular Expression Denial-of-Service (ReDoS) within your application as RegExP blocks the main thread. For this reason, it is recommended to use hardcoded regexes instead. If your regex is run on user-controlled input, consider performing input validation or use a regex checking/sanitization library such as https://www.npmjs.com/package/recheck to verify that the regex does not appear vulnerable to ReDoS. \\
\hline
\end{tabular}
\end{table}
\vspace{0.7cm}
\FloatBarrier
\begin{table}[!h]
\centering
\renewcommand{\arraystretch}{1.3}
\begin{tabular}{|l|p{10cm}|}
\hline
\multicolumn{2}{|c|}{\textbf{Vulnerability 27}} \\
\hline
\multicolumn{2}{|l|}{\textbf{Path}: ScrapedFiles/index/main/conversion.js} \\
\hline
\textbf{Vulnerability Class} & ['Denial-of-Service (DoS)'] \\
\hline
\textbf{Start} & line: 79 \quad col: 343 \\
\hline
\textbf{End} & line: 79 \quad col: 372 \\
\hline
\textbf{Message} & RegExp() called with a `b` function argument, this might allow an attacker to cause a Regular Expression Denial-of-Service (ReDoS) within your application as RegExP blocks the main thread. For this reason, it is recommended to use hardcoded regexes instead. If your regex is run on user-controlled input, consider performing input validation or use a regex checking/sanitization library such as https://www.npmjs.com/package/recheck to verify that the regex does not appear vulnerable to ReDoS. \\
\hline
\end{tabular}
\end{table}
\vspace{0.7cm}
\FloatBarrier
\begin{table}[!h]
\centering
\renewcommand{\arraystretch}{1.3}
\begin{tabular}{|l|p{10cm}|}
\hline
\multicolumn{2}{|c|}{\textbf{Vulnerability 28}} \\
\hline
\multicolumn{2}{|l|}{\textbf{Path}: ScrapedFiles/index/main/conversion.js} \\
\hline
\textbf{Vulnerability Class} & ['Cross-Site-Scripting (XSS)'] \\
\hline
\textbf{Start} & line: 117 \quad col: 382 \\
\hline
\textbf{End} & line: 117 \quad col: 399 \\
\hline
\textbf{Message} & User controlled data in methods like `innerHTML`, `outerHTML` or `document.write` is an anti-pattern that can lead to XSS vulnerabilities \\
\hline
\end{tabular}
\end{table}
\vspace{0.7cm}
\FloatBarrier
\begin{table}[!h]
\centering
\renewcommand{\arraystretch}{1.3}
\begin{tabular}{|l|p{10cm}|}
\hline
\multicolumn{2}{|c|}{\textbf{Vulnerability 29}} \\
\hline
\multicolumn{2}{|l|}{\textbf{Path}: ScrapedFiles/index/main/index.html} \\
\hline
\textbf{Vulnerability Class} & ['Cryptographic Issues'] \\
\hline
\textbf{Start} & line: 58 \quad col: 141 \\
\hline
\textbf{End} & line: 58 \quad col: 1070 \\
\hline
\textbf{Message} & This tag is missing an 'integrity' subresource integrity attribute. The 'integrity' attribute allows for the browser to verify that externally hosted files (for example from a CDN) are delivered without unexpected manipulation. Without this attribute, if an attacker can modify the externally hosted resource, this could lead to XSS and other types of attacks. To prevent this, include the base64-encoded cryptographic hash of the resource (file) you're telling the browser to fetch in the 'integrity' attribute for all externally hosted files. \\
\hline
\end{tabular}
\end{table}
\vspace{0.7cm}
\FloatBarrier
\begin{table}[!h]
\centering
\renewcommand{\arraystretch}{1.3}
\begin{tabular}{|l|p{10cm}|}
\hline
\multicolumn{2}{|c|}{\textbf{Vulnerability 30}} \\
\hline
\multicolumn{2}{|l|}{\textbf{Path}: ScrapedFiles/index/main/index.html} \\
\hline
\textbf{Vulnerability Class} & ['Cryptographic Issues'] \\
\hline
\textbf{Start} & line: 58 \quad col: 1070 \\
\hline
\textbf{End} & line: 58 \quad col: 1149 \\
\hline
\textbf{Message} & This tag is missing an 'integrity' subresource integrity attribute. The 'integrity' attribute allows for the browser to verify that externally hosted files (for example from a CDN) are delivered without unexpected manipulation. Without this attribute, if an attacker can modify the externally hosted resource, this could lead to XSS and other types of attacks. To prevent this, include the base64-encoded cryptographic hash of the resource (file) you're telling the browser to fetch in the 'integrity' attribute for all externally hosted files. \\
\hline
\end{tabular}
\end{table}
\vspace{0.7cm}
\FloatBarrier
\begin{table}[!h]
\centering
\renewcommand{\arraystretch}{1.3}
\begin{tabular}{|l|p{10cm}|}
\hline
\multicolumn{2}{|c|}{\textbf{Vulnerability 31}} \\
\hline
\multicolumn{2}{|l|}{\textbf{Path}: ScrapedFiles/index/main/index.html} \\
\hline
\textbf{Vulnerability Class} & ['Cryptographic Issues'] \\
\hline
\textbf{Start} & line: 115 \quad col: 3 \\
\hline
\textbf{End} & line: 115 \quad col: 92 \\
\hline
\textbf{Message} & This tag is missing an 'integrity' subresource integrity attribute. The 'integrity' attribute allows for the browser to verify that externally hosted files (for example from a CDN) are delivered without unexpected manipulation. Without this attribute, if an attacker can modify the externally hosted resource, this could lead to XSS and other types of attacks. To prevent this, include the base64-encoded cryptographic hash of the resource (file) you're telling the browser to fetch in the 'integrity' attribute for all externally hosted files. \\
\hline
\end{tabular}
\end{table}
\vspace{0.7cm}
\FloatBarrier
\begin{table}[!h]
\centering
\renewcommand{\arraystretch}{1.3}
\begin{tabular}{|l|p{10cm}|}
\hline
\multicolumn{2}{|c|}{\textbf{Vulnerability 32}} \\
\hline
\multicolumn{2}{|l|}{\textbf{Path}: ScrapedFiles/index/main/index.html} \\
\hline
\textbf{Vulnerability Class} & ['Cryptographic Issues'] \\
\hline
\textbf{Start} & line: 120 \quad col: 3 \\
\hline
\textbf{End} & line: 120 \quad col: 99 \\
\hline
\textbf{Message} & This tag is missing an 'integrity' subresource integrity attribute. The 'integrity' attribute allows for the browser to verify that externally hosted files (for example from a CDN) are delivered without unexpected manipulation. Without this attribute, if an attacker can modify the externally hosted resource, this could lead to XSS and other types of attacks. To prevent this, include the base64-encoded cryptographic hash of the resource (file) you're telling the browser to fetch in the 'integrity' attribute for all externally hosted files. \\
\hline
\end{tabular}
\end{table}
\vspace{0.7cm}
\FloatBarrier
\begin{table}[!h]
\centering
\renewcommand{\arraystretch}{1.3}
\begin{tabular}{|l|p{10cm}|}
\hline
\multicolumn{2}{|c|}{\textbf{Vulnerability 33}} \\
\hline
\multicolumn{2}{|l|}{\textbf{Path}: ScrapedFiles/index/main/index.html} \\
\hline
\textbf{Vulnerability Class} & ['Cryptographic Issues'] \\
\hline
\textbf{Start} & line: 121 \quad col: 3 \\
\hline
\textbf{End} & line: 121 \quad col: 123 \\
\hline
\textbf{Message} & This tag is missing an 'integrity' subresource integrity attribute. The 'integrity' attribute allows for the browser to verify that externally hosted files (for example from a CDN) are delivered without unexpected manipulation. Without this attribute, if an attacker can modify the externally hosted resource, this could lead to XSS and other types of attacks. To prevent this, include the base64-encoded cryptographic hash of the resource (file) you're telling the browser to fetch in the 'integrity' attribute for all externally hosted files. \\
\hline
\end{tabular}
\end{table}
\vspace{0.7cm}
\FloatBarrier
\begin{table}[!h]
\centering
\renewcommand{\arraystretch}{1.3}
\begin{tabular}{|l|p{10cm}|}
\hline
\multicolumn{2}{|c|}{\textbf{Vulnerability 34}} \\
\hline
\multicolumn{2}{|l|}{\textbf{Path}: ScrapedFiles/index/main/index.html} \\
\hline
\textbf{Vulnerability Class} & ['Cryptographic Issues'] \\
\hline
\textbf{Start} & line: 189 \quad col: 5 \\
\hline
\textbf{End} & line: 190 \quad col: 14 \\
\hline
\textbf{Message} & This tag is missing an 'integrity' subresource integrity attribute. The 'integrity' attribute allows for the browser to verify that externally hosted files (for example from a CDN) are delivered without unexpected manipulation. Without this attribute, if an attacker can modify the externally hosted resource, this could lead to XSS and other types of attacks. To prevent this, include the base64-encoded cryptographic hash of the resource (file) you're telling the browser to fetch in the 'integrity' attribute for all externally hosted files. \\
\hline
\end{tabular}
\end{table}
\vspace{0.7cm}
\FloatBarrier
\begin{table}[!h]
\centering
\renewcommand{\arraystretch}{1.3}
\begin{tabular}{|l|p{10cm}|}
\hline
\multicolumn{2}{|c|}{\textbf{Vulnerability 35}} \\
\hline
\multicolumn{2}{|l|}{\textbf{Path}: ScrapedFiles/index/main/index.html} \\
\hline
\textbf{Vulnerability Class} & ['Cryptographic Issues'] \\
\hline
\textbf{Start} & line: 190 \quad col: 14 \\
\hline
\textbf{End} & line: 190 \quad col: 645 \\
\hline
\textbf{Message} & This tag is missing an 'integrity' subresource integrity attribute. The 'integrity' attribute allows for the browser to verify that externally hosted files (for example from a CDN) are delivered without unexpected manipulation. Without this attribute, if an attacker can modify the externally hosted resource, this could lead to XSS and other types of attacks. To prevent this, include the base64-encoded cryptographic hash of the resource (file) you're telling the browser to fetch in the 'integrity' attribute for all externally hosted files. \\
\hline
\end{tabular}
\end{table}
\vspace{0.7cm}
\FloatBarrier
\begin{table}[!h]
\centering
\renewcommand{\arraystretch}{1.3}
\begin{tabular}{|l|p{10cm}|}
\hline
\multicolumn{2}{|c|}{\textbf{Vulnerability 36}} \\
\hline
\multicolumn{2}{|l|}{\textbf{Path}: ScrapedFiles/index/main/index.html} \\
\hline
\textbf{Vulnerability Class} & ['Cryptographic Issues'] \\
\hline
\textbf{Start} & line: 207 \quad col: 1 \\
\hline
\textbf{End} & line: 207 \quad col: 642 \\
\hline
\textbf{Message} & This tag is missing an 'integrity' subresource integrity attribute. The 'integrity' attribute allows for the browser to verify that externally hosted files (for example from a CDN) are delivered without unexpected manipulation. Without this attribute, if an attacker can modify the externally hosted resource, this could lead to XSS and other types of attacks. To prevent this, include the base64-encoded cryptographic hash of the resource (file) you're telling the browser to fetch in the 'integrity' attribute for all externally hosted files. \\
\hline
\end{tabular}
\end{table}
\vspace{0.7cm}
\FloatBarrier
\begin{table}[!h]
\centering
\renewcommand{\arraystretch}{1.3}
\begin{tabular}{|l|p{10cm}|}
\hline
\multicolumn{2}{|c|}{\textbf{Vulnerability 37}} \\
\hline
\multicolumn{2}{|l|}{\textbf{Path}: ScrapedFiles/lessons/main/analytics.js} \\
\hline
\textbf{Vulnerability Class} & ['Denial-of-Service (DoS)'] \\
\hline
\textbf{Start} & line: 10 \quad col: 486 \\
\hline
\textbf{End} & line: 10 \quad col: 529 \\
\hline
\textbf{Message} & RegExp() called with a `a` function argument, this might allow an attacker to cause a Regular Expression Denial-of-Service (ReDoS) within your application as RegExP blocks the main thread. For this reason, it is recommended to use hardcoded regexes instead. If your regex is run on user-controlled input, consider performing input validation or use a regex checking/sanitization library such as https://www.npmjs.com/package/recheck to verify that the regex does not appear vulnerable to ReDoS. \\
\hline
\end{tabular}
\end{table}
\vspace{0.7cm}
\FloatBarrier
\begin{table}[!h]
\centering
\renewcommand{\arraystretch}{1.3}
\begin{tabular}{|l|p{10cm}|}
\hline
\multicolumn{2}{|c|}{\textbf{Vulnerability 38}} \\
\hline
\multicolumn{2}{|l|}{\textbf{Path}: ScrapedFiles/lessons/main/analytics.js} \\
\hline
\textbf{Vulnerability Class} & ['Denial-of-Service (DoS)'] \\
\hline
\textbf{Start} & line: 27 \quad col: 150 \\
\hline
\textbf{End} & line: 27 \quad col: 189 \\
\hline
\textbf{Message} & RegExp() called with a `a` function argument, this might allow an attacker to cause a Regular Expression Denial-of-Service (ReDoS) within your application as RegExP blocks the main thread. For this reason, it is recommended to use hardcoded regexes instead. If your regex is run on user-controlled input, consider performing input validation or use a regex checking/sanitization library such as https://www.npmjs.com/package/recheck to verify that the regex does not appear vulnerable to ReDoS. \\
\hline
\end{tabular}
\end{table}
\vspace{0.7cm}
\FloatBarrier
\begin{table}[!h]
\centering
\renewcommand{\arraystretch}{1.3}
\begin{tabular}{|l|p{10cm}|}
\hline
\multicolumn{2}{|c|}{\textbf{Vulnerability 39}} \\
\hline
\multicolumn{2}{|l|}{\textbf{Path}: ScrapedFiles/lessons/main/analytics.js} \\
\hline
\textbf{Vulnerability Class} & ['Denial-of-Service (DoS)'] \\
\hline
\textbf{Start} & line: 28 \quad col: 304 \\
\hline
\textbf{End} & line: 28 \quad col: 368 \\
\hline
\textbf{Message} & RegExp() called with a `a` function argument, this might allow an attacker to cause a Regular Expression Denial-of-Service (ReDoS) within your application as RegExP blocks the main thread. For this reason, it is recommended to use hardcoded regexes instead. If your regex is run on user-controlled input, consider performing input validation or use a regex checking/sanitization library such as https://www.npmjs.com/package/recheck to verify that the regex does not appear vulnerable to ReDoS. \\
\hline
\end{tabular}
\end{table}
\vspace{0.7cm}
\FloatBarrier
\begin{table}[!h]
\centering
\renewcommand{\arraystretch}{1.3}
\begin{tabular}{|l|p{10cm}|}
\hline
\multicolumn{2}{|c|}{\textbf{Vulnerability 40}} \\
\hline
\multicolumn{2}{|l|}{\textbf{Path}: ScrapedFiles/lessons/main/analytics.js} \\
\hline
\textbf{Vulnerability Class} & ['Improper Encoding'] \\
\hline
\textbf{Start} & line: 37 \quad col: 130 \\
\hline
\textbf{End} & line: 37 \quad col: 184 \\
\hline
\textbf{Message} & `"https://www.google.\%/ads/ga-audiences".replace` method will only replace the first occurrence when used with a string argument ("\%"). If this method is used for escaping of dangerous data then there is a possibility for a bypass. Try to use sanitization library instead or use a Regex with a global flag. \\
\hline
\end{tabular}
\end{table}
\vspace{0.7cm}
\FloatBarrier
\begin{table}[!h]
\centering
\renewcommand{\arraystretch}{1.3}
\begin{tabular}{|l|p{10cm}|}
\hline
\multicolumn{2}{|c|}{\textbf{Vulnerability 41}} \\
\hline
\multicolumn{2}{|l|}{\textbf{Path}: ScrapedFiles/lessons/main/analytics.js} \\
\hline
\textbf{Vulnerability Class} & ['Denial-of-Service (DoS)'] \\
\hline
\textbf{Start} & line: 48 \quad col: 629 \\
\hline
\textbf{End} & line: 48 \quad col: 650 \\
\hline
\textbf{Message} & RegExp() called with a `a` function argument, this might allow an attacker to cause a Regular Expression Denial-of-Service (ReDoS) within your application as RegExP blocks the main thread. For this reason, it is recommended to use hardcoded regexes instead. If your regex is run on user-controlled input, consider performing input validation or use a regex checking/sanitization library such as https://www.npmjs.com/package/recheck to verify that the regex does not appear vulnerable to ReDoS. \\
\hline
\end{tabular}
\end{table}
\vspace{0.7cm}
\FloatBarrier
\begin{table}[!h]
\centering
\renewcommand{\arraystretch}{1.3}
\begin{tabular}{|l|p{10cm}|}
\hline
\multicolumn{2}{|c|}{\textbf{Vulnerability 42}} \\
\hline
\multicolumn{2}{|l|}{\textbf{Path}: ScrapedFiles/lessons/main/conversion.js} \\
\hline
\textbf{Vulnerability Class} & ['Mass Assignment'] \\
\hline
\textbf{Start} & line: 3 \quad col: 152 \\
\hline
\textbf{End} & line: 3 \quad col: 158 \\
\hline
\textbf{Message} & Possibility of prototype polluting function detected. By adding or modifying attributes of an object prototype, it is possible to create attributes that exist on every object, or replace critical attributes with malicious ones. This can be problematic if the software depends on existence or non-existence of certain attributes, or uses pre-defined attributes of object prototype (such as hasOwnProperty, toString or valueOf). Possible mitigations might be: freezing the object prototype, using an object without prototypes (via Object.create(null) ), blocking modifications of attributes that resolve to object prototype, using Map instead of object. \\
\hline
\end{tabular}
\end{table}
\vspace{0.7cm}
\FloatBarrier
\begin{table}[!h]
\centering
\renewcommand{\arraystretch}{1.3}
\begin{tabular}{|l|p{10cm}|}
\hline
\multicolumn{2}{|c|}{\textbf{Vulnerability 43}} \\
\hline
\multicolumn{2}{|l|}{\textbf{Path}: ScrapedFiles/lessons/main/conversion.js} \\
\hline
\textbf{Vulnerability Class} & ['Denial-of-Service (DoS)'] \\
\hline
\textbf{Start} & line: 34 \quad col: 97 \\
\hline
\textbf{End} & line: 34 \quad col: 134 \\
\hline
\textbf{Message} & RegExp() called with a `a` function argument, this might allow an attacker to cause a Regular Expression Denial-of-Service (ReDoS) within your application as RegExP blocks the main thread. For this reason, it is recommended to use hardcoded regexes instead. If your regex is run on user-controlled input, consider performing input validation or use a regex checking/sanitization library such as https://www.npmjs.com/package/recheck to verify that the regex does not appear vulnerable to ReDoS. \\
\hline
\end{tabular}
\end{table}
\vspace{0.7cm}
\FloatBarrier
\begin{table}[!h]
\centering
\renewcommand{\arraystretch}{1.3}
\begin{tabular}{|l|p{10cm}|}
\hline
\multicolumn{2}{|c|}{\textbf{Vulnerability 44}} \\
\hline
\multicolumn{2}{|l|}{\textbf{Path}: ScrapedFiles/lessons/main/conversion.js} \\
\hline
\textbf{Vulnerability Class} & ['Denial-of-Service (DoS)'] \\
\hline
\textbf{Start} & line: 39 \quad col: 1267 \\
\hline
\textbf{End} & line: 39 \quad col: 1331 \\
\hline
\textbf{Message} & RegExp() called with a `a` function argument, this might allow an attacker to cause a Regular Expression Denial-of-Service (ReDoS) within your application as RegExP blocks the main thread. For this reason, it is recommended to use hardcoded regexes instead. If your regex is run on user-controlled input, consider performing input validation or use a regex checking/sanitization library such as https://www.npmjs.com/package/recheck to verify that the regex does not appear vulnerable to ReDoS. \\
\hline
\end{tabular}
\end{table}
\vspace{0.7cm}
\FloatBarrier
\begin{table}[!h]
\centering
\renewcommand{\arraystretch}{1.3}
\begin{tabular}{|l|p{10cm}|}
\hline
\multicolumn{2}{|c|}{\textbf{Vulnerability 45}} \\
\hline
\multicolumn{2}{|l|}{\textbf{Path}: ScrapedFiles/lessons/main/conversion.js} \\
\hline
\textbf{Vulnerability Class} & ['Denial-of-Service (DoS)'] \\
\hline
\textbf{Start} & line: 79 \quad col: 343 \\
\hline
\textbf{End} & line: 79 \quad col: 372 \\
\hline
\textbf{Message} & RegExp() called with a `b` function argument, this might allow an attacker to cause a Regular Expression Denial-of-Service (ReDoS) within your application as RegExP blocks the main thread. For this reason, it is recommended to use hardcoded regexes instead. If your regex is run on user-controlled input, consider performing input validation or use a regex checking/sanitization library such as https://www.npmjs.com/package/recheck to verify that the regex does not appear vulnerable to ReDoS. \\
\hline
\end{tabular}
\end{table}
\vspace{0.7cm}
\FloatBarrier
\begin{table}[!h]
\centering
\renewcommand{\arraystretch}{1.3}
\begin{tabular}{|l|p{10cm}|}
\hline
\multicolumn{2}{|c|}{\textbf{Vulnerability 46}} \\
\hline
\multicolumn{2}{|l|}{\textbf{Path}: ScrapedFiles/lessons/main/conversion.js} \\
\hline
\textbf{Vulnerability Class} & ['Cross-Site-Scripting (XSS)'] \\
\hline
\textbf{Start} & line: 117 \quad col: 382 \\
\hline
\textbf{End} & line: 117 \quad col: 399 \\
\hline
\textbf{Message} & User controlled data in methods like `innerHTML`, `outerHTML` or `document.write` is an anti-pattern that can lead to XSS vulnerabilities \\
\hline
\end{tabular}
\end{table}
\vspace{0.7cm}
\FloatBarrier
\begin{table}[!h]
\centering
\renewcommand{\arraystretch}{1.3}
\begin{tabular}{|l|p{10cm}|}
\hline
\multicolumn{2}{|c|}{\textbf{Vulnerability 47}} \\
\hline
\multicolumn{2}{|l|}{\textbf{Path}: ScrapedFiles/lessons/main/gumroad.js} \\
\hline
\textbf{Vulnerability Class} & ['Cross-Site-Scripting (XSS)'] \\
\hline
\textbf{Start} & line: 6 \quad col: 5 \\
\hline
\textbf{End} & line: 6 \quad col: 125 \\
\hline
\textbf{Message} & User controlled data in methods like `innerHTML`, `outerHTML` or `document.write` is an anti-pattern that can lead to XSS vulnerabilities \\
\hline
\end{tabular}
\end{table}
\vspace{0.7cm}
\FloatBarrier
\begin{table}[!h]
\centering
\renewcommand{\arraystretch}{1.3}
\begin{tabular}{|l|p{10cm}|}
\hline
\multicolumn{2}{|c|}{\textbf{Vulnerability 48}} \\
\hline
\multicolumn{2}{|l|}{\textbf{Path}: ScrapedFiles/lessons/main/lessons.html} \\
\hline
\textbf{Vulnerability Class} & ['Cryptographic Issues'] \\
\hline
\textbf{Start} & line: 16 \quad col: 5 \\
\hline
\textbf{End} & line: 16 \quad col: 74 \\
\hline
\textbf{Message} & This tag is missing an 'integrity' subresource integrity attribute. The 'integrity' attribute allows for the browser to verify that externally hosted files (for example from a CDN) are delivered without unexpected manipulation. Without this attribute, if an attacker can modify the externally hosted resource, this could lead to XSS and other types of attacks. To prevent this, include the base64-encoded cryptographic hash of the resource (file) you're telling the browser to fetch in the 'integrity' attribute for all externally hosted files. \\
\hline
\end{tabular}
\end{table}
\vspace{0.7cm}
\FloatBarrier
\begin{table}[!h]
\centering
\renewcommand{\arraystretch}{1.3}
\begin{tabular}{|l|p{10cm}|}
\hline
\multicolumn{2}{|c|}{\textbf{Vulnerability 49}} \\
\hline
\multicolumn{2}{|l|}{\textbf{Path}: ScrapedFiles/lessons/main/lessons.html} \\
\hline
\textbf{Vulnerability Class} & ['Cryptographic Issues'] \\
\hline
\textbf{Start} & line: 19 \quad col: 137 \\
\hline
\textbf{End} & line: 19 \quad col: 1066 \\
\hline
\textbf{Message} & This tag is missing an 'integrity' subresource integrity attribute. The 'integrity' attribute allows for the browser to verify that externally hosted files (for example from a CDN) are delivered without unexpected manipulation. Without this attribute, if an attacker can modify the externally hosted resource, this could lead to XSS and other types of attacks. To prevent this, include the base64-encoded cryptographic hash of the resource (file) you're telling the browser to fetch in the 'integrity' attribute for all externally hosted files. \\
\hline
\end{tabular}
\end{table}
\vspace{0.7cm}
\FloatBarrier
\begin{table}[!h]
\centering
\renewcommand{\arraystretch}{1.3}
\begin{tabular}{|l|p{10cm}|}
\hline
\multicolumn{2}{|c|}{\textbf{Vulnerability 50}} \\
\hline
\multicolumn{2}{|l|}{\textbf{Path}: ScrapedFiles/lessons/main/lessons.html} \\
\hline
\textbf{Vulnerability Class} & ['Cryptographic Issues'] \\
\hline
\textbf{Start} & line: 19 \quad col: 1066 \\
\hline
\textbf{End} & line: 19 \quad col: 1145 \\
\hline
\textbf{Message} & This tag is missing an 'integrity' subresource integrity attribute. The 'integrity' attribute allows for the browser to verify that externally hosted files (for example from a CDN) are delivered without unexpected manipulation. Without this attribute, if an attacker can modify the externally hosted resource, this could lead to XSS and other types of attacks. To prevent this, include the base64-encoded cryptographic hash of the resource (file) you're telling the browser to fetch in the 'integrity' attribute for all externally hosted files. \\
\hline
\end{tabular}
\end{table}
\vspace{0.7cm}
\FloatBarrier
\begin{table}[!h]
\centering
\renewcommand{\arraystretch}{1.3}
\begin{tabular}{|l|p{10cm}|}
\hline
\multicolumn{2}{|c|}{\textbf{Vulnerability 51}} \\
\hline
\multicolumn{2}{|l|}{\textbf{Path}: ScrapedFiles/lessons/main/lessons.html} \\
\hline
\textbf{Vulnerability Class} & ['Cryptographic Issues'] \\
\hline
\textbf{Start} & line: 19 \quad col: 1223 \\
\hline
\textbf{End} & line: 19 \quad col: 1338 \\
\hline
\textbf{Message} & This tag is missing an 'integrity' subresource integrity attribute. The 'integrity' attribute allows for the browser to verify that externally hosted files (for example from a CDN) are delivered without unexpected manipulation. Without this attribute, if an attacker can modify the externally hosted resource, this could lead to XSS and other types of attacks. To prevent this, include the base64-encoded cryptographic hash of the resource (file) you're telling the browser to fetch in the 'integrity' attribute for all externally hosted files. \\
\hline
\end{tabular}
\end{table}
\vspace{0.7cm}
\FloatBarrier
\begin{table}[!h]
\centering
\renewcommand{\arraystretch}{1.3}
\begin{tabular}{|l|p{10cm}|}
\hline
\multicolumn{2}{|c|}{\textbf{Vulnerability 52}} \\
\hline
\multicolumn{2}{|l|}{\textbf{Path}: ScrapedFiles/lessons/main/lessons.html} \\
\hline
\textbf{Vulnerability Class} & ['Cryptographic Issues'] \\
\hline
\textbf{Start} & line: 324 \quad col: 9 \\
\hline
\textbf{End} & line: 324 \quad col: 116 \\
\hline
\textbf{Message} & This tag is missing an 'integrity' subresource integrity attribute. The 'integrity' attribute allows for the browser to verify that externally hosted files (for example from a CDN) are delivered without unexpected manipulation. Without this attribute, if an attacker can modify the externally hosted resource, this could lead to XSS and other types of attacks. To prevent this, include the base64-encoded cryptographic hash of the resource (file) you're telling the browser to fetch in the 'integrity' attribute for all externally hosted files. \\
\hline
\end{tabular}
\end{table}
\vspace{0.7cm}
\FloatBarrier
\begin{table}[!h]
\centering
\renewcommand{\arraystretch}{1.3}
\begin{tabular}{|l|p{10cm}|}
\hline
\multicolumn{2}{|c|}{\textbf{Vulnerability 53}} \\
\hline
\multicolumn{2}{|l|}{\textbf{Path}: ScrapedFiles/lessons/main/lessons.html} \\
\hline
\textbf{Vulnerability Class} & ['Cryptographic Issues'] \\
\hline
\textbf{Start} & line: 324 \quad col: 116 \\
\hline
\textbf{End} & line: 324 \quad col: 204 \\
\hline
\textbf{Message} & This tag is missing an 'integrity' subresource integrity attribute. The 'integrity' attribute allows for the browser to verify that externally hosted files (for example from a CDN) are delivered without unexpected manipulation. Without this attribute, if an attacker can modify the externally hosted resource, this could lead to XSS and other types of attacks. To prevent this, include the base64-encoded cryptographic hash of the resource (file) you're telling the browser to fetch in the 'integrity' attribute for all externally hosted files. \\
\hline
\end{tabular}
\end{table}
\vspace{0.7cm}
\FloatBarrier
\begin{table}[!h]
\centering
\renewcommand{\arraystretch}{1.3}
\begin{tabular}{|l|p{10cm}|}
\hline
\multicolumn{2}{|c|}{\textbf{Vulnerability 54}} \\
\hline
\multicolumn{2}{|l|}{\textbf{Path}: ScrapedFiles/lessons/main/lessons.html} \\
\hline
\textbf{Vulnerability Class} & ['Cryptographic Issues'] \\
\hline
\textbf{Start} & line: 324 \quad col: 204 \\
\hline
\textbf{End} & line: 324 \quad col: 873 \\
\hline
\textbf{Message} & This tag is missing an 'integrity' subresource integrity attribute. The 'integrity' attribute allows for the browser to verify that externally hosted files (for example from a CDN) are delivered without unexpected manipulation. Without this attribute, if an attacker can modify the externally hosted resource, this could lead to XSS and other types of attacks. To prevent this, include the base64-encoded cryptographic hash of the resource (file) you're telling the browser to fetch in the 'integrity' attribute for all externally hosted files. \\
\hline
\end{tabular}
\end{table}
\vspace{0.7cm}
\FloatBarrier
\begin{table}[!h]
\centering
\renewcommand{\arraystretch}{1.3}
\begin{tabular}{|l|p{10cm}|}
\hline
\multicolumn{2}{|c|}{\textbf{Vulnerability 55}} \\
\hline
\multicolumn{2}{|l|}{\textbf{Path}: ScrapedFiles/lessons/main/lessons.html} \\
\hline
\textbf{Vulnerability Class} & ['Cryptographic Issues'] \\
\hline
\textbf{Start} & line: 324 \quad col: 873 \\
\hline
\textbf{End} & line: 324 \quad col: 1552 \\
\hline
\textbf{Message} & This tag is missing an 'integrity' subresource integrity attribute. The 'integrity' attribute allows for the browser to verify that externally hosted files (for example from a CDN) are delivered without unexpected manipulation. Without this attribute, if an attacker can modify the externally hosted resource, this could lead to XSS and other types of attacks. To prevent this, include the base64-encoded cryptographic hash of the resource (file) you're telling the browser to fetch in the 'integrity' attribute for all externally hosted files. \\
\hline
\end{tabular}
\end{table}
\vspace{0.7cm}
\FloatBarrier
\begin{table}[!h]
\centering
\renewcommand{\arraystretch}{1.3}
\begin{tabular}{|l|p{10cm}|}
\hline
\multicolumn{2}{|c|}{\textbf{Vulnerability 56}} \\
\hline
\multicolumn{2}{|l|}{\textbf{Path}: ScrapedFiles/lessons/main/lessons.html} \\
\hline
\textbf{Vulnerability Class} & ['Cryptographic Issues'] \\
\hline
\textbf{Start} & line: 715 \quad col: 3 \\
\hline
\textbf{End} & line: 715 \quad col: 92 \\
\hline
\textbf{Message} & This tag is missing an 'integrity' subresource integrity attribute. The 'integrity' attribute allows for the browser to verify that externally hosted files (for example from a CDN) are delivered without unexpected manipulation. Without this attribute, if an attacker can modify the externally hosted resource, this could lead to XSS and other types of attacks. To prevent this, include the base64-encoded cryptographic hash of the resource (file) you're telling the browser to fetch in the 'integrity' attribute for all externally hosted files. \\
\hline
\end{tabular}
\end{table}
\vspace{0.7cm}
\FloatBarrier
\begin{table}[!h]
\centering
\renewcommand{\arraystretch}{1.3}
\begin{tabular}{|l|p{10cm}|}
\hline
\multicolumn{2}{|c|}{\textbf{Vulnerability 57}} \\
\hline
\multicolumn{2}{|l|}{\textbf{Path}: ScrapedFiles/lessons/main/lessons.html} \\
\hline
\textbf{Vulnerability Class} & ['Cryptographic Issues'] \\
\hline
\textbf{Start} & line: 720 \quad col: 5 \\
\hline
\textbf{End} & line: 720 \quad col: 137 \\
\hline
\textbf{Message} & This tag is missing an 'integrity' subresource integrity attribute. The 'integrity' attribute allows for the browser to verify that externally hosted files (for example from a CDN) are delivered without unexpected manipulation. Without this attribute, if an attacker can modify the externally hosted resource, this could lead to XSS and other types of attacks. To prevent this, include the base64-encoded cryptographic hash of the resource (file) you're telling the browser to fetch in the 'integrity' attribute for all externally hosted files. \\
\hline
\end{tabular}
\end{table}
\vspace{0.7cm}
\FloatBarrier
\begin{table}[!h]
\centering
\renewcommand{\arraystretch}{1.3}
\begin{tabular}{|l|p{10cm}|}
\hline
\multicolumn{2}{|c|}{\textbf{Vulnerability 58}} \\
\hline
\multicolumn{2}{|l|}{\textbf{Path}: ScrapedFiles/lessons/main/lessons.html} \\
\hline
\textbf{Vulnerability Class} & ['Cryptographic Issues'] \\
\hline
\textbf{Start} & line: 745 \quad col: 3 \\
\hline
\textbf{End} & line: 745 \quad col: 99 \\
\hline
\textbf{Message} & This tag is missing an 'integrity' subresource integrity attribute. The 'integrity' attribute allows for the browser to verify that externally hosted files (for example from a CDN) are delivered without unexpected manipulation. Without this attribute, if an attacker can modify the externally hosted resource, this could lead to XSS and other types of attacks. To prevent this, include the base64-encoded cryptographic hash of the resource (file) you're telling the browser to fetch in the 'integrity' attribute for all externally hosted files. \\
\hline
\end{tabular}
\end{table}
\vspace{0.7cm}
\FloatBarrier
\begin{table}[!h]
\centering
\renewcommand{\arraystretch}{1.3}
\begin{tabular}{|l|p{10cm}|}
\hline
\multicolumn{2}{|c|}{\textbf{Vulnerability 59}} \\
\hline
\multicolumn{2}{|l|}{\textbf{Path}: ScrapedFiles/lessons/main/lessons.html} \\
\hline
\textbf{Vulnerability Class} & ['Cryptographic Issues'] \\
\hline
\textbf{Start} & line: 746 \quad col: 3 \\
\hline
\textbf{End} & line: 746 \quad col: 123 \\
\hline
\textbf{Message} & This tag is missing an 'integrity' subresource integrity attribute. The 'integrity' attribute allows for the browser to verify that externally hosted files (for example from a CDN) are delivered without unexpected manipulation. Without this attribute, if an attacker can modify the externally hosted resource, this could lead to XSS and other types of attacks. To prevent this, include the base64-encoded cryptographic hash of the resource (file) you're telling the browser to fetch in the 'integrity' attribute for all externally hosted files. \\
\hline
\end{tabular}
\end{table}
\vspace{0.7cm}
\FloatBarrier
\begin{table}[!h]
\centering
\renewcommand{\arraystretch}{1.3}
\begin{tabular}{|l|p{10cm}|}
\hline
\multicolumn{2}{|c|}{\textbf{Vulnerability 60}} \\
\hline
\multicolumn{2}{|l|}{\textbf{Path}: ScrapedFiles/lessons/main/lessons.html} \\
\hline
\textbf{Vulnerability Class} & ['Cryptographic Issues'] \\
\hline
\textbf{Start} & line: 814 \quad col: 5 \\
\hline
\textbf{End} & line: 815 \quad col: 14 \\
\hline
\textbf{Message} & This tag is missing an 'integrity' subresource integrity attribute. The 'integrity' attribute allows for the browser to verify that externally hosted files (for example from a CDN) are delivered without unexpected manipulation. Without this attribute, if an attacker can modify the externally hosted resource, this could lead to XSS and other types of attacks. To prevent this, include the base64-encoded cryptographic hash of the resource (file) you're telling the browser to fetch in the 'integrity' attribute for all externally hosted files. \\
\hline
\end{tabular}
\end{table}
\vspace{0.7cm}
\FloatBarrier
\begin{table}[!h]
\centering
\renewcommand{\arraystretch}{1.3}
\begin{tabular}{|l|p{10cm}|}
\hline
\multicolumn{2}{|c|}{\textbf{Vulnerability 61}} \\
\hline
\multicolumn{2}{|l|}{\textbf{Path}: ScrapedFiles/lessons/main/overlay-f8f9015a9aabefa09736.js} \\
\hline
\textbf{Vulnerability Class} & ['Improper Authentication'] \\
\hline
\textbf{Start} & line: 1 \quad col: 5004 \\
\hline
\textbf{End} & line: 1 \quad col: 5436 \\
\hline
\textbf{Message} & No validation of origin is done by the addEventListener API. It may be possible to exploit this flaw to perform Cross Origin attacks such as Cross-Site Scripting(XSS). \\
\hline
\end{tabular}
\end{table}
\vspace{0.7cm}
\FloatBarrier
\begin{table}[!h]
\centering
\renewcommand{\arraystretch}{1.3}
\begin{tabular}{|l|p{10cm}|}
\hline
\multicolumn{2}{|c|}{\textbf{Vulnerability 62}} \\
\hline
\multicolumn{2}{|l|}{\textbf{Path}: ScrapedFiles/login/main/analytics.js} \\
\hline
\textbf{Vulnerability Class} & ['Denial-of-Service (DoS)'] \\
\hline
\textbf{Start} & line: 10 \quad col: 486 \\
\hline
\textbf{End} & line: 10 \quad col: 529 \\
\hline
\textbf{Message} & RegExp() called with a `a` function argument, this might allow an attacker to cause a Regular Expression Denial-of-Service (ReDoS) within your application as RegExP blocks the main thread. For this reason, it is recommended to use hardcoded regexes instead. If your regex is run on user-controlled input, consider performing input validation or use a regex checking/sanitization library such as https://www.npmjs.com/package/recheck to verify that the regex does not appear vulnerable to ReDoS. \\
\hline
\end{tabular}
\end{table}
\vspace{0.7cm}
\FloatBarrier
\begin{table}[!h]
\centering
\renewcommand{\arraystretch}{1.3}
\begin{tabular}{|l|p{10cm}|}
\hline
\multicolumn{2}{|c|}{\textbf{Vulnerability 63}} \\
\hline
\multicolumn{2}{|l|}{\textbf{Path}: ScrapedFiles/login/main/analytics.js} \\
\hline
\textbf{Vulnerability Class} & ['Denial-of-Service (DoS)'] \\
\hline
\textbf{Start} & line: 27 \quad col: 150 \\
\hline
\textbf{End} & line: 27 \quad col: 189 \\
\hline
\textbf{Message} & RegExp() called with a `a` function argument, this might allow an attacker to cause a Regular Expression Denial-of-Service (ReDoS) within your application as RegExP blocks the main thread. For this reason, it is recommended to use hardcoded regexes instead. If your regex is run on user-controlled input, consider performing input validation or use a regex checking/sanitization library such as https://www.npmjs.com/package/recheck to verify that the regex does not appear vulnerable to ReDoS. \\
\hline
\end{tabular}
\end{table}
\vspace{0.7cm}
\FloatBarrier
\begin{table}[!h]
\centering
\renewcommand{\arraystretch}{1.3}
\begin{tabular}{|l|p{10cm}|}
\hline
\multicolumn{2}{|c|}{\textbf{Vulnerability 64}} \\
\hline
\multicolumn{2}{|l|}{\textbf{Path}: ScrapedFiles/login/main/analytics.js} \\
\hline
\textbf{Vulnerability Class} & ['Denial-of-Service (DoS)'] \\
\hline
\textbf{Start} & line: 28 \quad col: 304 \\
\hline
\textbf{End} & line: 28 \quad col: 368 \\
\hline
\textbf{Message} & RegExp() called with a `a` function argument, this might allow an attacker to cause a Regular Expression Denial-of-Service (ReDoS) within your application as RegExP blocks the main thread. For this reason, it is recommended to use hardcoded regexes instead. If your regex is run on user-controlled input, consider performing input validation or use a regex checking/sanitization library such as https://www.npmjs.com/package/recheck to verify that the regex does not appear vulnerable to ReDoS. \\
\hline
\end{tabular}
\end{table}
\vspace{0.7cm}
\FloatBarrier
\begin{table}[!h]
\centering
\renewcommand{\arraystretch}{1.3}
\begin{tabular}{|l|p{10cm}|}
\hline
\multicolumn{2}{|c|}{\textbf{Vulnerability 65}} \\
\hline
\multicolumn{2}{|l|}{\textbf{Path}: ScrapedFiles/login/main/analytics.js} \\
\hline
\textbf{Vulnerability Class} & ['Improper Encoding'] \\
\hline
\textbf{Start} & line: 37 \quad col: 130 \\
\hline
\textbf{End} & line: 37 \quad col: 184 \\
\hline
\textbf{Message} & `"https://www.google.\%/ads/ga-audiences".replace` method will only replace the first occurrence when used with a string argument ("\%"). If this method is used for escaping of dangerous data then there is a possibility for a bypass. Try to use sanitization library instead or use a Regex with a global flag. \\
\hline
\end{tabular}
\end{table}
\vspace{0.7cm}
\FloatBarrier
\begin{table}[!h]
\centering
\renewcommand{\arraystretch}{1.3}
\begin{tabular}{|l|p{10cm}|}
\hline
\multicolumn{2}{|c|}{\textbf{Vulnerability 66}} \\
\hline
\multicolumn{2}{|l|}{\textbf{Path}: ScrapedFiles/login/main/analytics.js} \\
\hline
\textbf{Vulnerability Class} & ['Denial-of-Service (DoS)'] \\
\hline
\textbf{Start} & line: 48 \quad col: 629 \\
\hline
\textbf{End} & line: 48 \quad col: 650 \\
\hline
\textbf{Message} & RegExp() called with a `a` function argument, this might allow an attacker to cause a Regular Expression Denial-of-Service (ReDoS) within your application as RegExP blocks the main thread. For this reason, it is recommended to use hardcoded regexes instead. If your regex is run on user-controlled input, consider performing input validation or use a regex checking/sanitization library such as https://www.npmjs.com/package/recheck to verify that the regex does not appear vulnerable to ReDoS. \\
\hline
\end{tabular}
\end{table}
\vspace{0.7cm}
\FloatBarrier
\begin{table}[!h]
\centering
\renewcommand{\arraystretch}{1.3}
\begin{tabular}{|l|p{10cm}|}
\hline
\multicolumn{2}{|c|}{\textbf{Vulnerability 67}} \\
\hline
\multicolumn{2}{|l|}{\textbf{Path}: ScrapedFiles/login/main/conversion.js} \\
\hline
\textbf{Vulnerability Class} & ['Mass Assignment'] \\
\hline
\textbf{Start} & line: 3 \quad col: 152 \\
\hline
\textbf{End} & line: 3 \quad col: 158 \\
\hline
\textbf{Message} & Possibility of prototype polluting function detected. By adding or modifying attributes of an object prototype, it is possible to create attributes that exist on every object, or replace critical attributes with malicious ones. This can be problematic if the software depends on existence or non-existence of certain attributes, or uses pre-defined attributes of object prototype (such as hasOwnProperty, toString or valueOf). Possible mitigations might be: freezing the object prototype, using an object without prototypes (via Object.create(null) ), blocking modifications of attributes that resolve to object prototype, using Map instead of object. \\
\hline
\end{tabular}
\end{table}
\vspace{0.7cm}
\FloatBarrier
\begin{table}[!h]
\centering
\renewcommand{\arraystretch}{1.3}
\begin{tabular}{|l|p{10cm}|}
\hline
\multicolumn{2}{|c|}{\textbf{Vulnerability 68}} \\
\hline
\multicolumn{2}{|l|}{\textbf{Path}: ScrapedFiles/login/main/conversion.js} \\
\hline
\textbf{Vulnerability Class} & ['Denial-of-Service (DoS)'] \\
\hline
\textbf{Start} & line: 34 \quad col: 97 \\
\hline
\textbf{End} & line: 34 \quad col: 134 \\
\hline
\textbf{Message} & RegExp() called with a `a` function argument, this might allow an attacker to cause a Regular Expression Denial-of-Service (ReDoS) within your application as RegExP blocks the main thread. For this reason, it is recommended to use hardcoded regexes instead. If your regex is run on user-controlled input, consider performing input validation or use a regex checking/sanitization library such as https://www.npmjs.com/package/recheck to verify that the regex does not appear vulnerable to ReDoS. \\
\hline
\end{tabular}
\end{table}
\vspace{0.7cm}
\FloatBarrier
\begin{table}[!h]
\centering
\renewcommand{\arraystretch}{1.3}
\begin{tabular}{|l|p{10cm}|}
\hline
\multicolumn{2}{|c|}{\textbf{Vulnerability 69}} \\
\hline
\multicolumn{2}{|l|}{\textbf{Path}: ScrapedFiles/login/main/conversion.js} \\
\hline
\textbf{Vulnerability Class} & ['Denial-of-Service (DoS)'] \\
\hline
\textbf{Start} & line: 39 \quad col: 1267 \\
\hline
\textbf{End} & line: 39 \quad col: 1331 \\
\hline
\textbf{Message} & RegExp() called with a `a` function argument, this might allow an attacker to cause a Regular Expression Denial-of-Service (ReDoS) within your application as RegExP blocks the main thread. For this reason, it is recommended to use hardcoded regexes instead. If your regex is run on user-controlled input, consider performing input validation or use a regex checking/sanitization library such as https://www.npmjs.com/package/recheck to verify that the regex does not appear vulnerable to ReDoS. \\
\hline
\end{tabular}
\end{table}
\vspace{0.7cm}
\FloatBarrier
\begin{table}[!h]
\centering
\renewcommand{\arraystretch}{1.3}
\begin{tabular}{|l|p{10cm}|}
\hline
\multicolumn{2}{|c|}{\textbf{Vulnerability 70}} \\
\hline
\multicolumn{2}{|l|}{\textbf{Path}: ScrapedFiles/login/main/conversion.js} \\
\hline
\textbf{Vulnerability Class} & ['Denial-of-Service (DoS)'] \\
\hline
\textbf{Start} & line: 79 \quad col: 343 \\
\hline
\textbf{End} & line: 79 \quad col: 372 \\
\hline
\textbf{Message} & RegExp() called with a `b` function argument, this might allow an attacker to cause a Regular Expression Denial-of-Service (ReDoS) within your application as RegExP blocks the main thread. For this reason, it is recommended to use hardcoded regexes instead. If your regex is run on user-controlled input, consider performing input validation or use a regex checking/sanitization library such as https://www.npmjs.com/package/recheck to verify that the regex does not appear vulnerable to ReDoS. \\
\hline
\end{tabular}
\end{table}
\vspace{0.7cm}
\FloatBarrier
\begin{table}[!h]
\centering
\renewcommand{\arraystretch}{1.3}
\begin{tabular}{|l|p{10cm}|}
\hline
\multicolumn{2}{|c|}{\textbf{Vulnerability 71}} \\
\hline
\multicolumn{2}{|l|}{\textbf{Path}: ScrapedFiles/login/main/conversion.js} \\
\hline
\textbf{Vulnerability Class} & ['Cross-Site-Scripting (XSS)'] \\
\hline
\textbf{Start} & line: 117 \quad col: 382 \\
\hline
\textbf{End} & line: 117 \quad col: 399 \\
\hline
\textbf{Message} & User controlled data in methods like `innerHTML`, `outerHTML` or `document.write` is an anti-pattern that can lead to XSS vulnerabilities \\
\hline
\end{tabular}
\end{table}
\vspace{0.7cm}
\FloatBarrier
\begin{table}[!h]
\centering
\renewcommand{\arraystretch}{1.3}
\begin{tabular}{|l|p{10cm}|}
\hline
\multicolumn{2}{|c|}{\textbf{Vulnerability 72}} \\
\hline
\multicolumn{2}{|l|}{\textbf{Path}: ScrapedFiles/login/main/login.html} \\
\hline
\textbf{Vulnerability Class} & ['Cryptographic Issues'] \\
\hline
\textbf{Start} & line: 58 \quad col: 141 \\
\hline
\textbf{End} & line: 58 \quad col: 1070 \\
\hline
\textbf{Message} & This tag is missing an 'integrity' subresource integrity attribute. The 'integrity' attribute allows for the browser to verify that externally hosted files (for example from a CDN) are delivered without unexpected manipulation. Without this attribute, if an attacker can modify the externally hosted resource, this could lead to XSS and other types of attacks. To prevent this, include the base64-encoded cryptographic hash of the resource (file) you're telling the browser to fetch in the 'integrity' attribute for all externally hosted files. \\
\hline
\end{tabular}
\end{table}
\vspace{0.7cm}
\FloatBarrier
\begin{table}[!h]
\centering
\renewcommand{\arraystretch}{1.3}
\begin{tabular}{|l|p{10cm}|}
\hline
\multicolumn{2}{|c|}{\textbf{Vulnerability 73}} \\
\hline
\multicolumn{2}{|l|}{\textbf{Path}: ScrapedFiles/login/main/login.html} \\
\hline
\textbf{Vulnerability Class} & ['Cryptographic Issues'] \\
\hline
\textbf{Start} & line: 58 \quad col: 1070 \\
\hline
\textbf{End} & line: 58 \quad col: 1149 \\
\hline
\textbf{Message} & This tag is missing an 'integrity' subresource integrity attribute. The 'integrity' attribute allows for the browser to verify that externally hosted files (for example from a CDN) are delivered without unexpected manipulation. Without this attribute, if an attacker can modify the externally hosted resource, this could lead to XSS and other types of attacks. To prevent this, include the base64-encoded cryptographic hash of the resource (file) you're telling the browser to fetch in the 'integrity' attribute for all externally hosted files. \\
\hline
\end{tabular}
\end{table}
\vspace{0.7cm}
\FloatBarrier
\begin{table}[!h]
\centering
\renewcommand{\arraystretch}{1.3}
\begin{tabular}{|l|p{10cm}|}
\hline
\multicolumn{2}{|c|}{\textbf{Vulnerability 74}} \\
\hline
\multicolumn{2}{|l|}{\textbf{Path}: ScrapedFiles/login/main/login.html} \\
\hline
\textbf{Vulnerability Class} & ['Cross-Site Request Forgery (CSRF)'] \\
\hline
\textbf{Start} & line: 84 \quad col: 25 \\
\hline
\textbf{End} & line: 95 \quad col: 32 \\
\hline
\textbf{Message} & Manually-created forms in django templates should specify a csrf\_token to prevent CSRF attacks. \\
\hline
\end{tabular}
\end{table}
\vspace{0.7cm}
\FloatBarrier
\begin{table}[!h]
\centering
\renewcommand{\arraystretch}{1.3}
\begin{tabular}{|l|p{10cm}|}
\hline
\multicolumn{2}{|c|}{\textbf{Vulnerability 75}} \\
\hline
\multicolumn{2}{|l|}{\textbf{Path}: ScrapedFiles/login/main/login.html} \\
\hline
\textbf{Vulnerability Class} & ['Cryptographic Issues'] \\
\hline
\textbf{Start} & line: 115 \quad col: 3 \\
\hline
\textbf{End} & line: 115 \quad col: 92 \\
\hline
\textbf{Message} & This tag is missing an 'integrity' subresource integrity attribute. The 'integrity' attribute allows for the browser to verify that externally hosted files (for example from a CDN) are delivered without unexpected manipulation. Without this attribute, if an attacker can modify the externally hosted resource, this could lead to XSS and other types of attacks. To prevent this, include the base64-encoded cryptographic hash of the resource (file) you're telling the browser to fetch in the 'integrity' attribute for all externally hosted files. \\
\hline
\end{tabular}
\end{table}
\vspace{0.7cm}
\FloatBarrier
\begin{table}[!h]
\centering
\renewcommand{\arraystretch}{1.3}
\begin{tabular}{|l|p{10cm}|}
\hline
\multicolumn{2}{|c|}{\textbf{Vulnerability 76}} \\
\hline
\multicolumn{2}{|l|}{\textbf{Path}: ScrapedFiles/login/main/login.html} \\
\hline
\textbf{Vulnerability Class} & ['Cryptographic Issues'] \\
\hline
\textbf{Start} & line: 120 \quad col: 3 \\
\hline
\textbf{End} & line: 120 \quad col: 99 \\
\hline
\textbf{Message} & This tag is missing an 'integrity' subresource integrity attribute. The 'integrity' attribute allows for the browser to verify that externally hosted files (for example from a CDN) are delivered without unexpected manipulation. Without this attribute, if an attacker can modify the externally hosted resource, this could lead to XSS and other types of attacks. To prevent this, include the base64-encoded cryptographic hash of the resource (file) you're telling the browser to fetch in the 'integrity' attribute for all externally hosted files. \\
\hline
\end{tabular}
\end{table}
\vspace{0.7cm}
\FloatBarrier
\begin{table}[!h]
\centering
\renewcommand{\arraystretch}{1.3}
\begin{tabular}{|l|p{10cm}|}
\hline
\multicolumn{2}{|c|}{\textbf{Vulnerability 77}} \\
\hline
\multicolumn{2}{|l|}{\textbf{Path}: ScrapedFiles/login/main/login.html} \\
\hline
\textbf{Vulnerability Class} & ['Cryptographic Issues'] \\
\hline
\textbf{Start} & line: 121 \quad col: 3 \\
\hline
\textbf{End} & line: 121 \quad col: 123 \\
\hline
\textbf{Message} & This tag is missing an 'integrity' subresource integrity attribute. The 'integrity' attribute allows for the browser to verify that externally hosted files (for example from a CDN) are delivered without unexpected manipulation. Without this attribute, if an attacker can modify the externally hosted resource, this could lead to XSS and other types of attacks. To prevent this, include the base64-encoded cryptographic hash of the resource (file) you're telling the browser to fetch in the 'integrity' attribute for all externally hosted files. \\
\hline
\end{tabular}
\end{table}
\vspace{0.7cm}
\FloatBarrier
\begin{table}[!h]
\centering
\renewcommand{\arraystretch}{1.3}
\begin{tabular}{|l|p{10cm}|}
\hline
\multicolumn{2}{|c|}{\textbf{Vulnerability 78}} \\
\hline
\multicolumn{2}{|l|}{\textbf{Path}: ScrapedFiles/login/main/login.html} \\
\hline
\textbf{Vulnerability Class} & ['Cryptographic Issues'] \\
\hline
\textbf{Start} & line: 189 \quad col: 5 \\
\hline
\textbf{End} & line: 190 \quad col: 14 \\
\hline
\textbf{Message} & This tag is missing an 'integrity' subresource integrity attribute. The 'integrity' attribute allows for the browser to verify that externally hosted files (for example from a CDN) are delivered without unexpected manipulation. Without this attribute, if an attacker can modify the externally hosted resource, this could lead to XSS and other types of attacks. To prevent this, include the base64-encoded cryptographic hash of the resource (file) you're telling the browser to fetch in the 'integrity' attribute for all externally hosted files. \\
\hline
\end{tabular}
\end{table}
\vspace{0.7cm}
\FloatBarrier
\begin{table}[!h]
\centering
\renewcommand{\arraystretch}{1.3}
\begin{tabular}{|l|p{10cm}|}
\hline
\multicolumn{2}{|c|}{\textbf{Vulnerability 79}} \\
\hline
\multicolumn{2}{|l|}{\textbf{Path}: ScrapedFiles/login/main/login.html} \\
\hline
\textbf{Vulnerability Class} & ['Cryptographic Issues'] \\
\hline
\textbf{Start} & line: 190 \quad col: 14 \\
\hline
\textbf{End} & line: 190 \quad col: 679 \\
\hline
\textbf{Message} & This tag is missing an 'integrity' subresource integrity attribute. The 'integrity' attribute allows for the browser to verify that externally hosted files (for example from a CDN) are delivered without unexpected manipulation. Without this attribute, if an attacker can modify the externally hosted resource, this could lead to XSS and other types of attacks. To prevent this, include the base64-encoded cryptographic hash of the resource (file) you're telling the browser to fetch in the 'integrity' attribute for all externally hosted files. \\
\hline
\end{tabular}
\end{table}
\vspace{0.7cm}
\FloatBarrier
\begin{table}[!h]
\centering
\renewcommand{\arraystretch}{1.3}
\begin{tabular}{|l|p{10cm}|}
\hline
\multicolumn{2}{|c|}{\textbf{Vulnerability 80}} \\
\hline
\multicolumn{2}{|l|}{\textbf{Path}: ScrapedFiles/login/main/login.html} \\
\hline
\textbf{Vulnerability Class} & ['Cryptographic Issues'] \\
\hline
\textbf{Start} & line: 207 \quad col: 1 \\
\hline
\textbf{End} & line: 207 \quad col: 664 \\
\hline
\textbf{Message} & This tag is missing an 'integrity' subresource integrity attribute. The 'integrity' attribute allows for the browser to verify that externally hosted files (for example from a CDN) are delivered without unexpected manipulation. Without this attribute, if an attacker can modify the externally hosted resource, this could lead to XSS and other types of attacks. To prevent this, include the base64-encoded cryptographic hash of the resource (file) you're telling the browser to fetch in the 'integrity' attribute for all externally hosted files. \\
\hline
\end{tabular}
\end{table}
\vspace{0.7cm}
\FloatBarrier
\begin{table}[!h]
\centering
\renewcommand{\arraystretch}{1.3}
\begin{tabular}{|l|p{10cm}|}
\hline
\multicolumn{2}{|c|}{\textbf{Vulnerability 81}} \\
\hline
\multicolumn{2}{|l|}{\textbf{Path}: ScrapedFiles/pages/forms/main/analytics.js} \\
\hline
\textbf{Vulnerability Class} & ['Denial-of-Service (DoS)'] \\
\hline
\textbf{Start} & line: 10 \quad col: 486 \\
\hline
\textbf{End} & line: 10 \quad col: 529 \\
\hline
\textbf{Message} & RegExp() called with a `a` function argument, this might allow an attacker to cause a Regular Expression Denial-of-Service (ReDoS) within your application as RegExP blocks the main thread. For this reason, it is recommended to use hardcoded regexes instead. If your regex is run on user-controlled input, consider performing input validation or use a regex checking/sanitization library such as https://www.npmjs.com/package/recheck to verify that the regex does not appear vulnerable to ReDoS. \\
\hline
\end{tabular}
\end{table}
\vspace{0.7cm}
\FloatBarrier
\begin{table}[!h]
\centering
\renewcommand{\arraystretch}{1.3}
\begin{tabular}{|l|p{10cm}|}
\hline
\multicolumn{2}{|c|}{\textbf{Vulnerability 82}} \\
\hline
\multicolumn{2}{|l|}{\textbf{Path}: ScrapedFiles/pages/forms/main/analytics.js} \\
\hline
\textbf{Vulnerability Class} & ['Denial-of-Service (DoS)'] \\
\hline
\textbf{Start} & line: 27 \quad col: 150 \\
\hline
\textbf{End} & line: 27 \quad col: 189 \\
\hline
\textbf{Message} & RegExp() called with a `a` function argument, this might allow an attacker to cause a Regular Expression Denial-of-Service (ReDoS) within your application as RegExP blocks the main thread. For this reason, it is recommended to use hardcoded regexes instead. If your regex is run on user-controlled input, consider performing input validation or use a regex checking/sanitization library such as https://www.npmjs.com/package/recheck to verify that the regex does not appear vulnerable to ReDoS. \\
\hline
\end{tabular}
\end{table}
\vspace{0.7cm}
\FloatBarrier
\begin{table}[!h]
\centering
\renewcommand{\arraystretch}{1.3}
\begin{tabular}{|l|p{10cm}|}
\hline
\multicolumn{2}{|c|}{\textbf{Vulnerability 83}} \\
\hline
\multicolumn{2}{|l|}{\textbf{Path}: ScrapedFiles/pages/forms/main/analytics.js} \\
\hline
\textbf{Vulnerability Class} & ['Denial-of-Service (DoS)'] \\
\hline
\textbf{Start} & line: 28 \quad col: 304 \\
\hline
\textbf{End} & line: 28 \quad col: 368 \\
\hline
\textbf{Message} & RegExp() called with a `a` function argument, this might allow an attacker to cause a Regular Expression Denial-of-Service (ReDoS) within your application as RegExP blocks the main thread. For this reason, it is recommended to use hardcoded regexes instead. If your regex is run on user-controlled input, consider performing input validation or use a regex checking/sanitization library such as https://www.npmjs.com/package/recheck to verify that the regex does not appear vulnerable to ReDoS. \\
\hline
\end{tabular}
\end{table}
\vspace{0.7cm}
\FloatBarrier
\begin{table}[!h]
\centering
\renewcommand{\arraystretch}{1.3}
\begin{tabular}{|l|p{10cm}|}
\hline
\multicolumn{2}{|c|}{\textbf{Vulnerability 84}} \\
\hline
\multicolumn{2}{|l|}{\textbf{Path}: ScrapedFiles/pages/forms/main/analytics.js} \\
\hline
\textbf{Vulnerability Class} & ['Improper Encoding'] \\
\hline
\textbf{Start} & line: 37 \quad col: 130 \\
\hline
\textbf{End} & line: 37 \quad col: 184 \\
\hline
\textbf{Message} & `"https://www.google.\%/ads/ga-audiences".replace` method will only replace the first occurrence when used with a string argument ("\%"). If this method is used for escaping of dangerous data then there is a possibility for a bypass. Try to use sanitization library instead or use a Regex with a global flag. \\
\hline
\end{tabular}
\end{table}
\vspace{0.7cm}
\FloatBarrier
\begin{table}[!h]
\centering
\renewcommand{\arraystretch}{1.3}
\begin{tabular}{|l|p{10cm}|}
\hline
\multicolumn{2}{|c|}{\textbf{Vulnerability 85}} \\
\hline
\multicolumn{2}{|l|}{\textbf{Path}: ScrapedFiles/pages/forms/main/analytics.js} \\
\hline
\textbf{Vulnerability Class} & ['Denial-of-Service (DoS)'] \\
\hline
\textbf{Start} & line: 48 \quad col: 629 \\
\hline
\textbf{End} & line: 48 \quad col: 650 \\
\hline
\textbf{Message} & RegExp() called with a `a` function argument, this might allow an attacker to cause a Regular Expression Denial-of-Service (ReDoS) within your application as RegExP blocks the main thread. For this reason, it is recommended to use hardcoded regexes instead. If your regex is run on user-controlled input, consider performing input validation or use a regex checking/sanitization library such as https://www.npmjs.com/package/recheck to verify that the regex does not appear vulnerable to ReDoS. \\
\hline
\end{tabular}
\end{table}
\vspace{0.7cm}
\FloatBarrier
\begin{table}[!h]
\centering
\renewcommand{\arraystretch}{1.3}
\begin{tabular}{|l|p{10cm}|}
\hline
\multicolumn{2}{|c|}{\textbf{Vulnerability 86}} \\
\hline
\multicolumn{2}{|l|}{\textbf{Path}: ScrapedFiles/pages/main/analytics.js} \\
\hline
\textbf{Vulnerability Class} & ['Denial-of-Service (DoS)'] \\
\hline
\textbf{Start} & line: 10 \quad col: 486 \\
\hline
\textbf{End} & line: 10 \quad col: 529 \\
\hline
\textbf{Message} & RegExp() called with a `a` function argument, this might allow an attacker to cause a Regular Expression Denial-of-Service (ReDoS) within your application as RegExP blocks the main thread. For this reason, it is recommended to use hardcoded regexes instead. If your regex is run on user-controlled input, consider performing input validation or use a regex checking/sanitization library such as https://www.npmjs.com/package/recheck to verify that the regex does not appear vulnerable to ReDoS. \\
\hline
\end{tabular}
\end{table}
\vspace{0.7cm}
\FloatBarrier
\begin{table}[!h]
\centering
\renewcommand{\arraystretch}{1.3}
\begin{tabular}{|l|p{10cm}|}
\hline
\multicolumn{2}{|c|}{\textbf{Vulnerability 87}} \\
\hline
\multicolumn{2}{|l|}{\textbf{Path}: ScrapedFiles/pages/main/analytics.js} \\
\hline
\textbf{Vulnerability Class} & ['Denial-of-Service (DoS)'] \\
\hline
\textbf{Start} & line: 27 \quad col: 150 \\
\hline
\textbf{End} & line: 27 \quad col: 189 \\
\hline
\textbf{Message} & RegExp() called with a `a` function argument, this might allow an attacker to cause a Regular Expression Denial-of-Service (ReDoS) within your application as RegExP blocks the main thread. For this reason, it is recommended to use hardcoded regexes instead. If your regex is run on user-controlled input, consider performing input validation or use a regex checking/sanitization library such as https://www.npmjs.com/package/recheck to verify that the regex does not appear vulnerable to ReDoS. \\
\hline
\end{tabular}
\end{table}
\vspace{0.7cm}
\FloatBarrier
\begin{table}[!h]
\centering
\renewcommand{\arraystretch}{1.3}
\begin{tabular}{|l|p{10cm}|}
\hline
\multicolumn{2}{|c|}{\textbf{Vulnerability 88}} \\
\hline
\multicolumn{2}{|l|}{\textbf{Path}: ScrapedFiles/pages/main/analytics.js} \\
\hline
\textbf{Vulnerability Class} & ['Denial-of-Service (DoS)'] \\
\hline
\textbf{Start} & line: 28 \quad col: 304 \\
\hline
\textbf{End} & line: 28 \quad col: 368 \\
\hline
\textbf{Message} & RegExp() called with a `a` function argument, this might allow an attacker to cause a Regular Expression Denial-of-Service (ReDoS) within your application as RegExP blocks the main thread. For this reason, it is recommended to use hardcoded regexes instead. If your regex is run on user-controlled input, consider performing input validation or use a regex checking/sanitization library such as https://www.npmjs.com/package/recheck to verify that the regex does not appear vulnerable to ReDoS. \\
\hline
\end{tabular}
\end{table}
\vspace{0.7cm}
\FloatBarrier
\begin{table}[!h]
\centering
\renewcommand{\arraystretch}{1.3}
\begin{tabular}{|l|p{10cm}|}
\hline
\multicolumn{2}{|c|}{\textbf{Vulnerability 89}} \\
\hline
\multicolumn{2}{|l|}{\textbf{Path}: ScrapedFiles/pages/main/analytics.js} \\
\hline
\textbf{Vulnerability Class} & ['Improper Encoding'] \\
\hline
\textbf{Start} & line: 37 \quad col: 130 \\
\hline
\textbf{End} & line: 37 \quad col: 184 \\
\hline
\textbf{Message} & `"https://www.google.\%/ads/ga-audiences".replace` method will only replace the first occurrence when used with a string argument ("\%"). If this method is used for escaping of dangerous data then there is a possibility for a bypass. Try to use sanitization library instead or use a Regex with a global flag. \\
\hline
\end{tabular}
\end{table}
\vspace{0.7cm}
\FloatBarrier
\begin{table}[!h]
\centering
\renewcommand{\arraystretch}{1.3}
\begin{tabular}{|l|p{10cm}|}
\hline
\multicolumn{2}{|c|}{\textbf{Vulnerability 90}} \\
\hline
\multicolumn{2}{|l|}{\textbf{Path}: ScrapedFiles/pages/main/analytics.js} \\
\hline
\textbf{Vulnerability Class} & ['Denial-of-Service (DoS)'] \\
\hline
\textbf{Start} & line: 48 \quad col: 629 \\
\hline
\textbf{End} & line: 48 \quad col: 650 \\
\hline
\textbf{Message} & RegExp() called with a `a` function argument, this might allow an attacker to cause a Regular Expression Denial-of-Service (ReDoS) within your application as RegExP blocks the main thread. For this reason, it is recommended to use hardcoded regexes instead. If your regex is run on user-controlled input, consider performing input validation or use a regex checking/sanitization library such as https://www.npmjs.com/package/recheck to verify that the regex does not appear vulnerable to ReDoS. \\
\hline
\end{tabular}
\end{table}
\vspace{0.7cm}
\FloatBarrier
\begin{table}[!h]
\centering
\renewcommand{\arraystretch}{1.3}
\begin{tabular}{|l|p{10cm}|}
\hline
\multicolumn{2}{|c|}{\textbf{Vulnerability 91}} \\
\hline
\multicolumn{2}{|l|}{\textbf{Path}: ScrapedFiles/pages/main/conversion.js} \\
\hline
\textbf{Vulnerability Class} & ['Mass Assignment'] \\
\hline
\textbf{Start} & line: 3 \quad col: 152 \\
\hline
\textbf{End} & line: 3 \quad col: 158 \\
\hline
\textbf{Message} & Possibility of prototype polluting function detected. By adding or modifying attributes of an object prototype, it is possible to create attributes that exist on every object, or replace critical attributes with malicious ones. This can be problematic if the software depends on existence or non-existence of certain attributes, or uses pre-defined attributes of object prototype (such as hasOwnProperty, toString or valueOf). Possible mitigations might be: freezing the object prototype, using an object without prototypes (via Object.create(null) ), blocking modifications of attributes that resolve to object prototype, using Map instead of object. \\
\hline
\end{tabular}
\end{table}
\vspace{0.7cm}
\FloatBarrier
\begin{table}[!h]
\centering
\renewcommand{\arraystretch}{1.3}
\begin{tabular}{|l|p{10cm}|}
\hline
\multicolumn{2}{|c|}{\textbf{Vulnerability 92}} \\
\hline
\multicolumn{2}{|l|}{\textbf{Path}: ScrapedFiles/pages/main/conversion.js} \\
\hline
\textbf{Vulnerability Class} & ['Denial-of-Service (DoS)'] \\
\hline
\textbf{Start} & line: 34 \quad col: 97 \\
\hline
\textbf{End} & line: 34 \quad col: 134 \\
\hline
\textbf{Message} & RegExp() called with a `a` function argument, this might allow an attacker to cause a Regular Expression Denial-of-Service (ReDoS) within your application as RegExP blocks the main thread. For this reason, it is recommended to use hardcoded regexes instead. If your regex is run on user-controlled input, consider performing input validation or use a regex checking/sanitization library such as https://www.npmjs.com/package/recheck to verify that the regex does not appear vulnerable to ReDoS. \\
\hline
\end{tabular}
\end{table}
\vspace{0.7cm}
\FloatBarrier
\begin{table}[!h]
\centering
\renewcommand{\arraystretch}{1.3}
\begin{tabular}{|l|p{10cm}|}
\hline
\multicolumn{2}{|c|}{\textbf{Vulnerability 93}} \\
\hline
\multicolumn{2}{|l|}{\textbf{Path}: ScrapedFiles/pages/main/conversion.js} \\
\hline
\textbf{Vulnerability Class} & ['Denial-of-Service (DoS)'] \\
\hline
\textbf{Start} & line: 39 \quad col: 1267 \\
\hline
\textbf{End} & line: 39 \quad col: 1331 \\
\hline
\textbf{Message} & RegExp() called with a `a` function argument, this might allow an attacker to cause a Regular Expression Denial-of-Service (ReDoS) within your application as RegExP blocks the main thread. For this reason, it is recommended to use hardcoded regexes instead. If your regex is run on user-controlled input, consider performing input validation or use a regex checking/sanitization library such as https://www.npmjs.com/package/recheck to verify that the regex does not appear vulnerable to ReDoS. \\
\hline
\end{tabular}
\end{table}
\vspace{0.7cm}
\FloatBarrier
\begin{table}[!h]
\centering
\renewcommand{\arraystretch}{1.3}
\begin{tabular}{|l|p{10cm}|}
\hline
\multicolumn{2}{|c|}{\textbf{Vulnerability 94}} \\
\hline
\multicolumn{2}{|l|}{\textbf{Path}: ScrapedFiles/pages/main/conversion.js} \\
\hline
\textbf{Vulnerability Class} & ['Denial-of-Service (DoS)'] \\
\hline
\textbf{Start} & line: 79 \quad col: 343 \\
\hline
\textbf{End} & line: 79 \quad col: 372 \\
\hline
\textbf{Message} & RegExp() called with a `b` function argument, this might allow an attacker to cause a Regular Expression Denial-of-Service (ReDoS) within your application as RegExP blocks the main thread. For this reason, it is recommended to use hardcoded regexes instead. If your regex is run on user-controlled input, consider performing input validation or use a regex checking/sanitization library such as https://www.npmjs.com/package/recheck to verify that the regex does not appear vulnerable to ReDoS. \\
\hline
\end{tabular}
\end{table}
\vspace{0.7cm}
\FloatBarrier
\begin{table}[!h]
\centering
\renewcommand{\arraystretch}{1.3}
\begin{tabular}{|l|p{10cm}|}
\hline
\multicolumn{2}{|c|}{\textbf{Vulnerability 95}} \\
\hline
\multicolumn{2}{|l|}{\textbf{Path}: ScrapedFiles/pages/main/conversion.js} \\
\hline
\textbf{Vulnerability Class} & ['Cross-Site-Scripting (XSS)'] \\
\hline
\textbf{Start} & line: 117 \quad col: 382 \\
\hline
\textbf{End} & line: 117 \quad col: 399 \\
\hline
\textbf{Message} & User controlled data in methods like `innerHTML`, `outerHTML` or `document.write` is an anti-pattern that can lead to XSS vulnerabilities \\
\hline
\end{tabular}
\end{table}
\vspace{0.7cm}
\FloatBarrier
\begin{table}[!h]
\centering
\renewcommand{\arraystretch}{1.3}
\begin{tabular}{|l|p{10cm}|}
\hline
\multicolumn{2}{|c|}{\textbf{Vulnerability 96}} \\
\hline
\multicolumn{2}{|l|}{\textbf{Path}: ScrapedFiles/pages/main/pages.html} \\
\hline
\textbf{Vulnerability Class} & ['Cryptographic Issues'] \\
\hline
\textbf{Start} & line: 58 \quad col: 141 \\
\hline
\textbf{End} & line: 58 \quad col: 1070 \\
\hline
\textbf{Message} & This tag is missing an 'integrity' subresource integrity attribute. The 'integrity' attribute allows for the browser to verify that externally hosted files (for example from a CDN) are delivered without unexpected manipulation. Without this attribute, if an attacker can modify the externally hosted resource, this could lead to XSS and other types of attacks. To prevent this, include the base64-encoded cryptographic hash of the resource (file) you're telling the browser to fetch in the 'integrity' attribute for all externally hosted files. \\
\hline
\end{tabular}
\end{table}
\vspace{0.7cm}
\FloatBarrier
\begin{table}[!h]
\centering
\renewcommand{\arraystretch}{1.3}
\begin{tabular}{|l|p{10cm}|}
\hline
\multicolumn{2}{|c|}{\textbf{Vulnerability 97}} \\
\hline
\multicolumn{2}{|l|}{\textbf{Path}: ScrapedFiles/pages/main/pages.html} \\
\hline
\textbf{Vulnerability Class} & ['Cryptographic Issues'] \\
\hline
\textbf{Start} & line: 58 \quad col: 1070 \\
\hline
\textbf{End} & line: 58 \quad col: 1149 \\
\hline
\textbf{Message} & This tag is missing an 'integrity' subresource integrity attribute. The 'integrity' attribute allows for the browser to verify that externally hosted files (for example from a CDN) are delivered without unexpected manipulation. Without this attribute, if an attacker can modify the externally hosted resource, this could lead to XSS and other types of attacks. To prevent this, include the base64-encoded cryptographic hash of the resource (file) you're telling the browser to fetch in the 'integrity' attribute for all externally hosted files. \\
\hline
\end{tabular}
\end{table}
\vspace{0.7cm}
\FloatBarrier
\begin{table}[!h]
\centering
\renewcommand{\arraystretch}{1.3}
\begin{tabular}{|l|p{10cm}|}
\hline
\multicolumn{2}{|c|}{\textbf{Vulnerability 98}} \\
\hline
\multicolumn{2}{|l|}{\textbf{Path}: ScrapedFiles/pages/main/pages.html} \\
\hline
\textbf{Vulnerability Class} & ['Cryptographic Issues'] \\
\hline
\textbf{Start} & line: 155 \quad col: 3 \\
\hline
\textbf{End} & line: 155 \quad col: 92 \\
\hline
\textbf{Message} & This tag is missing an 'integrity' subresource integrity attribute. The 'integrity' attribute allows for the browser to verify that externally hosted files (for example from a CDN) are delivered without unexpected manipulation. Without this attribute, if an attacker can modify the externally hosted resource, this could lead to XSS and other types of attacks. To prevent this, include the base64-encoded cryptographic hash of the resource (file) you're telling the browser to fetch in the 'integrity' attribute for all externally hosted files. \\
\hline
\end{tabular}
\end{table}
\vspace{0.7cm}
\FloatBarrier
\begin{table}[!h]
\centering
\renewcommand{\arraystretch}{1.3}
\begin{tabular}{|l|p{10cm}|}
\hline
\multicolumn{2}{|c|}{\textbf{Vulnerability 99}} \\
\hline
\multicolumn{2}{|l|}{\textbf{Path}: ScrapedFiles/pages/main/pages.html} \\
\hline
\textbf{Vulnerability Class} & ['Cryptographic Issues'] \\
\hline
\textbf{Start} & line: 160 \quad col: 3 \\
\hline
\textbf{End} & line: 160 \quad col: 99 \\
\hline
\textbf{Message} & This tag is missing an 'integrity' subresource integrity attribute. The 'integrity' attribute allows for the browser to verify that externally hosted files (for example from a CDN) are delivered without unexpected manipulation. Without this attribute, if an attacker can modify the externally hosted resource, this could lead to XSS and other types of attacks. To prevent this, include the base64-encoded cryptographic hash of the resource (file) you're telling the browser to fetch in the 'integrity' attribute for all externally hosted files. \\
\hline
\end{tabular}
\end{table}
\vspace{0.7cm}
\FloatBarrier
\begin{table}[!h]
\centering
\renewcommand{\arraystretch}{1.3}
\begin{tabular}{|l|p{10cm}|}
\hline
\multicolumn{2}{|c|}{\textbf{Vulnerability 100}} \\
\hline
\multicolumn{2}{|l|}{\textbf{Path}: ScrapedFiles/pages/main/pages.html} \\
\hline
\textbf{Vulnerability Class} & ['Cryptographic Issues'] \\
\hline
\textbf{Start} & line: 161 \quad col: 3 \\
\hline
\textbf{End} & line: 161 \quad col: 123 \\
\hline
\textbf{Message} & This tag is missing an 'integrity' subresource integrity attribute. The 'integrity' attribute allows for the browser to verify that externally hosted files (for example from a CDN) are delivered without unexpected manipulation. Without this attribute, if an attacker can modify the externally hosted resource, this could lead to XSS and other types of attacks. To prevent this, include the base64-encoded cryptographic hash of the resource (file) you're telling the browser to fetch in the 'integrity' attribute for all externally hosted files. \\
\hline
\end{tabular}
\end{table}
\vspace{0.7cm}
\FloatBarrier
\begin{table}[!h]
\centering
\renewcommand{\arraystretch}{1.3}
\begin{tabular}{|l|p{10cm}|}
\hline
\multicolumn{2}{|c|}{\textbf{Vulnerability 101}} \\
\hline
\multicolumn{2}{|l|}{\textbf{Path}: ScrapedFiles/pages/main/pages.html} \\
\hline
\textbf{Vulnerability Class} & ['Cryptographic Issues'] \\
\hline
\textbf{Start} & line: 229 \quad col: 5 \\
\hline
\textbf{End} & line: 230 \quad col: 14 \\
\hline
\textbf{Message} & This tag is missing an 'integrity' subresource integrity attribute. The 'integrity' attribute allows for the browser to verify that externally hosted files (for example from a CDN) are delivered without unexpected manipulation. Without this attribute, if an attacker can modify the externally hosted resource, this could lead to XSS and other types of attacks. To prevent this, include the base64-encoded cryptographic hash of the resource (file) you're telling the browser to fetch in the 'integrity' attribute for all externally hosted files. \\
\hline
\end{tabular}
\end{table}
\vspace{0.7cm}
\FloatBarrier
\begin{table}[!h]
\centering
\renewcommand{\arraystretch}{1.3}
\begin{tabular}{|l|p{10cm}|}
\hline
\multicolumn{2}{|c|}{\textbf{Vulnerability 102}} \\
\hline
\multicolumn{2}{|l|}{\textbf{Path}: ScrapedFiles/pages/main/pages.html} \\
\hline
\textbf{Vulnerability Class} & ['Cryptographic Issues'] \\
\hline
\textbf{Start} & line: 230 \quad col: 14 \\
\hline
\textbf{End} & line: 230 \quad col: 696 \\
\hline
\textbf{Message} & This tag is missing an 'integrity' subresource integrity attribute. The 'integrity' attribute allows for the browser to verify that externally hosted files (for example from a CDN) are delivered without unexpected manipulation. Without this attribute, if an attacker can modify the externally hosted resource, this could lead to XSS and other types of attacks. To prevent this, include the base64-encoded cryptographic hash of the resource (file) you're telling the browser to fetch in the 'integrity' attribute for all externally hosted files. \\
\hline
\end{tabular}
\end{table}
\vspace{0.7cm}
\FloatBarrier
\begin{table}[!h]
\centering
\renewcommand{\arraystretch}{1.3}
\begin{tabular}{|l|p{10cm}|}
\hline
\multicolumn{2}{|c|}{\textbf{Vulnerability 103}} \\
\hline
\multicolumn{2}{|l|}{\textbf{Path}: ScrapedFiles/pages/main/pages.html} \\
\hline
\textbf{Vulnerability Class} & ['Cryptographic Issues'] \\
\hline
\textbf{Start} & line: 247 \quad col: 1 \\
\hline
\textbf{End} & line: 247 \quad col: 681 \\
\hline
\textbf{Message} & This tag is missing an 'integrity' subresource integrity attribute. The 'integrity' attribute allows for the browser to verify that externally hosted files (for example from a CDN) are delivered without unexpected manipulation. Without this attribute, if an attacker can modify the externally hosted resource, this could lead to XSS and other types of attacks. To prevent this, include the base64-encoded cryptographic hash of the resource (file) you're telling the browser to fetch in the 'integrity' attribute for all externally hosted files. \\
\hline
\end{tabular}
\end{table}
\vspace{0.7cm}
\FloatBarrier
\begin{table}[!h]
\centering
\renewcommand{\arraystretch}{1.3}
\begin{tabular}{|l|p{10cm}|}
\hline
\multicolumn{2}{|c|}{\textbf{Vulnerability 104}} \\
\hline
\multicolumn{2}{|l|}{\textbf{Path}: ScrapedFiles/pages/simple/main/analytics.js} \\
\hline
\textbf{Vulnerability Class} & ['Denial-of-Service (DoS)'] \\
\hline
\textbf{Start} & line: 10 \quad col: 486 \\
\hline
\textbf{End} & line: 10 \quad col: 529 \\
\hline
\textbf{Message} & RegExp() called with a `a` function argument, this might allow an attacker to cause a Regular Expression Denial-of-Service (ReDoS) within your application as RegExP blocks the main thread. For this reason, it is recommended to use hardcoded regexes instead. If your regex is run on user-controlled input, consider performing input validation or use a regex checking/sanitization library such as https://www.npmjs.com/package/recheck to verify that the regex does not appear vulnerable to ReDoS. \\
\hline
\end{tabular}
\end{table}
\vspace{0.7cm}
\FloatBarrier
\begin{table}[!h]
\centering
\renewcommand{\arraystretch}{1.3}
\begin{tabular}{|l|p{10cm}|}
\hline
\multicolumn{2}{|c|}{\textbf{Vulnerability 105}} \\
\hline
\multicolumn{2}{|l|}{\textbf{Path}: ScrapedFiles/pages/simple/main/analytics.js} \\
\hline
\textbf{Vulnerability Class} & ['Denial-of-Service (DoS)'] \\
\hline
\textbf{Start} & line: 27 \quad col: 150 \\
\hline
\textbf{End} & line: 27 \quad col: 189 \\
\hline
\textbf{Message} & RegExp() called with a `a` function argument, this might allow an attacker to cause a Regular Expression Denial-of-Service (ReDoS) within your application as RegExP blocks the main thread. For this reason, it is recommended to use hardcoded regexes instead. If your regex is run on user-controlled input, consider performing input validation or use a regex checking/sanitization library such as https://www.npmjs.com/package/recheck to verify that the regex does not appear vulnerable to ReDoS. \\
\hline
\end{tabular}
\end{table}
\vspace{0.7cm}
\FloatBarrier
\begin{table}[!h]
\centering
\renewcommand{\arraystretch}{1.3}
\begin{tabular}{|l|p{10cm}|}
\hline
\multicolumn{2}{|c|}{\textbf{Vulnerability 106}} \\
\hline
\multicolumn{2}{|l|}{\textbf{Path}: ScrapedFiles/pages/simple/main/analytics.js} \\
\hline
\textbf{Vulnerability Class} & ['Denial-of-Service (DoS)'] \\
\hline
\textbf{Start} & line: 28 \quad col: 304 \\
\hline
\textbf{End} & line: 28 \quad col: 368 \\
\hline
\textbf{Message} & RegExp() called with a `a` function argument, this might allow an attacker to cause a Regular Expression Denial-of-Service (ReDoS) within your application as RegExP blocks the main thread. For this reason, it is recommended to use hardcoded regexes instead. If your regex is run on user-controlled input, consider performing input validation or use a regex checking/sanitization library such as https://www.npmjs.com/package/recheck to verify that the regex does not appear vulnerable to ReDoS. \\
\hline
\end{tabular}
\end{table}
\vspace{0.7cm}
\FloatBarrier
\begin{table}[!h]
\centering
\renewcommand{\arraystretch}{1.3}
\begin{tabular}{|l|p{10cm}|}
\hline
\multicolumn{2}{|c|}{\textbf{Vulnerability 107}} \\
\hline
\multicolumn{2}{|l|}{\textbf{Path}: ScrapedFiles/pages/simple/main/analytics.js} \\
\hline
\textbf{Vulnerability Class} & ['Improper Encoding'] \\
\hline
\textbf{Start} & line: 37 \quad col: 130 \\
\hline
\textbf{End} & line: 37 \quad col: 184 \\
\hline
\textbf{Message} & `"https://www.google.\%/ads/ga-audiences".replace` method will only replace the first occurrence when used with a string argument ("\%"). If this method is used for escaping of dangerous data then there is a possibility for a bypass. Try to use sanitization library instead or use a Regex with a global flag. \\
\hline
\end{tabular}
\end{table}
\vspace{0.7cm}
\FloatBarrier
\begin{table}[!h]
\centering
\renewcommand{\arraystretch}{1.3}
\begin{tabular}{|l|p{10cm}|}
\hline
\multicolumn{2}{|c|}{\textbf{Vulnerability 108}} \\
\hline
\multicolumn{2}{|l|}{\textbf{Path}: ScrapedFiles/pages/simple/main/analytics.js} \\
\hline
\textbf{Vulnerability Class} & ['Denial-of-Service (DoS)'] \\
\hline
\textbf{Start} & line: 48 \quad col: 629 \\
\hline
\textbf{End} & line: 48 \quad col: 650 \\
\hline
\textbf{Message} & RegExp() called with a `a` function argument, this might allow an attacker to cause a Regular Expression Denial-of-Service (ReDoS) within your application as RegExP blocks the main thread. For this reason, it is recommended to use hardcoded regexes instead. If your regex is run on user-controlled input, consider performing input validation or use a regex checking/sanitization library such as https://www.npmjs.com/package/recheck to verify that the regex does not appear vulnerable to ReDoS. \\
\hline
\end{tabular}
\end{table}
\vspace{0.7cm}
\FloatBarrier
\begin{table}[!h]
\centering
\renewcommand{\arraystretch}{1.3}
\begin{tabular}{|l|p{10cm}|}
\hline
\multicolumn{2}{|c|}{\textbf{Vulnerability 109}} \\
\hline
\multicolumn{2}{|l|}{\textbf{Path}: ScrapedFiles/pages/simple/main/conversion.js} \\
\hline
\textbf{Vulnerability Class} & ['Mass Assignment'] \\
\hline
\textbf{Start} & line: 3 \quad col: 152 \\
\hline
\textbf{End} & line: 3 \quad col: 158 \\
\hline
\textbf{Message} & Possibility of prototype polluting function detected. By adding or modifying attributes of an object prototype, it is possible to create attributes that exist on every object, or replace critical attributes with malicious ones. This can be problematic if the software depends on existence or non-existence of certain attributes, or uses pre-defined attributes of object prototype (such as hasOwnProperty, toString or valueOf). Possible mitigations might be: freezing the object prototype, using an object without prototypes (via Object.create(null) ), blocking modifications of attributes that resolve to object prototype, using Map instead of object. \\
\hline
\end{tabular}
\end{table}
\vspace{0.7cm}
\FloatBarrier
\begin{table}[!h]
\centering
\renewcommand{\arraystretch}{1.3}
\begin{tabular}{|l|p{10cm}|}
\hline
\multicolumn{2}{|c|}{\textbf{Vulnerability 110}} \\
\hline
\multicolumn{2}{|l|}{\textbf{Path}: ScrapedFiles/pages/simple/main/conversion.js} \\
\hline
\textbf{Vulnerability Class} & ['Denial-of-Service (DoS)'] \\
\hline
\textbf{Start} & line: 34 \quad col: 97 \\
\hline
\textbf{End} & line: 34 \quad col: 134 \\
\hline
\textbf{Message} & RegExp() called with a `a` function argument, this might allow an attacker to cause a Regular Expression Denial-of-Service (ReDoS) within your application as RegExP blocks the main thread. For this reason, it is recommended to use hardcoded regexes instead. If your regex is run on user-controlled input, consider performing input validation or use a regex checking/sanitization library such as https://www.npmjs.com/package/recheck to verify that the regex does not appear vulnerable to ReDoS. \\
\hline
\end{tabular}
\end{table}
\vspace{0.7cm}
\FloatBarrier
\begin{table}[!h]
\centering
\renewcommand{\arraystretch}{1.3}
\begin{tabular}{|l|p{10cm}|}
\hline
\multicolumn{2}{|c|}{\textbf{Vulnerability 111}} \\
\hline
\multicolumn{2}{|l|}{\textbf{Path}: ScrapedFiles/pages/simple/main/conversion.js} \\
\hline
\textbf{Vulnerability Class} & ['Denial-of-Service (DoS)'] \\
\hline
\textbf{Start} & line: 39 \quad col: 1267 \\
\hline
\textbf{End} & line: 39 \quad col: 1331 \\
\hline
\textbf{Message} & RegExp() called with a `a` function argument, this might allow an attacker to cause a Regular Expression Denial-of-Service (ReDoS) within your application as RegExP blocks the main thread. For this reason, it is recommended to use hardcoded regexes instead. If your regex is run on user-controlled input, consider performing input validation or use a regex checking/sanitization library such as https://www.npmjs.com/package/recheck to verify that the regex does not appear vulnerable to ReDoS. \\
\hline
\end{tabular}
\end{table}
\vspace{0.7cm}
\FloatBarrier
\begin{table}[!h]
\centering
\renewcommand{\arraystretch}{1.3}
\begin{tabular}{|l|p{10cm}|}
\hline
\multicolumn{2}{|c|}{\textbf{Vulnerability 112}} \\
\hline
\multicolumn{2}{|l|}{\textbf{Path}: ScrapedFiles/pages/simple/main/conversion.js} \\
\hline
\textbf{Vulnerability Class} & ['Denial-of-Service (DoS)'] \\
\hline
\textbf{Start} & line: 79 \quad col: 343 \\
\hline
\textbf{End} & line: 79 \quad col: 372 \\
\hline
\textbf{Message} & RegExp() called with a `b` function argument, this might allow an attacker to cause a Regular Expression Denial-of-Service (ReDoS) within your application as RegExP blocks the main thread. For this reason, it is recommended to use hardcoded regexes instead. If your regex is run on user-controlled input, consider performing input validation or use a regex checking/sanitization library such as https://www.npmjs.com/package/recheck to verify that the regex does not appear vulnerable to ReDoS. \\
\hline
\end{tabular}
\end{table}
\vspace{0.7cm}
\FloatBarrier
\begin{table}[!h]
\centering
\renewcommand{\arraystretch}{1.3}
\begin{tabular}{|l|p{10cm}|}
\hline
\multicolumn{2}{|c|}{\textbf{Vulnerability 113}} \\
\hline
\multicolumn{2}{|l|}{\textbf{Path}: ScrapedFiles/pages/simple/main/conversion.js} \\
\hline
\textbf{Vulnerability Class} & ['Cross-Site-Scripting (XSS)'] \\
\hline
\textbf{Start} & line: 117 \quad col: 382 \\
\hline
\textbf{End} & line: 117 \quad col: 399 \\
\hline
\textbf{Message} & User controlled data in methods like `innerHTML`, `outerHTML` or `document.write` is an anti-pattern that can lead to XSS vulnerabilities \\
\hline
\end{tabular}
\end{table}
\vspace{0.7cm}
\FloatBarrier
\begin{table}[!h]
\centering
\renewcommand{\arraystretch}{1.3}
\begin{tabular}{|l|p{10cm}|}
\hline
\multicolumn{2}{|c|}{\textbf{Vulnerability 114}} \\
\hline
\multicolumn{2}{|l|}{\textbf{Path}: ScrapedFiles/pages/simple/main/simple.html} \\
\hline
\textbf{Vulnerability Class} & ['Cryptographic Issues'] \\
\hline
\textbf{Start} & line: 15 \quad col: 1 \\
\hline
\textbf{End} & line: 15 \quad col: 87 \\
\hline
\textbf{Message} & This tag is missing an 'integrity' subresource integrity attribute. The 'integrity' attribute allows for the browser to verify that externally hosted files (for example from a CDN) are delivered without unexpected manipulation. Without this attribute, if an attacker can modify the externally hosted resource, this could lead to XSS and other types of attacks. To prevent this, include the base64-encoded cryptographic hash of the resource (file) you're telling the browser to fetch in the 'integrity' attribute for all externally hosted files. \\
\hline
\end{tabular}
\end{table}
\vspace{0.7cm}
\FloatBarrier
\begin{table}[!h]
\centering
\renewcommand{\arraystretch}{1.3}
\begin{tabular}{|l|p{10cm}|}
\hline
\multicolumn{2}{|c|}{\textbf{Vulnerability 115}} \\
\hline
\multicolumn{2}{|l|}{\textbf{Path}: ScrapedFiles/pages/simple/main/simple.html} \\
\hline
\textbf{Vulnerability Class} & ['Cryptographic Issues'] \\
\hline
\textbf{Start} & line: 61 \quad col: 141 \\
\hline
\textbf{End} & line: 61 \quad col: 1070 \\
\hline
\textbf{Message} & This tag is missing an 'integrity' subresource integrity attribute. The 'integrity' attribute allows for the browser to verify that externally hosted files (for example from a CDN) are delivered without unexpected manipulation. Without this attribute, if an attacker can modify the externally hosted resource, this could lead to XSS and other types of attacks. To prevent this, include the base64-encoded cryptographic hash of the resource (file) you're telling the browser to fetch in the 'integrity' attribute for all externally hosted files. \\
\hline
\end{tabular}
\end{table}
\vspace{0.7cm}
\FloatBarrier
\begin{table}[!h]
\centering
\renewcommand{\arraystretch}{1.3}
\begin{tabular}{|l|p{10cm}|}
\hline
\multicolumn{2}{|c|}{\textbf{Vulnerability 116}} \\
\hline
\multicolumn{2}{|l|}{\textbf{Path}: ScrapedFiles/pages/simple/main/simple.html} \\
\hline
\textbf{Vulnerability Class} & ['Cryptographic Issues'] \\
\hline
\textbf{Start} & line: 61 \quad col: 1070 \\
\hline
\textbf{End} & line: 61 \quad col: 1149 \\
\hline
\textbf{Message} & This tag is missing an 'integrity' subresource integrity attribute. The 'integrity' attribute allows for the browser to verify that externally hosted files (for example from a CDN) are delivered without unexpected manipulation. Without this attribute, if an attacker can modify the externally hosted resource, this could lead to XSS and other types of attacks. To prevent this, include the base64-encoded cryptographic hash of the resource (file) you're telling the browser to fetch in the 'integrity' attribute for all externally hosted files. \\
\hline
\end{tabular}
\end{table}
\vspace{0.7cm}
\FloatBarrier
\begin{table}[!h]
\centering
\renewcommand{\arraystretch}{1.3}
\begin{tabular}{|l|p{10cm}|}
\hline
\multicolumn{2}{|c|}{\textbf{Vulnerability 117}} \\
\hline
\multicolumn{2}{|l|}{\textbf{Path}: ScrapedFiles/pages/simple/main/simple.html} \\
\hline
\textbf{Vulnerability Class} & ['Mishandled Sensitive Information'] \\
\hline
\textbf{Start} & line: 114 \quad col: 33 \\
\hline
\textbf{End} & line: 114 \quad col: 161 \\
\hline
\textbf{Message} & This link points to a plaintext HTTP URL. Prefer an encrypted HTTPS URL if possible. \\
\hline
\end{tabular}
\end{table}
\vspace{0.7cm}
\FloatBarrier
\begin{table}[!h]
\centering
\renewcommand{\arraystretch}{1.3}
\begin{tabular}{|l|p{10cm}|}
\hline
\multicolumn{2}{|c|}{\textbf{Vulnerability 118}} \\
\hline
\multicolumn{2}{|l|}{\textbf{Path}: ScrapedFiles/pages/simple/main/simple.html} \\
\hline
\textbf{Vulnerability Class} & ['Cryptographic Issues'] \\
\hline
\textbf{Start} & line: 3892 \quad col: 3 \\
\hline
\textbf{End} & line: 3892 \quad col: 92 \\
\hline
\textbf{Message} & This tag is missing an 'integrity' subresource integrity attribute. The 'integrity' attribute allows for the browser to verify that externally hosted files (for example from a CDN) are delivered without unexpected manipulation. Without this attribute, if an attacker can modify the externally hosted resource, this could lead to XSS and other types of attacks. To prevent this, include the base64-encoded cryptographic hash of the resource (file) you're telling the browser to fetch in the 'integrity' attribute for all externally hosted files. \\
\hline
\end{tabular}
\end{table}
\vspace{0.7cm}
\FloatBarrier
\begin{table}[!h]
\centering
\renewcommand{\arraystretch}{1.3}
\begin{tabular}{|l|p{10cm}|}
\hline
\multicolumn{2}{|c|}{\textbf{Vulnerability 119}} \\
\hline
\multicolumn{2}{|l|}{\textbf{Path}: ScrapedFiles/pages/simple/main/simple.html} \\
\hline
\textbf{Vulnerability Class} & ['Cryptographic Issues'] \\
\hline
\textbf{Start} & line: 3897 \quad col: 3 \\
\hline
\textbf{End} & line: 3897 \quad col: 99 \\
\hline
\textbf{Message} & This tag is missing an 'integrity' subresource integrity attribute. The 'integrity' attribute allows for the browser to verify that externally hosted files (for example from a CDN) are delivered without unexpected manipulation. Without this attribute, if an attacker can modify the externally hosted resource, this could lead to XSS and other types of attacks. To prevent this, include the base64-encoded cryptographic hash of the resource (file) you're telling the browser to fetch in the 'integrity' attribute for all externally hosted files. \\
\hline
\end{tabular}
\end{table}
\vspace{0.7cm}
\FloatBarrier
\begin{table}[!h]
\centering
\renewcommand{\arraystretch}{1.3}
\begin{tabular}{|l|p{10cm}|}
\hline
\multicolumn{2}{|c|}{\textbf{Vulnerability 120}} \\
\hline
\multicolumn{2}{|l|}{\textbf{Path}: ScrapedFiles/pages/simple/main/simple.html} \\
\hline
\textbf{Vulnerability Class} & ['Cryptographic Issues'] \\
\hline
\textbf{Start} & line: 3898 \quad col: 3 \\
\hline
\textbf{End} & line: 3898 \quad col: 123 \\
\hline
\textbf{Message} & This tag is missing an 'integrity' subresource integrity attribute. The 'integrity' attribute allows for the browser to verify that externally hosted files (for example from a CDN) are delivered without unexpected manipulation. Without this attribute, if an attacker can modify the externally hosted resource, this could lead to XSS and other types of attacks. To prevent this, include the base64-encoded cryptographic hash of the resource (file) you're telling the browser to fetch in the 'integrity' attribute for all externally hosted files. \\
\hline
\end{tabular}
\end{table}
\vspace{0.7cm}
\FloatBarrier
\begin{table}[!h]
\centering
\renewcommand{\arraystretch}{1.3}
\begin{tabular}{|l|p{10cm}|}
\hline
\multicolumn{2}{|c|}{\textbf{Vulnerability 121}} \\
\hline
\multicolumn{2}{|l|}{\textbf{Path}: ScrapedFiles/pages/simple/main/simple.html} \\
\hline
\textbf{Vulnerability Class} & ['Cryptographic Issues'] \\
\hline
\textbf{Start} & line: 3966 \quad col: 5 \\
\hline
\textbf{End} & line: 3967 \quad col: 14 \\
\hline
\textbf{Message} & This tag is missing an 'integrity' subresource integrity attribute. The 'integrity' attribute allows for the browser to verify that externally hosted files (for example from a CDN) are delivered without unexpected manipulation. Without this attribute, if an attacker can modify the externally hosted resource, this could lead to XSS and other types of attacks. To prevent this, include the base64-encoded cryptographic hash of the resource (file) you're telling the browser to fetch in the 'integrity' attribute for all externally hosted files. \\
\hline
\end{tabular}
\end{table}
\vspace{0.7cm}
\FloatBarrier
\begin{table}[!h]
\centering
\renewcommand{\arraystretch}{1.3}
\begin{tabular}{|l|p{10cm}|}
\hline
\multicolumn{2}{|c|}{\textbf{Vulnerability 122}} \\
\hline
\multicolumn{2}{|l|}{\textbf{Path}: ScrapedFiles/pages/simple/main/simple.html} \\
\hline
\textbf{Vulnerability Class} & ['Cryptographic Issues'] \\
\hline
\textbf{Start} & line: 3967 \quad col: 14 \\
\hline
\textbf{End} & line: 3967 \quad col: 725 \\
\hline
\textbf{Message} & This tag is missing an 'integrity' subresource integrity attribute. The 'integrity' attribute allows for the browser to verify that externally hosted files (for example from a CDN) are delivered without unexpected manipulation. Without this attribute, if an attacker can modify the externally hosted resource, this could lead to XSS and other types of attacks. To prevent this, include the base64-encoded cryptographic hash of the resource (file) you're telling the browser to fetch in the 'integrity' attribute for all externally hosted files. \\
\hline
\end{tabular}
\end{table}
\vspace{0.7cm}
\FloatBarrier
\begin{table}[!h]
\centering
\renewcommand{\arraystretch}{1.3}
\begin{tabular}{|l|p{10cm}|}
\hline
\multicolumn{2}{|c|}{\textbf{Vulnerability 123}} \\
\hline
\multicolumn{2}{|l|}{\textbf{Path}: ScrapedFiles/pages/simple/main/simple.html} \\
\hline
\textbf{Vulnerability Class} & ['Cryptographic Issues'] \\
\hline
\textbf{Start} & line: 3984 \quad col: 1 \\
\hline
\textbf{End} & line: 3984 \quad col: 722 \\
\hline
\textbf{Message} & This tag is missing an 'integrity' subresource integrity attribute. The 'integrity' attribute allows for the browser to verify that externally hosted files (for example from a CDN) are delivered without unexpected manipulation. Without this attribute, if an attacker can modify the externally hosted resource, this could lead to XSS and other types of attacks. To prevent this, include the base64-encoded cryptographic hash of the resource (file) you're telling the browser to fetch in the 'integrity' attribute for all externally hosted files. \\
\hline
\end{tabular}
\end{table}
\vspace{0.7cm}
\FloatBarrier


%--------------------------------------------------------Dynamic Analysis-----------------------------------------------
\newpage
\section{Dynamic Analysis}
Details about dynamic analysis...

\section{Analysis Report}

\subsection{Nmap Scan Results}

\subsection*{SQLMap Injection Points}
Injection Point 1:
\begin{itemize}
    \item \textbf{Parameter:} artist (GET)
        \subitem Type: boolean-based blind
        \subitem Title: AND boolean-based blind - WHERE or HAVING clause
        \subitem 
        \subitem Type: time-based blind
        \subitem Title: MySQL >= 5.0.12 AND time-based blind (query SLEEP)
        \subitem 
        \subitem Type: UNION query
        \subitem Title: MySQL UNION query (NULL) - 3 columns
\end{itemize}

\subsection{Risk Summary}
\begin{table}[h!]
\centering
\renewcommand{\arraystretch}{1.5}
\begin{tabular}{|c|c|}
\hline
\textbf{Risk Level} & \textbf{Number of Findings} \\
\hline
Low Risk & zaplc \\ 
\hline
Medium Risk & zapmc \\ 
\hline
High Risk & zaphc \\ 
\hline
Critical Risk & zapcc \\ 
\hline
\end{tabular}
\caption{Summary of Risk Findings}
\label{tab:risk_summary}
\end{table}

\subsection{Vulnerability Categories}
\begin{itemize}
\item ZapCategories:
% Insert vulnerability categories here
\end{itemize}

\subsection{Vulnerabilities by Page}
%ZapVulnerabilities by Page:
\end{document}
